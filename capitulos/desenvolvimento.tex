\section{Tipo de Pesquisa e Etapas de Construção}
       A pesquisa caracteriza-se como aplicada, de abordagem mista, com natureza exploratória e experimental. Inicialmente, foi conduzido um estudo teórico sobre os princípios da automação residencial, sensores e controladores IoT, buscando embasar a proposta de automação de piscinas. As informações foram coletadas em artigos científicos, manuais técnicos e normas da ABNT e ANVISA, possibilitando compreender os processos de tratamento e manutenção da água, bem como os requisitos técnicos para integração de hardware e software.

    % \subsection{CONCEITO E FUNCIONAMENTO GERAL}
    %     Aqui eu colo o processo de limpeza de uma piscina automatizada, listando vários tipo de automação e incluindo o do meu projeto(talvez).

    % \subsection{DIFERENÇAS ENTRE PROCESSOS MANUAIS E AUTOMATIZADOS}

    %     \textcolor{red}{Desenvolver com base em estudos técnicos (aprodutos existentes como sodramar, nautilus, etc.) Citar exemplos de bombas inteligentes, ozonizadores, sensores de pH automatizados.}
    
    %     Aqui como mais acima eu já expliquei sobre a automação manual e automatizada, eu rapidamente relembro e depois discorro pontos positivos e negativos de ambas com o fim de comparação, sempre tentando enaltecer a automatizar, afinal é de fato melhor e é isso que eu pretendo provar.
    
    % \subsection{TECNOLOGIAS ESPECÍFICAS USADAS NA AUTOMAÇÃO DE PISCINAS}
    %     \textcolor{red}{Novo subtópico importante, aqui o aluno pode falar sobre: 
    %     Sensores de ph e orp
    %     medidores de turbidez
    %     bombas peristalticas para dosagem automática
    %     controladores (espn32, raspberry Pi, etc...)}

    % \subsection{ASPECTOS SANITÁRIOS E SAÚDE PÚBLICA}
    %     \textcolor{red}{Ampliar com base na vigilância sanitária e riscos da má manutenção de piscinas. Sugestão: normas ANVISA ou artigos sobre dermatites, otites, doenças bacterianas e parasitárias}

    % \subsection{ACESSIBILIDADE E DEMOCRATIZAÇÃO DA AUTOMAÇÃO}
    %    %\textcolor{red}{Reescrever com liguagme técnica e respeitosa. Sugerir expressões como: "Viabilidade técnica e econômica para população com menor poder aquisitivo", "proposta de custo acessível para residências de médio padrão"}


\section{Processo de Desenvolvimento de Software}
    Apresentar o RUP:

    Explicar que o RUP é um processo de engenharia de software estruturado em fases iterativas e incrementais;

    Justificar o uso: o RUP favorece documentação clara, validação de requisitos, prototipagem contínua e testes progressivos — ideal para projetos acadêmicos.

    As quatro fases do RUP:

    Concepção (Inception) – definição do problema e dos requisitos gerais;

    Elaboração (Elaboration) – modelagem do sistema, arquitetura e protótipos;

    Construção (Construction) – codificação e integração dos módulos;

    Transição (Transition) – implantação, testes e ajustes finais.

\section{Fase de Concepção (\textit{Inception})}
    Identificação do problema: dificuldade de limpeza manual, desperdício de água e produtos químicos. Objetivo do sistema: automatizar o processo de filtragem e tratamento químico com sensores inteligentes.

    Requisitos iniciais:

    Monitoramento de pH, turbidez e temperatura da água;

    Acionamento automático da bomba;

    Interface web para visualização dos parâmetros;

    Comunicação entre sensores e Raspberry Pi.

    Ferramentas e tecnologias selecionadas: Raspberry Pi, sensores de pH, turbidez e temperatura, Node.js, React, e banco de dados SQLite.

    Lá encima é o que é. Aqui é o como

\section{Fase de Elaboração (\textit{Elaboration})}
    Modelagem dos casos de uso:
    Criar um diagrama simples mostrando as interações — Usuário → Interface Web → Controlador → Sensores e Atuadores.

    Arquitetura do sistema:
    Explicar o fluxo de dados:

    Sensores capturam informações (pH, turbidez, temperatura);

    Raspberry Pi processa e envia dados ao banco;

    Interface web exibe os resultados e permite comandos manuais.
    
    Prototipagem inicial:
    Descrever testes iniciais dos sensores em bancada para validação da leitura de dados.

    Prototipagem inicial:
    Descrever testes iniciais dos sensores em bancada para validação da leitura de dados.

\section{Fase de contrução (\textit{Construction})}
    Codificação e integração:

    Implementação dos scripts em Python para leitura dos sensores no Raspberry Pi;

    Comunicação com o servidor via API (HTTP/MQTT);

    Interface web em React para monitoramento em tempo real.

    Testes unitários e integração:
    Explicar que foram realizados testes de calibração dos sensores e de resposta dos atuadores.

    Documentação:
    Inserir diagramas e trechos de código comentados.

\section{Fase de Transição (\textit{Transition})}
    Implantação e testes práticos:
    Teste do sistema em uma piscina real ou simulação controlada;

    Avaliação de desempenho:
    Medir tempo de resposta, precisão dos sensores e consumo elétrico;

    Ajustes finais:
    Calibração dos sensores e correção de erros observados.


\section{Integração entre o Raspberry Pi e os Sensores}
    Esquema de ligação física:
    Apresentar diagrama elétrico mostrando conexões GPIO → sensores → relés → atuadores.

    Descrição técnica:

    Sensor de pH (entrada analógica via ADC);

    Sensor de temperatura (protocolo 1-Wire);

    Sensor de turbidez (entrada analógica);

    Módulo de relé (saída digital para controle da bomba).

    Comunicação de dados:
    Explicar como o Raspberry Pi coleta os valores, processa e envia via rede Wi-Fi para a aplicação web.

    Segurança:
    Isolamento elétrico e uso de fontes separadas.