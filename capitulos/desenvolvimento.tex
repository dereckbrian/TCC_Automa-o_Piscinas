\section{Tipo de Pesquisa e Etapas de Construção}
       A pesquisa caracteriza-se como aplicada, de abordagem mista, com natureza exploratória e experimental. Inicialmente, foi conduzido um estudo teórico sobre os princípios da automação residencial, sensores e controladores IoT, buscando embasar a proposta de automação de piscinas. As informações foram coletadas em artigos científicos, manuais técnicos e normas da ABNT e ANVISA, possibilitando compreender os processos de tratamento e manutenção da água, bem como os requisitos técnicos para integração de hardware e software.

    % \subsection{CONCEITO E FUNCIONAMENTO GERAL}
    %     Aqui eu colo o processo de limpeza de uma piscina automatizada, listando vários tipo de automação e incluindo o do meu projeto(talvez).

    % \subsection{DIFERENÇAS ENTRE PROCESSOS MANUAIS E AUTOMATIZADOS}

    %     \textcolor{red}{Desenvolver com base em estudos técnicos (aprodutos existentes como sodramar, nautilus, etc.) Citar exemplos de bombas inteligentes, ozonizadores, sensores de pH automatizados.}
    
    %     Aqui como mais acima eu já expliquei sobre a automação manual e automatizada, eu rapidamente relembro e depois discorro pontos positivos e negativos de ambas com o fim de comparação, sempre tentando enaltecer a automatizar, afinal é de fato melhor e é isso que eu pretendo provar.
    
    % \subsection{TECNOLOGIAS ESPECÍFICAS USADAS NA AUTOMAÇÃO DE PISCINAS}
    %     \textcolor{red}{Novo subtópico importante, aqui o aluno pode falar sobre: 
    %     Sensores de ph e orp
    %     medidores de turbidez
    %     bombas peristalticas para dosagem automática
    %     controladores (espn32, raspberry Pi, etc...)}

    % \subsection{ASPECTOS SANITÁRIOS E SAÚDE PÚBLICA}
    %     \textcolor{red}{Ampliar com base na vigilância sanitária e riscos da má manutenção de piscinas. Sugestão: normas ANVISA ou artigos sobre dermatites, otites, doenças bacterianas e parasitárias}

    % \subsection{ACESSIBILIDADE E DEMOCRATIZAÇÃO DA AUTOMAÇÃO}
    %    %\textcolor{red}{Reescrever com liguagme técnica e respeitosa. Sugerir expressões como: "Viabilidade técnica e econômica para população com menor poder aquisitivo", "proposta de custo acessível para residências de médio padrão"}


\section{Processo de Desenvolvimento de Software}
    % Apresentar o RUP:

    % Explicar que o RUP é um processo de engenharia de software estruturado em fases iterativas e incrementais;

    % Justificar o uso: o RUP favorece documentação clara, validação de requisitos, prototipagem contínua e testes progressivos — ideal para projetos acadêmicos.

    % As quatro fases do RUP:

    % Concepção (Inception) – definição do problema e dos requisitos gerais;

    % Elaboração (Elaboration) – modelagem do sistema, arquitetura e protótipos;

    % Construção (Construction) – codificação e integração dos módulos;

    % Transição (Transition) – implantação, testes e ajustes finais.

    No desenvolvimento deste projeto, foi utilizada a metodologia de desenvolvimento RUP (Rational Unified Process), que, conforme descrito nas seções anteriores, trata-se de um processo de engenharia de software com estrutura bem definida, cujo objetivo é reduzir riscos e tornar o desenvolvimento do projeto mais eficiente, por meio de sua abordagem iterativa e incremental.

    A escolha do RUP deve-se à sua documentação clara, à validação contínua de requisitos, à prototipagem incremental e aos testes progressivos, características ideais para projetos acadêmicos. O modelo é composto por quatro etapas principais: Fase de Concepção, Fase de Elaboração, Fase de Construção e Fase de Transição, que serão detalhadas nas seções subsequentes.

\section{Fase de Concepção (\textit{Inception})}
    % Identificação do problema: dificuldade de limpeza manual, desperdício de água e produtos químicos. Objetivo do sistema: automatizar o processo de filtragem e tratamento químico com sensores inteligentes.

    % Requisitos iniciais:

    % Monitoramento de pH, turbidez e temperatura da água;

    % Acionamento automático da bomba;

    % Interface web para visualização dos parâmetros;

    % Comunicação entre sensores e Raspberry Pi.

    % Ferramentas e tecnologias selecionadas: Raspberry Pi, sensores de pH, turbidez e temperatura, Node.js, React, e banco de dados SQLite.

    % Lá encima é o que é. Aqui é o como

    Foi identificada uma dificuldade na limpeza e manutenção manual de piscinas. O problema ocorre devido à complexidade na coleta das informações sobre a qualidade da água e na identificação dos produtos adequados para o tratamento, exigindo do usuário um conhecimento prévio sobre diversos parâmetros e cálculos. Além disso, há o risco de desperdício de produtos químicos, decorrente da falta de precisão nas dosagens e da ausência de um controle automatizado.

    O sistema de automação para limpeza e manutenção de piscinas surge como uma solução para esses desafios, possibilitando a coleta automatizada dos dados, a análise das informações e a indicação precisa dos produtos e quantidades adequadas. Dessa forma, o sistema busca facilitar o processo, otimizar o uso de recursos e reduzir o desperdício de insumos.

    Diante do exposto, foi realizado o levantamento dos requisitos iniciais para o desenvolvimento do sistema. São eles: monitoramento automático de pH, turbidez, temperatura e nível da água; acionamento automático da bomba; e uma interface web integrada aos componentes, permitindo a visualização dos parâmetros coletados. 
    
    As ferramentas e tecnologias utilizadas incluem o Raspberry Pi como microcontrolador, os sensores mencionados, o React como framework front-end, o PostgreSQL para armazenamento de dados, e o Spring Boot para o desenvolvimento do back-end da aplicação. Logo após foi definido os requisitos funcionais e não funcionais do sistemas:

    \begin{table}[H]
        \centering
        \caption{Requisitos Funcionais do Sistema de Automação de Piscinas.}
        \label{tab:req_funcionais}
        \begin{tabular}{|p{1.5cm}|p{11cm}|}
        \hline
        \textbf{Código} & \textbf{Descrição} \\ \hline
        \textbf{RF01} & O sistema deve monitorar automaticamente os níveis de pH, turbidez, temperatura e nível da água. \\ \hline
        \textbf{RF02} & O sistema deve acionar automaticamente a bomba de filtragem e o aquecedor conforme os parâmetros definidos ou coletados pelos sensores. \\ \hline
        \textbf{RF03} & O sistema deve disponibilizar uma interface web para visualização dos parâmetros monitorados. \\ \hline
        \textbf{RF04} & O sistema deve permitir o cadastro e armazenamento dos dados coletados no banco de dados. \\ \hline
        \textbf{RF05} & O sistema deve permitir o cadastro de piscinas vinculadas a usuários. \\ \hline
        \textbf{RF06} & O sistema deve emitir alertas quando algum parâmetro ultrapassar o limite ideal. \\ \hline
        \textbf{RF06} & O sistema deve emitir gráficos de acordo com os ultimos dados de determinadas parâmetros coletados. \\ \hline
        \end{tabular}
    \end{table}

        \vspace{0.5cm}

    \begin{table}[H]
        \centering
        \caption{Requisitos Não Funcionais do Sistema de Automação de Piscinas.}
        \label{tab:req_nao_funcionais}
        \begin{tabular}{|p{1.5cm}|p{11cm}|}
        \hline
        \textbf{Código} & \textbf{Descrição} \\ \hline
        \textbf{RNF01} & O sistema deve utilizar o banco de dados PostgreSQL para armazenamento das informações. \\ \hline
        \textbf{RNF03} & A interface web deve ser responsiva e acessível em dispositivos móveis e desktops. \\ \hline
        \textbf{RNF04} & A comunicação entre o microcontrolador e o servidor deve ocorrer de forma segura, utilizando protocolos HTTP. \\ \hline
        \textbf{RNF05} & O sistema deve ser desenvolvido com o framework Spring Boot no back-end e React no front-end. \\ \hline
        \end{tabular}
    \end{table}

\section{Fase de Elaboração (\textit{Elaboration})}

    Para facilitar a compreensão do funcionamento do sistema, foi elaborado um diagrama de caso de uso com a utilização da UML (Linguagem de Modelagem Unificada). Esse diagrama é essencial para oferecer uma visão clara e objetiva da estrutura do sistema, permitindo que qualquer leitor compreenda, de maneira rápida e eficiente, as funcionalidades envolvidas e a forma como seus componentes se relacionam. A seguir, apresentam-se o diagrama desenvolvido.

    \begin{figure}[H]
        \centering
        \caption{ }  
        \centering
        \label{fig:cont}
        \includegraphics[width=1.00\textwidth]{imagens/meuDiagramaTcc.png}
        \caption*{Diagrama de Caso de Uso}
        \caption*{Fonte: Autor}
    \end{figure}

    Logo após, iniciou-se o processo de definição do fluxo de dados do sistema. Todo o funcionamento começa na coleta de informações realizada pelos sensores instalados na piscina. Esses sensores são conectados a um Arduino, responsável pela leitura inicial dos dados e pelo envio dessas informações ao Raspberry Pi. O Raspberry, por sua vez, mantém a comunicação constante com o \textit{back-end} por meio de requisições HTTP, enviando periodicamente os valores coletados, como a temperatura medida a cada cinco minutos ou alterações no nível da água, e também recebendo comandos vindos do servidor, como a instrução para ligar a bomba.

    Além disso, o sistema conta com um front-end que permite ao usuário acompanhar os dados coletados pelos sensores. A interface oferece, por exemplo, gráficos que exibem a média de temperatura diária ou cartões que são atualizados automaticamente conforme novas leituras são realizadas. O usuário também pode enviar comandos pelo \textit{front-end}, como: ligar a bomba d’água, acionar o filtro, ativar o aquecedor ou a cascata.

    Os sensores utilizados no sistema incluem um sensor de temperatura, um sensor de nível e um sensor de pH. Além disso, duas bombas submersas de 5v \footnote{É uma unidade de medida de potência para motores} serão responsáveis pelas funções de enchimento, filtragem e acionamento da cascata.

    Para complementar a compreensão do funcionamento do sistema, será inserido um fluxograma do fluxo de dados, ilustrando desde a coleta pelo Arduino até a comunicação final com o \textit{back-end} e a visualização pelo usuário.

    Também serão apresentadas fotos dos componentes utilizados, cada uma acompanhada de uma breve explicação sobre sua função e sobre como se conecta ao Arduino. A estrutura sugerida é a seguinte:

    \subsection*{Sensor de Temperatura}

    \begin{figure}[H]
        \centering
        \caption{ }  
        \centering
        \label{fig:cont}
        \includegraphics[width=0.80\textwidth]{imagens/sensorTemperatura.png}
        \caption*{Sensor de Temperatura}
        \caption*{Fonte: \cite{siteComprei}}
    \end{figure}
    O sensor utilizado para a medição da temperatura da água foi o modelo MF58 (NTC 10K à prova d'água). Trata-se de um termistor do tipo NTC (\textit{Negative Temperature Coefficient}), cuja resistência elétrica diminui à medida que a temperatura aumenta.

    A interface com o microcontrolador é realizada de forma analógica. O sensor é conectado em uma configuração de divisor de tensão em série com um resistor de 10k$\Omega$. Dessa forma, a variação de resistência do sensor gera uma variação de tensão correspondente, que é lida por uma das portas de conversão analógico-digital (ADC) do sistema.

    \subsection*{Sensor de Nível}

    \begin{figure}[H]
        \centering
        \caption{ }  
        \centering
        \label{fig:cont}
        \includegraphics[width=0.60\textwidth]{imagens/sensorNive.png}
        \caption*{Sensor de Nível}
        \caption*{Fonte: \cite{siteComprei2}}
    \end{figure}

    Para a detecção do nível de líquido no reservatório, foi selecionado o sensor modelo LC26M-40 da fabricante Eicos. O dispositivo é fabricado em Polipropileno (PP) e opera na vertical, sendo instalado através de um orifício de 16mm.

    O princípio de funcionamento baseia-se na interação entre um flutuador magnético e um contato elétrico interno (\textit{Reed Switch}). Conforme o nível da água movimenta o flutuador, o campo magnético atua sobre o contato, alterando seu estado lógico.

    Ele atua como uma chave elétrica do tipo SPST (\textit{Single Pole Single Throw}), fornecendo um sinal digital binário (\textit{ON/OFF}) ao microcontrolador, indicando se o nível de líquido atingiu o ponto de montagem do sensor.

    \subsection*{Sensor de pH}

    Foto do sensor:
    Este sensor mede o nível de acidez da água. Os dados coletados são tratados inicialmente pelo Arduino e, em seguida, enviados ao back-end para que o sistema avalie possíveis recomendações químicas.

    \subsection*{Bombas Submersas}

    \begin{figure}[H]
        \centering
        \caption{ }  
        \centering
        \label{fig:cont}
        \includegraphics[width=0.60\textwidth]{imagens/bombaSubmersa.png}
        \caption*{Bomba Submersa}
        \caption*{Fonte: \cite{siteComprei}}
    \end{figure}
    
    Foram utilizadas duas bombas submersas de 3 a 5v, modelo JT100, para transportar a água até a piscina, realizar a filtragem e conduzir a água até a cascata. Essas bombas possuem a capacidade de impulsionar de 1000 ml a 1500 ml por minuto e têm um alcance de impulso de até 1 metro de altura. A escolha dessa bomba se deu pela sua facilidade de montagem e integração direta com o Arduino, o que facilita a implementação no sistema de automação.

    A entrada de água ocorre pela frente da bomba, e a saída ocorre por cima, geralmente conectada a uma mangueira de silicone que transporta a água até o local desejado. A entrada possui um diâmetro de 5,5 mm, enquanto a saída tem 4,5 mm. A bomba também conta com um fio à prova d'água, que pode ser ligado a uma fonte externa para fornecer energia.
    \subsection*{Arduino}

    \begin{figure}[H]
        \centering
        \caption{ }  
        \centering
        \label{fig:cont}
        \includegraphics[width=0.60\textwidth]{imagens/arduino.png}
        \caption*{Arduino}
        \caption*{Fonte: \cite{siteComprei4}}
    \end{figure}

    A utilização do Arduino Uno R3 (Modelo ATmega328 e Atmega16U2) como controlador central do sistema se justifica pela sua facilidade de montagem, baixo custo e grande capacidade de integração com sensores e atuadores. Esse modelo foi escolhido para a automação do sistema de manutenção da piscina, pois oferece a robustez necessária para garantir um funcionamento contínuo e eficiente, além de ser altamente compatível com as bibliotecas open source e sistemas de controle que facilitam o desenvolvimento do projeto.

    A Placa Arduino Uno R3, equipada com o microcontrolador Atmega328 (tipo DIP) e o \textit{chip} de interface serial Atmega16U2, opera com uma tensão de 5V e é recomendada para uma tensão de entrada entre 7V e 12V. Possui um total de 14 pinos de entrada/saída, dos quais 6 são PWM, e 6 pinos de entrada analógica, proporcionando flexibilidade para interagir com uma ampla gama de sensores e atuadores. A placa também oferece uma corrente DC de 20mA por pino I/O e uma corrente de 50mA para o pino 3,3V.

    Em termos de memória, o Arduino Uno R3 conta com 32KB de memória \textit{flash}, sendo que 0,5KB são utilizados pelo bootloader. A placa possui 2KB de SRAM e 1KB de EEPROM, o que proporciona um bom espaço para armazenar variáveis temporárias e dados importantes do sistema. O \textit{clock} da placa opera a 16MHz, garantindo um processamento eficiente para executar as funções de controle da automação.

    \subsection*{\textit{Raspberry Pi}}

    \begin{figure}[H]
        \centering
        \caption{ }  
        \centering
        \label{fig:cont}
        \includegraphics[width=0.60\textwidth]{imagens/raspBerryQeuUso.png}
        \caption*{\textit{Raspberry Pi}}
        \caption*{Fonte: \cite{siteComprei3}}
    \end{figure}

    Foi tomada a decisão de utilizar o \textit{Raspberry Pi 3 Model B} como o microcontrolador responsável por tranferir as infomações coletadas pelo arduino para o \textit{back-end}. Esse modelo foi escolhido devido à sua eficiência e custo-benefício, atendendo bem às necessidades do projeto sem comprometer o desempenho.

    Com um processador \textit{Quad-core} de 1.2 GHz e 1 GB de RAM, o \textit{Raspberry Pi} é capaz de executar os processos necessários para coletar dados dos sensores, realizar requisições HTTP e comunicar-se com o \textit{back-end}. Embora não seja um modelo de alta performance, sua capacidade de processar dados em tempo real e interagir com o sistema é adequada ao escopo do projeto, atendendo aos requisitos da fase de construção.

    Além disso, sua compatibilidade com o sistema operacional Linux e a linguagem de programação Python são recursos essenciais que facilitam a implementação das funcionalidades necessárias, como o controle de parâmetros da água e a interação com dispositivos como bombas e aquecedores.

    A escolha do \textit{Raspberry Pi} também reflete uma estratégia de otimização de custos, já que ele oferece baixa manutenção e alta flexibilidade para o desenvolvimento do sistema de automação da piscina.

\section{Fase de contrução (\textit{Construction})}

    Durante esta fase, foi realizada a implementação do sistema, integrando sensores que antes estavam isolados e conectando-os ao Arduino, garantindo que funcionassem em conjunto. Primeiramente, foi implementado e codificado o sensor de temperatura, seguido pela introdução do sensor de nível, que foi conectado à bomba submersa para que ambos trabalhassem de forma coordenada. A lógica implementada foi a seguinte: caso o nível da água desça, o sensor de nível envia um alerta para acionar a bomba; quando o nível da água sobe, o sensor avisa à bomba para desligá-la.

    O código responsável pelo funcionamento dos sensores e atuadores foi desenvolvido utilizando a IDE do Arduino, que é altamente compatível com a linguagem C-like reconhecida pela plataforma. O Arduino foi conectado a uma das portas USB do Raspberry Pi, que, por meio de um código em Python, captura as informações dos sensores e atuadores. Em seguida, essas informações são enviadas ao back-end através de requisições HTTP, para serem salvas no banco de dados e exibidas no front-end. A seção subsequente detalhará melhor o funcionamento do código.

    As linguagens e frameworks escolhidos para o desenvolvimento deste projeto foram selecionados por sua curva de aprendizado rápida. Para o front-end, foi utilizado React, e para o back-end, SpringBoot. A aplicação funciona de maneira que o front-end faz requisições ao back-end via HTTP, e o back-end responde com as informações solicitadas. Em termos práticos, o fluxo de dados funciona da seguinte forma: a temperatura é coletada e enviada ao back-end a cada 5 minutos, sendo salva no banco de dados. Um card no front-end, responsável por exibir essa informação, faz requisições constantes ao back-end para obter a temperatura mais recente, e o back-end responde com os dados atualizados.

    Paralelamente à codificação, foram realizados testes unitários e de integração para validar a confiabilidade dos componentes do sistema. Para o sensor de temperatura (NTC), foi feita uma calibração cruzada, comparando suas leituras com um termômetro digital. Isso permitiu ajustar o parâmetro de \textit{offset} no algoritmo de conversão de resistência para graus Celsius, garantindo maior precisão nas medições.

    No caso do sensor de nível, os testes envolveram a movimentação controlada do flutuador magnético para verificar a precisão da chave Reed Switch. O objetivo foi garantir que a transição entre os estados lógicos (nível baixo/alto) ocorresse de maneira estável, sem variações inesperadas que pudessem causar falsos positivos no sistema.

    Simultaneamente, testes de estresse foram realizados com as bombas submersas para validar a lógica de segurança do sistema. Foi simulado o esvaziamento rápido do reservatório para garantir que o microcontrolador interrompesse imediatamente o funcionamento da bomba ao detectar o nível crítico, prevenindo o funcionamento a seco (\textit{dry-run}) e protegendo o equipamento contra superaquecimento.
    

    % Codificação e integração:

    % Implementação dos scripts em Python para leitura dos sensores no Raspberry Pi;

    % Comunicação com o servidor via API (HTTP/MQTT);

    % Interface web em React para monitoramento em tempo real.

    % Testes unitários e integração:
    % Explicar que foram realizados testes de calibração dos sensores e de resposta dos atuadores.

    % Documentação:
    % Inserir diagramas e trechos de código comentados.

\section{Fase de Transição (\textit{Transition})}
    % Implantação e testes práticos:
    % Teste do sistema em uma piscina real ou simulação controlada;

    % Avaliação de desempenho:
    % Medir tempo de resposta, precisão dos sensores e consumo elétrico;

    % Ajustes finais:
    % Calibração dos sensores e correção de erros observados.


\section{Integração entre o Raspberry Pi e os Sensores}
    % Esquema de ligação física:
    % Apresentar diagrama elétrico mostrando conexões GPIO → sensores → relés → atuadores.

    % Descrição técnica:

    % Sensor de pH (entrada analógica via ADC);

    % Sensor de temperatura (protocolo 1-Wire);

    % Sensor de turbidez (entrada analógica);

    % Módulo de relé (saída digital para controle da bomba).

    % Comunicação de dados:
    % Explicar como o Raspberry Pi coleta os valores, processa e envia via rede Wi-Fi para a aplicação web.

    % Segurança:
    % Isolamento elétrico e uso de fontes separadas.