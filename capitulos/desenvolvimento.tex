\chapter{Desenvolvimento}

\section{Tipo de Pesquisa e Etapas de Construção}

A pesquisa caracteriza-se como aplicada, de abordagem mista e natureza exploratória e experimental. Inicialmente, foi realizado estudo teórico sobre automação residencial, sensores, atuadores e controladores associados à Internet das Coisas (IoT). Essa etapa possibilitou compreensão dos processos físico-químicos envolvidos no tratamento da água, normas sanitárias aplicáveis e requisitos técnicos necessários para integração entre \textit{hardware}, \textit{software} e dispositivos embarcados.

O levantamento bibliográfico incluiu artigos científicos, manuais técnicos e normas da Associação Brasileira de Normas Técnicas (ABNT), assegurando base conceitual e técnica para orientar decisões posteriores de arquitetura, modelagem e implementação do sistema. Os critérios adotados para a seleção das fontes contemplaram: atualidade das informações, priorizando publicações dos últimos dez anos para temas relacionados à tecnologia, automação residencial e Internet das Coisas (IoT); relevância técnica, com a utilização de manuais de fabricantes e guias especializados para fundamentar processos físico-químicos de tratamento da água; e a robustez teórica, com seleção de autores e trabalhos acadêmicos reconhecidos nas áreas de Engenharia de \textit{Software} e arquitetura de sistemas.

Essa etapa permitiu mapear requisitos técnicos necessários à integração entre \textit{hardware}, \textit{software} e dispositivos embarcados, fornecendo subsídios para orientar decisões de projeto adotadas ao longo do desenvolvimento do sistema.

\section{Processo de Desenvolvimento de Software}

Neste projeto, adotou-se o \textit{Rational Unified Process} (RUP) como metodologia de desenvolvimento, em razão de sua estrutura iterativa e incremental, foco na mitigação de riscos e ênfase na modelagem e documentação do \textit{software}. Conforme apresentado na fundamentação teórica, o RUP organiza o processo de engenharia de \textit{software} em quatro fases: Concepção, Elaboração, Construção e Transição. Cada fase possui objetivos específicos que orientam a evolução do sistema, assegurando rastreabilidade entre requisitos, arquitetura e implementação.

Considerando que o sistema desenvolvido envolve automação e dispositivos IoT, o ciclo de vida do desenvolvimento de \textit{hardware} foi adaptado às fases do RUP. Nesse contexto, a definição e seleção dos componentes eletrônicos ocorreram durante a fase de Concepção, enquanto a montagem física e calibração dos sensores foram realizadas de forma integrada à fase de Construção, em paralelo à implementação do software.

A adoção dessa metodologia estabelece fluxo contínuo de desenvolvimento, no qual cada artefato produzido em uma fase serve como base para validação da fase subsequente. Assim, requisitos levantados na fase de Concepção subsidiam a modelagem arquitetural realizada na fase de Elaboração, cujos diagramas orientam a implementação na fase de Construção. A conformidade do sistema desenvolvido é verificada durante a fase de Transição, por meio de testes funcionais e ajustes finais.

As subseções seguintes descrevem cada uma dessas fases, destacando as atividades executadas e sua relação com os artefatos produzidos ao longo do processo.

\section{Fase de Concepção}

Na fase de Concepção foram definidos o problema de pesquisa, escopo inicial do projeto e requisitos fundamentais do sistema. O problema identificado refere-se às dificuldades enfrentadas pelos usuários no processo manual de manutenção de piscinas, decorrentes da necessidade de medições frequentes cálculos físico-químicos, além do risco de desperdício de produtos ocasionado pela aplicação imprecisa das dosagens recomendadas.

Com objetivo de mitigar essas limitações, o sistema foi concebido para automatizar a leitura contínua dos principais parâmetros físico-químicos da água, reduzir a subjetividade na avaliação da qualidade, atuar de forma autônoma na reposição de água com base no sensor de nível e disponibilizar interfaces remotas para monitoramento e acompanhamento do processo de manutenção.

O levantamento de requisitos resultou na definição dos requisitos funcionais (RF) e não funcionais (RNF), apresentados na Tabela \ref{tab:req_funcionais} e Tabela \ref{tab:req_nao_funcionais}, respectivamente.

\begin{table}[H]
    \centering
    \caption{Requisitos Funcionais do Sistema de Automação de Piscinas}
    \label{tab:req_funcionais}
    \begin{tabularx}{\textwidth}{l X}
    \toprule
    \textbf{Código} & \textbf{Descrição} \\
    \midrule
    \textbf{RF01} & O sistema deve monitorar automaticamente valores de pH, turbidez, temperatura e nível da água. \\
    \addlinespace
    \textbf{RF02} & O sistema deve acionar automaticamente a bomba de água conforme os parâmetros coletado pelo sensor de nível. \\
    \addlinespace
    \textbf{RF03} & O sistema deve disponibilizar interface \textit{web} para visualização dos parâmetros monitorados. \\
    \addlinespace
    \textbf{RF04} & O sistema deve permitir cadastro e armazenamento dos dados coletados. \\
    \addlinespace
    \textbf{RF05} & O sistema deve emitir alertas quando algum parâmetro ultrapassar limites estabelecidos. \\
    \addlinespace
    \textbf{RF06} & O sistema deve gerar gráficos com base nos dados históricos dos parâmetros monitorados. \\
    \bottomrule
    \end{tabularx}
    \caption*{Fonte: Autoria própria (2025).}
\end{table}

Para validação dos requisitos definidos, foram estabelecidos critérios quantitativos de aceitação, considerando o desempenho esperado para um protótipo IoT. A precisão dos sensores (RF01) foi definida com margem de erro inferior a $\pm$5\% em relação a medições de referência. Quanto ao desempenho temporal, estabeleceu-se que o tempo de latência total, compreendido entre a aquisição do dado físico pelo microcontrolador e sua visualização na interface \textit{web}, não deve exceder oito segundos, caracterizando monitoração em tempo quase real. No que se refere à comunicação (RNF03), o sistema é considerado estável quando requisições HTTP apresentam taxa de sucesso superior a 95\% em operação contínua, com tempo de resposta do servidor inferior a seis segundos por requisição. O requisito de automação (RF02) é considerado atendido quando o acionamento do atuador ocorrer em até três segundos após detecção da falta de água.

\begin{table}[H]
    \centering
    \caption{Requisitos Não Funcionais do Sistema de Automação de Piscinas}
    \label{tab:req_nao_funcionais}
    \begin{tabularx}{\textwidth}{l X}
    \toprule
    \textbf{Código} & \textbf{Descrição} \\
    \midrule
    \textbf{RNF01} & O sistema deve utilizar SGBD \textit{PostgreSQL} para armazenamento das informações. \\
    \addlinespace
    \textbf{RNF02} & A interface \textit{web} deve ser responsiva e acessível em dispositivos móveis e \textit{desktops}. \\
    \addlinespace
    \textbf{RNF03} & A comunicação entre \textit{gateway} (\textit{Raspberry Pi}) e servidor deve ocorrer utilizando protocolo HTTP. \\
    \addlinespace
    \textbf{RNF04} & O sistema deve ser desenvolvido utilizando o \textit{framework Spring Boot} no \textit{back-end} e \textit{React} no \textit{front-end}. \\
    \bottomrule
    \end{tabularx}
    \caption*{Fonte: Autoria própria (2025).}
\end{table}

Para atender ao escopo definido, adotou-se arquitetura distribuída de \textit{hardware} e \textit{software}. Na camada física, selecionou-se o \textit{Arduino Uno} para leitura dos sensores e o acionamento dos atuadores, enquanto o \textit{Raspberry Pi} foi designado para comunicação entre microcontrolador, servidor e  aplicação \textit{web}. Na camada de \textit{software}, definiu-se o uso do \textit{framework Spring Boot} no \textit{back-end}, em virtude de sua robustez na implementação de \textit{APIs REST}, \textit{React} para desenvolvimento da interface \textit{web} responsiva e \textit{PostgreSQL} para persistência e kgerenciamento dos dados coletados.

\section{Fase de Elaboração}
Com objetivo de mitigar riscos técnicos e validar decisões arquiteturais do sistema, foram desenvolvidos protótipos funcionais parciais, conforme recomendado pelo RUP. Nessa etapa, realizaram-se testes preliminares de integração entre sensores, o microcontrolador \textit{Arduino} e \textit{Raspberry Pi}, com foco na verificação da estabilidade da comunicação serial, latência no envio de dados e viabilidade da arquitetura proposta. Esses experimentos permitiram confirmar que a infraestrutura de \textit{hardware} selecionada atende aos requisitos de desempenho estabelecidos na fase de Concepção.

Simultaneamente, foram definidos artefatos estruturais do sistema, incluindo modelagem dos casos de uso, arquitetura de comunicação entre dispositivos físicos e especificação dos componentes embarcados que compõem o protótipo. Dessa forma, a fase de Elaboração estabelece ligação entre os requisitos identificados na Concepção e a implementação conduzida na fase de Construção, assegurando que o sistema seja desenvolvido com base em arquitetura previamente validada e devidamente documentada.

\subsection*{Modelagem de Caso de Uso}

A Figura \ref{fig:usecase} apresenta o Diagrama de Casos de Uso do sistema, detalhando interações permitidas ao usuário, como visualização de parâmetros e configuração de alertas, bem como integração dos dispositivos físicos com o \textit{back-end}.

\begin{figure}[H]
    \centering
    \caption{Diagrama de Caso de Uso do Sistema}
    \label{fig:usecase}
    \includegraphics[width=1.00\textwidth]{imagens/meuDiagramaTcc.png}
    \caption*{Fonte: Autoria própria (2025).}
\end{figure}

\subsection*{Arquitetura Geral do Sistema}

O sistema foi concebido a partir de arquitetura distribuída, composta por sensores, microcontrolador \textit{Arduino}, um microcomputador \textit{Raspberry Pi}, \textit{back-end} desenvolvido com o \textit{framework Spring Boot} e uma interface \textit{web} implementada em \textit{React}. O fluxo de funcionamento do sistema ocorre conforme as etapas:

\begin{enumerate}
    \item Coleta dos dados pelos sensores conectados ao \textit{Arduino};
    \item Transmissão das leituras do \textit{Arduino} para o \textit{Raspberry Pi};
    \item Envio periódico das informações ao servidor por meio de requisições HTTP;
    \item Armazenamento dos dados utilizando o SGBD;
    \item Apresentação dos parâmetros ao usuário por meio da interface \textit{web}.
\end{enumerate}

Essa arquitetura favorece escalabilidade e desacoplamento entre as camadas do sistema. Entretanto, a distribuição dos componentes físicos introduz riscos operacionais, como dependência da integridade funcional de múltiplos dispositivos. A falha de um nó, seja sensor ou controlador, pode comprometer o fluxo de dados.

Além disso, a comunicação baseada em protocolos HTTP está sujeita a latências e instabilidades de rede, especialmente em infraestruturas domésticas, o que exige implementação de mecanismos de tratamento de exceções, tentativas de reconexão automática e tolerância a falhas. Tais estratégias são fundamentais para garantir que o sistema seja capaz de recuperar operabilidade sem necessidade de intervenção humana após falhas momentâneas de conectividade.

\subsection*{Componentes Utilizados no Sistema}

A seguir, são apresentados os principais componentes físicos selecionados para o desenvolvimento do protótipo, incluindo sensores, atuadores e dispositivos de controle. 

% A definição desses elementos nesta fase está alinhada ao RUP, uma vez que a Elaboração contempla a consolidação da arquitetura física e lógica do sistema. 

\subsubsection*{Sensores de Monitoramento}

Para medição da temperatura da água, optou-se pelo sensor NTC 10K (Modelo MF58), ilustrado na Figura \ref{fig:sensortemp} (Apêndice \ref{apendice:hardware}). O componente foi integrado ao circuito por meio de divisor de tensão e conectado a uma entrada analógica do \textit{Arduino} para leitura dos dados térmicos.

O monitoramento do nível da água é realizado pelo sensor LC26M-40, apresentado na Figura \ref{fig:sensorlevel} (Apêndice \ref{apendice:hardware}). Fabricado em polipropileno, fornece sinal digital (nível lógico alto ou baixo) diretamente ao microcontrolador, permitindo identificação imediata de níveis críticos no reservatório.

A qualidade química da água é aferida por Sensor de pH (Figura \ref{fig:sensorph}, Apêndice \ref{apendice:hardware}) e Sensor de Turbidez Modelo ST100 (Figura \ref{fig:sensorturbidez}, Apêndice \ref{apendice:hardware}). O sensor de pH monitora a acidez e alcalinidade, enquanto o sensor de turbidez identifica partículas em suspensão. Ambos geram sinais analógicos processados pelo \textit{Arduino} e enviados ao servidor para análise e tomada de decisão.

\subsubsection*{Atuadores}

Para controle do fluxo de água, foi utilizado bomba submersa modelo JT100 (3V a 5V), conforme detalhado na Figura \ref{fig:bombasub} (Apêndice \ref{apendice:hardware}). Com vazão entre 1000 e 1500 ml/min e elevação de até 1 metro, este equipamento é responsável pelo controle de nível de água da piscina, sendo acionado quando o sensor de nível indicar leituras abaixo do limite configurado.

\subsubsection*{Controladores e Gateway}

O controle de \textit{hardware} é exercido pela placa \textit{Arduino Uno R3} (Figura \ref{fig:arduino}, Apêndice \ref{apendice:hardware}) sua escolha deve-se à robustez nas leituras analógicas e à vasta disponibilidade de bibliotecas para controle dos atuadores em tempo real.

Como elemento intermediário (\textit{Gateway}), utiliza-se o \textit{Raspberry Pi 3 Model B}, exibido na Figura \ref{fig:raspberry} (Apêndice \ref{apendice:hardware}). Equipado com processador, gerencia a comunicação Serial com o \textit{Arduino} e executa \textit{scripts} responsáveis pelo envio dos dados à API, isolando a lógica de rede da lógica de controle físico. As especificações técnicas dos componentes de \textit{hardware} selecionados estão detalhadas na Tabela \ref{tab:resumo_hardware}.

\begin{table}[H]
    \centering
    \caption{Resumo das Especificações Técnicas dos Componentes de Hardware}
    \label{tab:resumo_hardware}
    \begin{tabularx}{\textwidth}{l >{\raggedright\arraybackslash}X c >{\raggedright\arraybackslash}X}
    \toprule
    \textbf{Componente} & \textbf{Função Principal} & \textbf{Tensão} & \textbf{Comunicação / Sinal} \\
    \midrule
    \textit{Raspberry Pi 3 B} & \textit{Gateway} de comunicação e envio de dados ao servidor & 5\,V & Serial (USB) / \textit{Wi-Fi} (HTTP) \\
    \addlinespace
    \textit{Arduino Uno R3} & Leitura de sensores e controle de atuadores & 5\,V  & \textit{Serial} / I/O digital e analógico \\
    \addlinespace
    Sensor de Temperatura (NTC) & Monitoramento térmico da água & 5\,V & Analógico (divisor de tensão) \\
    \addlinespace
    Sensor de Nível & Detecção de nível do reservatório & 5\,V & Digital (\textit{On/Off}) \\
    \addlinespace
    Sensor de pH & Medição da acidez e alcalinidade da água & 5\,V & Analógico \\
    \addlinespace
    Sensor de Turbidez & Avaliação da transparência da água & 5\,V & Analógico \\
    \addlinespace
    Bomba Submersa & Circulação e filtragem da água & 3--6\,V & Acionamento via relé (sinal digital) \\
    \bottomrule
    \end{tabularx}
    \caption*{Fonte: Autoria própria (2025).}
\end{table}

\section{Fase de Construção}

Em conformidade com a metodologia RUP, a fase de Construção foi conduzida por meio de ciclos iterativos e incrementais, nos quais funcionalidades do sistema foram implementadas e validadas progressivamente. O desenvolvimento iniciou-se pela camada física, com implementação e validação individual dos sensores e atuadores conectados ao microcontrolador. Após estabilização da aquisição dos dados, procedeu-se à implementação da camada lógica, englobando \textit{back-end}, o banco de dados e \textit{front-end}. 

A etapa final concentrou-se na integração completa entre \textit{hardware} e \textit{software}, validando o fluxo de transmissão das informações, desde a coleta no ambiente físico até a apresentação na interface \textit{web}. O desenvolvimento do sistema foi estruturado em duas camadas, camada embarcada, responsável pela leitura, pré-processamento e envio dos dados coletados e camada de aplicação, encarregada de receber, armazenar e disponibilizar essas informações aos usuários.

Na camada embarcada, o \textit{Arduino} foi programado por meio de IDE nativa para realizar leitura dos sensores analógicos e digitais, bem como acionamento condicional dos atuadores. A montagem física e organização dos componentes durante os testes de bancada pode ser visualizada na Figura \ref{fig:testes_bancada}.

\begin{figure}[H]
    \centering
    \caption{Testes em bancada}
    \label{fig:testes_bancada}
    \includegraphics[width=0.6\textwidth]{imagens/prototipoMesa.jpeg}
    \caption*{Fonte: Autoria própria (2025).}
\end{figure}

Conforme Apêndice \ref{apendice:codigos}, a Seção \ref{cod:arduino} ilustra a rotina de coleta dos parâmetros da água e a lógica empregada para controle dos dispositivos físicos.

Com intuito de  garantir confiabilidade das informações coletadas e evitar armazenamento de dados incorretos, foi desenvolvido fluxo lógico de validação e normalização antes da persistência no banco de dados. A Figura \ref{fig:fluxo_logica} demonstra o fluxo de decisão executado pelo sistema a cada ciclo de leitura.

\begin{figure}[H]
    \centering
    \caption{Fluxograma de validação de dados e tomada de decisão do sistema}
    \label{fig:fluxo_logica}
    \includegraphics[width=0.5\textwidth]{imagens/diagramaFluxo.png}
    \caption*{Fonte: Autoria própria (2025).}
\end{figure}

O \textit{Raspberry Pi} desempenhou função de \textit{gateway} de comunicação. Um \textit{script} desenvolvido em \textit{Python} foi responsável por estabelecer comunicação serial com o microcontrolador, interpretar dados recebidos e encaminhá-los ao servidor por meio de requisições HTTP do tipo \textit{POST}. A lógica de implementação desse \textit{gateway}, incluindo mecanismos de tratamento de falhas de comunicação e serialização dos dados em formato \textit{JSON}, encontra-se detalhada no Apêndice \ref{apendice:codigos}, Seção \ref{cod:python}.

Na camada de aplicação, o \textit{back-end}, desenvolvido utilizando \textit{framework} \textit{Spring Boo}t, centraliza a lógica de persistência e fornece dados base para o sistema. Uma funcionalidade fundamental implementada foi a personalização do monitoramento: o sistema gerencia perfil de usuário com dimensões físicas da piscina e limites de alerta. O \textit{back-end} utiliza esses parâmetros para calcular dinamicamente o volume exato a cada requisição. A partir desse dado, a interface \textit{web} executa algoritmo de recomendação de dosagem química, sugerindo correções precisas, como evidenciado na Figura \ref{fig:imageModal}. O \textit{controller} responsável pelo cálculo de volume e persistência está apresentado no Apêndice \ref{apendice:codigos}, Seção \ref{cod:java}, enquanto a lógica de recomendação e alertas é detalhada na Seção \ref{cod:react_logic}. Essa arquitetura assegura integridade, consistência e disponibilidade do histórico de monitoramento.

\begin{figure}[H]
    \centering
    \caption{Modal de recomendação de dosagem}
    \label{fig:imageModal}
    \includegraphics[width=0.8\textwidth]{imagens/printModal.png}
    \caption*{Fonte: Autoria própria (2025).}
\end{figure}

A interface do usuário, desenvolvida em \textit{React}, consome dados fornecidos pela API para apresentar o estado atual da piscina em tempo quase real. A implementação da lógica de consumo da API e o gerenciamento de estado para visualização dos parâmetros e acionamento manual dos atuadores estão descritos no Apêndice \ref{apendice:codigos}, Seção \ref{cod:react}.

Além do monitoramento passivo, a interface \textit{web} foi dotada de lógica reativa para auxílio à tomada de decisão. Ao detectar que parâmetros críticos (como pH ou turbidez) estão fora dos limites estabelecidos, o sistema habilita funcionalidade interativa de cálculo de correção. Por meio dela, o usuário pode visualizar instantaneamente a sugestão de dosagem exata dos produtos químicos necessários (ex: quantidade de clarificante ou redutor de pH), calculada com base no volume específico da piscina.

\section{Fase de Transição}

Na fase de Transição, o sistema desenvolvido foi submetido a testes práticos com objetivo de validar seu desempenho, estabilidade e conformidade com os requisitos estabelecidos nas fases anteriores. Foram realizados testes unitários nos sensores, abrangendo calibração do sensor NTC de temperatura e pH, verificação da estabilidade do sensor de nível e simulações de condições críticas de funcionamento das bombas, de modo a prevenir situações de \textit{dry-run}\footnote{Ensaio geral, simulação ou teste prático de um processo, apresentação ou sistema, realizado antes do evento real para identificar falhas sem consequências.}. Paralelamente, testes de integração avaliaram o fluxo completo dos dados, desde a aquisição no ambiente físico até a apresentação das informações na interface \textit{web}, conforme detalhado na Tabela \ref{tab:testes_realizados}. 

\begin{table}[H]
    \centering
    \caption{Resumo dos Testes de Validação Sistêmica}
    \label{tab:testes_realizados}
    \begin{tabularx}{\textwidth}{ >{\raggedright\arraybackslash}p{4cm} >{\raggedright\arraybackslash}X >{\raggedright\arraybackslash}p{3.5cm} }
    \toprule
    \textbf{Teste Realizado} & \textbf{Objetivo do Teste} & \textbf{Critério/Resultado} \\
    \midrule
    Calibração do Sensor NTC & Verificar precisão térmica por comparação com termômetro de referência & Margem de erro aceitável ($<\pm 2^{\circ}$C) \\
    \addlinespace
    Calibração do pH & Ajuste de ganho via \textit{trimpot}\footnote{Um potenciômetro miniatura ajustável, utilizado para calibrações finas e ocasionais diretamente na placa de circuito impresso.} do módulo comparado com valor de referência & Erro absoluto médio $< 0,3$ pH \\
    \addlinespace
    Calibração de Turbidez & Definição dos limiares de tensão para água limpa e suja (sensor ST100) & Diferenciação clara (Limpa $> 4$V / Suja $< 1$V) \\
    \addlinespace
    Estabilidade do Nível & Confirmar acionamento da boia sem oscilações espúrias & \textit{Debounce}\footnote{ma técnica para garantir que uma função ou evento seja executado apenas uma vez após um determinado intervalo de tempo, ignorando chamadas repetidas e rápidas.} lógico funcional \\
    \addlinespace
    Integração de Dados & Validar fluxo completo Arduino $\rightarrow$ Raspberry Pi $\rightarrow$ Banco de Dados & Persistência verificada no PostgreSQL \\
    \addlinespace
    Latência da Interface & Medir tempo entre leitura do sensor e atualização na interface \textit{web} & Atualização em tempo quase real ($<$ 5s) \\
    \bottomrule
    \end{tabularx}
    \caption*{Fonte: Autoria própria (2025).}
\end{table}

Adicionalmente, foram realizados testes de responsividade da interface \textit{web} e de confiabilidade da comunicação entre servidor e dispositivos embarcados. Os ajustes finais envolveram correção de algoritmos de conversão dos sensores, adequação de temporizações de leitura e transmissão, bem como melhorias na visualização gráfica dos dados apresentados ao usuário.

Embora os testes tenham validado a funcionalidade do sistema, observaram-se limitações decorrentes da escala reduzida do protótipo. A discussão detalhada sobre esses desafios e o desempenho quantitativo do sistema será apresentada a seguir.

% A principal restrição técnica observada refere-se à calibração fina dos sensores analógicos de turbidez e pH, os quais apresentaram sensibilidade a ruídos elétricos e variações ambientais, demandando tratamento adicional dos sinais por meio de filtragem e normalização em software. Tais limitações evidenciam que, embora o sistema atenda aos requisitos funcionais propostos, sua precisão pode ser impactada por fatores externos em cenários de uso prolongado.

% Outro aspecto relevante diz respeito à dependência da infraestrutura de rede local. Falhas momentâneas de conectividade podem ocasionar atrasos ou lacunas temporárias no monitoramento em tempo real, ainda que não comprometam o armazenamento posterior dos dados. Esse comportamento reforça a necessidade de estratégias adicionais de tolerância a falhas, como armazenamento local temporário e mecanismos de retransmissão automática.


