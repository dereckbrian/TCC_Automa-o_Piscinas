\chapter{Desenvolvimento}

% =======================
% SEÇÃO 3.1 — TIPO DE PESQUISA E ETAPAS
% =======================

\section{Tipo de Pesquisa e Etapas de Construção}

A pesquisa caracteriza-se como aplicada, de abordagem mista e natureza exploratória e experimental. Inicialmente, foi conduzido um estudo teórico sobre automação residencial, sensores, atuadores e controladores IoT, fundamentado em autores apresentados no \autoref{cap:fundamentacao}. Essa etapa permitiu compreender os processos físico-químicos envolvidos no tratamento da água, as normas sanitárias aplicáveis e os requisitos técnicos necessários para a integração entre hardware, software e dispositivos embarcados.

% Inserir após este parágrafo uma breve problematização das limitações dos métodos manuais (ex.: falta de precisão, riscos sanitários, dependência da experiência do usuário).
% Objetivo: fortalecer a justificativa metodológica da automação e conectar com o Capítulo 2.        (FEITO)

% A motivação principal desta pesquisa surge das limitações significativas associadas à limpeza manual de piscinas, que envolvem uma série de desafios e riscos. Primeiramente, há o risco à saúde dos usuários, uma vez que a limpeza inadequada pode comprometer a qualidade da água, criando condições propícias para o crescimento de microrganismos prejudiciais à saúde. Além disso, a falta de precisão no momento de adição dos produtos químicos à piscina pode resultar no desperdício desses produtos, prejudicando tanto a eficácia do tratamento quanto gerando custos desnecessários. Por fim, a limpeza de piscinas ainda depende fortemente do conhecimento de quem a realiza, o que cria uma dependência de profissionais qualificados. Isso significa que, em muitas situações, indivíduos sem a experiência necessária precisam contar com a ajuda de outros, o que pode gerar dificuldades operacionais e inconsistências na qualidade da manutenção.       ((REMOVIDO POR SER COISA DA INTRODUÇÃO))

O levantamento bibliográfico incluiu artigos científicos, manuais técnicos e normas da ABNT, garantindo base técnica suficiente para orientar as decisões posteriores de arquitetura, modelagem e implementação do sistema. Os critérios escolhidos para seleção das fontes bibliográficas foram: a atualidade das informações, priorizando-se publicações dos últimos 10 anos para temas relacionados à tecnologia, mercado de automação e Internet das Coisas (\textit{IoT}); a relevância técnica, com o uso de manuais de fabricantes e guias especializados para fundamentar os processos físico-químicos de tratamento da água; e a robustez teórica, buscando-se autores consolidados e trabalhos acadêmicos reconhecidos para embasar as metodologias de Engenharia de Software e arquitetura de sistemas.

Essa etapa inicial permitiu mapear os requisitos técnicos necessários para a integração entre hardware, software e dispositivos embarcados, garantindo base suficiente para orientar as decisões posteriores de arquitetura, modelagem e implementação do sistema.

% Aqui ainda cabe acrescentar uma frase explicando o critério de seleção das fontes (relevância técnica, atualidade, normalização).
% Isso deixa a metodologia mais robusta.       (FEITO)


% =======================
% SEÇÃO 3.2 — RUP
% =======================

\section{Processo de Desenvolvimento de Software}

Neste projeto, adotou-se o Rational Unified Process (RUP) como metodologia de desenvolvimento devido à sua estrutura iterativa e incremental, ao foco na mitigação de riscos e à ênfase na modelagem e documentação. Conforme exposto na fundamentação teórica, o RUP organiza o processo de engenharia de software em quatro fases principais: Concepção, Elaboração, Construção e Transição. Cada fase apresenta objetivos específicos que orientam a evolução do sistema, garantindo rastreabilidade entre requisitos, arquitetura e implementação.

É importante destacar que, por se tratar de um sistema de automação (IoT), o ciclo de vida do desenvolvimento de hardware foi adaptado com as fases do RUP. Sendo assim, a definição dos componentes eletrônicos ocorreu durante a análise de requisitos (Concepção), enquanto a montagem física e a calibração dos sensores foram realizadas em paralelo à codificação (Construção).  %Adicionado para tentar manter a coerencia.

A escolha dessa metodologia consolida um fluxo contínuo durante o desenvolvimento, no qual cada artefato produzido em uma fase é responsável por validar a etapa seguinte. Dessa forma, o levantamento de requisitos realizado na fase de Concepção serve como base para a modelagem arquitetural na fase de Elaboração, com os diagramas orientando o desenvolvimento do sistema na fase de Construção. Por fim, a conformidade do software desenvolvido é verificada durante a fase de Transição.

As subseções seguintes descrevem cada etapa, destacando as atividades realizadas e sua relação com os artefatos produzidos.

% Introduzir aqui uma frase de ligação explicando que cada fase do RUP produzirá artefatos específicos validados na fase seguinte.
% Isso reforça coerência metodológica.       (FEITO)


% =======================
% SEÇÃO 3.3 — INCEPTION 
% =======================

\section{Fase de Concepção (\textit{Inception})}

Nesta fase, foram definidos o problema, o escopo inicial do projeto e os requisitos fundamentais. O problema identificado consiste na dificuldade encontradas por usuários no processo manual de manutenção de piscinas, devido à necessidade de medições constantes, cálculos físicos e químicos, além da possibilidade de desperdício de produtos decorrente da aplicação incorreta das dosagens.

Diante disso, a execução inadequada dos processos de limpeza acarreta riscos diretos à integridade dos usuários e dos equipamentos. Segundo a literatura técnica, a falha no controle do pH é crítica: valores abaixo de 7,2 provocam irritações imediatas na pele e nos olhos dos banhistas, além de acelerar a corrosão de tubulações e componentes metálicos. Já a manutenção de um pH acima de 7,8, comum em correções manuais imprecisas, reduz drasticamente a eficácia dos agentes sanitizantes, comprometendo a desinfecção da água.

% Recomendo adicionar aqui dados estimados (literatura) sobre erros comuns na manutenção manual:
% - subdosagem de cloro
% - sobrecorreção de pH
% - riscos sanitários
% Isso fortalecerá a crítica e a motivação científica.      (FEITO)

A partir dessa análise, concebeu-se o sistema de automação com base nos seguintes objetivos: automatizar a leitura dos principais parâmetros da água; reduzir o uso inadequado de produtos químicos; otimizar o acionamento dos mecanismos de filtragem; e fornecer ao usuário meios de visualização e interação com os dados.


O levantamento de requisitos resultou nos seguintes requisitos funcionais e não funcionais, apresentados nas Tabelas \ref{tab:req_funcionais} e \ref{tab:req_nao_funcionais}, respectivamente.

% Ainda falta criar critérios objetivos de aceitação dos requisitos.
% Sugestão: após as tabelas, adicionar parágrafo explicando métricas:
% - Faixa de erro aceitável dos sensores
% - Tempo máximo entre coleta e exibição
% - Limite de latência nas requisições
% - Condições para considerar um requisito atendido     (FEITO)

\begin{table}[H] 
    \centering 
    \caption{Requisitos Funcionais do Sistema de Automação de Piscinas} 
    \label{tab:req_funcionais} 
    % X faz a coluna Descrição ocupar todo o espaço restante e quebra linha sozinho
    \begin{tabularx}{\textwidth}{l X} 
    \toprule
    \textbf{Código} & \textbf{Descrição} \\ 
    \midrule
    \textbf{RF01} & O sistema deve monitorar automaticamente os níveis de pH, turbidez, temperatura e nível da água. \\ 
    \addlinespace % Espaço extra para leitura
    \textbf{RF02} & O sistema deve acionar automaticamente a bomba de filtragem e o aquecedor conforme os parâmetros definidos ou coletados pelos sensores. \\ 
    \addlinespace
    \textbf{RF03} & O sistema deve disponibilizar uma interface web para visualização dos parâmetros monitorados. \\ 
    \addlinespace
    \textbf{RF04} & O sistema deve permitir o cadastro e armazenamento dos dados coletados no banco de dados. \\ 
    \addlinespace
    \textbf{RF05} & O sistema deve permitir o cadastro de piscinas vinculadas a usuários. \\ 
    \addlinespace
    \textbf{RF06} & O sistema deve emitir alertas quando algum parâmetro ultrapassar o limite ideal. \\ 
    \addlinespace
    \textbf{RF07} & O sistema deve emitir gráficos de acordo com os últimos dados de determinados parâmetros coletados. \\ % Corrigi RF06 duplicado para RF07
    \bottomrule
    \end{tabularx}
    \caption*{Fonte: Autoria própria (2025).}
\end{table}

Para a validação dos requisitos elencados, foram estabelecidos critérios quantitativos de aceitação com base no desempenho esperado para um protótipo IoT. Determinou-se que a precisão dos sensores (RF01) deve operar com margem de erro inferior a $\pm$5\% em comparação a medições de referência. Quanto ao desempenho temporal, estabeleceu-se que o tempo de latência total, compreendido entre a aquisição do dado físico pelo microcontrolador e sua visualização na interface web, não deve exceder 5 segundos, garantindo a monitoração em tempo quase real. No que tange à comunicação (RNF03), considera-se o sistema estável se as requisições HTTP apresentarem taxa de sucesso superior a 95\% em operação contínua, com tempo de resposta do servidor inferior a 4 segundos por requisição. Por fim, o requisito de automação (RF02) será considerado atendido se o acionamento dos atuadores (bombas) ocorrer em até 2 segundos após a detecção de uma condição crítica parametrizada no sistema.

\vspace{0.5cm} 

\begin{table}[H] 
    \centering 
    \caption{Requisitos Não Funcionais do Sistema de Automação de Piscinas} 
    \label{tab:req_nao_funcionais} 
    \begin{tabularx}{\textwidth}{l X} 
    \toprule
    \textbf{Código} & \textbf{Descrição} \\ 
    \midrule
    \textbf{RNF01} & O sistema deve utilizar o banco de dados PostgreSQL para armazenamento das informações. \\ 
    \addlinespace
    \textbf{RNF02} & A interface web deve ser responsiva e acessível em dispositivos móveis e desktops. \\ 
    \addlinespace
    \textbf{RNF03} & A comunicação entre o microcontrolador e o servidor deve ocorrer de forma segura, utilizando protocolos HTTP/HTTPS. \\ 
    \addlinespace
    \textbf{RNF04} & O sistema deve ser desenvolvido com o framework Spring Boot no back-end e React no front-end. \\ 
    \bottomrule
    \end{tabularx}
    \caption*{Fonte: Autoria própria (2025).}
\end{table}

As tecnologias selecionadas para atender ao escopo inicial incluíram o Raspberry Pi como dispositivo de controle, sensores de pH, nível e temperatura, o framework React para interface web e PostgreSQL como banco de dados. O uso do Spring Boot no \textit{back-end} foi definido devido à sua robustez e integração com bibliotecas de segurança, conforme discutido na \autoref{cap:fundamentacaoSoftware}.

% =======================
% SEÇÃO 3.4 — ELABORATION
% =======================

\section{Fase de Elaboração (\textit{Elaboration})}

% Antes do texto que já está, inserir um parágrafo explicando que esta fase valida a arquitetura por meio de protótipos parciais (RUP exige).
% Algo como: “Protótipos de comunicação sensor–Arduino–Raspberry foram testados para verificar viabilidade.”    (FEITO)

Visando a mitigação de riscos técnicos, foram desenvolvidos protótipos funcionais com o intuito de validar a estabilidade da arquitetura proposta. Testes preliminares de integração entre os sensores, o microcontrolador Arduino e o Raspberry Pi foram executados para verificar a capacidade da comunicação serial e a latência no envio de dados, garantindo que a infraestrutura de hardware escolhida suportaria os requisitos de desempenho do sistema.

Nesta etapa foram definidos os artefatos estruturais do sistema, incluindo a modelagem dos casos de uso, a arquitetura de comunicação entre os dispositivos físicos e a especificação dos componentes embarcados que compõem o protótipo. Assim, esta fase estabelece o vínculo entre os requisitos levantados na Concepção e a implementação realizada na fase de Construção, garantindo que o sistema seja desenvolvido com base em uma arquitetura validada e documentada.

% A modelagem UML poderia incluir também um diagrama de classes ou sequência, caso haja tempo.

\subsection*{Modelagem de Caso de Uso}

O Diagrama de Caso de Uso, elaborado segundo a UML, fornece uma visão geral das interações entre os atores e as funcionalidades principais do sistema. Esse diagrama permite compreender como o usuário acessa as informações coletadas, como interage com o sistema e como os dispositivos físicos se integram às operações lógicas implementadas.

\begin{figure}[H]
    \centering
    \caption{Diagrama de Caso de Uso do Sistema}
    \label{fig:usecase}
    \includegraphics[width=1.0\textwidth]{imagens/meuDiagramaTcc.png}
    \caption*{Fonte: Autor.}
\end{figure}

\subsection*{Arquitetura Geral do Sistema}

O sistema foi concebido em uma arquitetura distribuída composta por sensores, um microcontrolador (Arduino), um microcomputador (Raspberry Pi), o \textit{back-end} desenvolvido em Spring Boot e a interface web criada com React. O fluxo principal consiste em:

1. Coleta dos dados pelos sensores conectados ao Arduino;
2. Comunicação entre Arduino e Raspberry Pi para envio das leituras;
3. Envio periódico das informações ao servidor por meio de requisições HTTP;
4. Armazenamento dos dados no banco PostgreSQL;
5. Apresentação dos parâmetros ao usuário via interface web.

Essa arquitetura permite escalabilidade e desacoplamento entre as camadas, em conformidade com os princípios discutidos no \autoref{cap:fundamentacao}. Contudo, a distribuição dos componentes introduz riscos operacionais, como a dependência crítica da integridade física de múltiplos dispositivos, onde a falha em um nó (sensor ou controlador) pode interromper o fluxo de dados. Adicionalmente, a comunicação via protocolos HTTP está sujeita a latências e instabilidades de rede, como as observadas na infraestrutura local, o que impõe a necessidade de implementar mecanismos de tratamento de exceções e reconexão automática. Tais medidas são essenciais para garantir que o sistema recupere sua operabilidade sem intervenção humana após falhas momentâneas de conectividade.

% Adicionar aqui uma análise crítica dos riscos identificados na arquitetura:
% - dependência de múltiplos dispositivos físicos
% - possíveis falhas de comunicação HTTP
% - necessidade de reconexão automática
% Isso demonstra maturidade metodológica.   (FEITO)

\subsection*{Componentes Utilizados no Sistema}

A seguir apresentam-se os componentes físicos selecionados para o desenvolvimento do protótipo, incluindo sensores, atuadores e microcontroladores. A apresentação destes elementos nesta fase é coerente com o RUP, uma vez que a Elaboração contempla a definição da arquitetura física e lógica do sistema.

% -------------------------------------------------
% SENSOR DE TEMPERATURA
% -------------------------------------------------

\subsubsection*{Sensor de Temperatura}

\begin{figure}[H]
    \centering
    \caption{Sensor de Temperatura MF58 (NTC 10K)}
    \label{fig:sensortemp}
    \includegraphics[width=0.80\textwidth]{imagens/sensorTemperatura.png}
    \caption*{Fonte: \cite{siteComprei}.}
\end{figure}

O sensor selecionado para a medição da temperatura da água é o modelo MF58 (NTC 10K), um termistor cujo coeficiente de resistência decresce conforme a temperatura aumenta. A leitura é realizada por uma porta analógica do Arduino por meio de um divisor de tensão, permitindo a conversão da resistência em valores de temperatura.

% -------------------------------------------------
% SENSOR DE NÍVEL
% -------------------------------------------------

\subsubsection*{Sensor de Nível}

\begin{figure}[H]
    \centering
    \caption{Sensor de Nível LC26M-40}
    \label{fig:sensorlevel}
    \includegraphics[width=0.60\textwidth]{imagens/sensorNive.png}
    \caption*{Fonte: \cite{siteComprei2}.}
\end{figure}

O sensor de nível utilizado é o modelo LC26M-40, fabricado em Polipropileno (PP). Seu funcionamento baseia-se em um interruptor magnético interno (\textit{Reed Switch}), que altera seu estado conforme o flutuador se move com a variação do nível da água. O componente fornece um sinal digital tipo SPST, permitindo a identificação de nível crítico no reservatório.

% -------------------------------------------------
% SENSOR DE PH
% -------------------------------------------------

\subsubsection*{Sensor de pH}

\begin{figure}[H]
    \centering
    \caption{Sensor de Nível LC26M-40}
    \label{fig:sensorlevel}
    \includegraphics[width=0.60\textwidth]{imagens/sensorpH.png}
    \caption*{Fonte: \cite{siteCompreiPh}.}
\end{figure}

O sensor de pH empregado no sistema realiza a medição da acidez da água da piscina. Os valores capturados são processados inicialmente no Arduino e enviados ao servidor para análise dos parâmetros e posterior recomendação de ajuste químico. A imagem será adicionada posteriormente, conforme previsto pelo aluno.

% -------------------------------------------------
% SENSOR DE Turbidez
% -------------------------------------------------

\subsubsection*{Sensor de Turbidez}

\begin{figure}[H]
    \centering
    \caption{Sensor de Nível LC26M-40}
    \label{fig:sensorlevel}
    \includegraphics[width=0.60\textwidth]{imagens/sensorTurbidez.png}
    \caption*{Fonte: \cite{siteCompreiTurbidez}.}
\end{figure}

O sensor de turbidez selecionado para o desenvolvimento do sistema foi o ST100, responsável por medir a presença de partículas em suspensão na água (turbidez). Os valores analógicos capturados pelo módulo de leitura são inicialmente processados pelo Arduino e, em seguida, enviados ao servidor para análise dos parâmetros. Com base nesses dados, o sistema faz a recomendação para o acionamento do sistema de filtragem ou aplicação do decantador, conforme necessário.

% -------------------------------------------------
% BOMBAS SUBMERSAS
% -------------------------------------------------

\subsubsection*{Bombas Submersas}

\begin{figure}[H]
    \centering
    \caption{Bomba Submersa JT100}
    \label{fig:bombasub}
    \includegraphics[width=0.60\textwidth]{imagens/bombaSubmersa.png}
    \caption*{Fonte: \cite{siteComprei}.}
\end{figure}

Foram empregadas bombas submersas de 3 a 5V, modelo JT100, responsáveis pelo enchimento, filtragem e acionamento da cascata. O equipamento apresenta vazão de 1000 a 1500 ml/min e altura máxima de elevação de 1 metro. Suas dimensões reduzidas e facilidade de integração justificam sua escolha para este protótipo.

% -------------------------------------------------
% ARDUINO
% -------------------------------------------------

\subsubsection*{Arduino}

\begin{figure}[H]
    \centering
    \caption{Placa Arduino Uno R3}
    \label{fig:arduino}
    \includegraphics[width=0.60\textwidth]{imagens/arduino.png}
    \caption*{Fonte: \cite{siteComprei4}.}
\end{figure}

O Arduino Uno R3 foi selecionado como microcontrolador responsável pela leitura direta dos sensores e pelo acionamento dos atuadores. Sua ampla compatibilidade com bibliotecas abertas e a quantidade adequada de pinos digitais e analógicos possibilitam atender aos requisitos levantados na Concepção.

% -------------------------------------------------
% RASPBERRY
% -------------------------------------------------

\subsubsection*{\textit{Raspberry Pi}}

\begin{figure}[H]
    \centering
    \caption{\textit{Raspberry Pi 3 Model B}}
    \label{fig:raspberry}
    \includegraphics[width=1.00\textwidth]{imagens/raspBerryQeuUso.png}
    \caption*{Fonte: \cite{siteComprei3}.}
\end{figure}

O \textit{Raspberry Pi 3 Model B} foi definido como o dispositivo intermediário entre o Arduino e o servidor. Seu processador quad-core, 1GB de RAM e compatibilidade com Linux permitem a execução de rotinas contínuas de envio de dados, comunicação co

% Dereck incluir na Elaboration uma tabela resumo com:
% - componente
% - função
% - tensão
% - protocolo de comunicação
% Isso melhora a organização e facilita leitura.    (FEITO)

\begin{table}[H]
    \centering
    \caption{Resumo das Especificações Técnicas dos Componentes de Hardware}
    \label{tab:resumo_hardware}
    
    % Definição das colunas:
    % l = alinhado à esquerda (ajusta ao tamanho do texto)
    % X = ocupa o espaço restante e quebra linha (bom para textos longos)
    % c = centralizado
    \begin{tabularx}{\textwidth}{l >{\raggedright\arraybackslash}X c >{\raggedright\arraybackslash}X}
    \toprule
    \textbf{Componente} & \textbf{Função Principal} & \textbf{Tensão} & \textbf{Comunicação / Sinal} \\ 
    \midrule
    
    Raspberry Pi 3 B & Gateway de comunicação e envio de dados ao servidor & 5V & Serial (USB) / Wi-Fi (HTTP) \\ 
    \addlinespace
    
    Arduino Uno R3 & Microcontrolador para leitura de sensores e controle de atuadores & 5V / 9V & Serial / I/O Digital e Analógico \\ 
    \addlinespace
    
    Sensor Temp. (NTC) & Monitoramento térmico da água & 5V & Analógico (Divisor de Tensão) \\ 
    \addlinespace
    
    Sensor de Nível & Detecção de nível crítico do reservatório & 5V & Digital (On/Off) \\ 
    \addlinespace
    
    Sensor de pH & Medição da acidez e alcalinidade & 5V & Analógico \\ 
    \addlinespace
    
    Sensor de Turbidez & Análise da transparência da água & 5V & Analógico \\ 
    \addlinespace
    
    Bomba Submersa & Atuador para circulação e filtragem & 3V - 6V & Acionamento via Relé (Sinal Digital) \\ 
    
    \bottomrule
    \end{tabularx}
    \caption*{Fonte: Autoria própria (2025).}
\end{table}


% =======================
% SEÇÃO 3.5 — CONSTRUCTION
% =======================

\section{Fase de Construção (\textit{Construction})}

% Antes da descrição, inserir uma frase explicando que esta fase implementou incrementos sucessivos conforme o RUP recomenda.
% Ex.: “A implementação ocorreu em ciclos, priorizando primeiro sensores e comunicação básica.”     (FEITO)

Em conformidade com a metodologia RUP, a fase de Construção foi executada por meio de ciclos iterativos e incrementais. O desenvolvimento iniciou-se pela camada física, com a implementação e validação individual dos sensores e atuadores no microcontrolador. Após garantir a estabilidade da leitura dos dados, iniciou-se a codificação da camada lógica, abrangendo o back-end (Spring Boot), o banco de dados e o front-end (React). A etapa final consistiu na integração completa entre o hardware e o software, validando o fluxo de transmissão das informações desde a coleta no ambiente físico até a visualização na interface web.

O desenvolvimento do código foi conduzido em duas camadas: a camada embarcada, responsável pela leitura e envio dos dados; e a camada de aplicação, que recebeu, armazenou e exibiu as informações aos usuários.

Na camada embarcada, o Arduino foi programado via IDE própria para realizar a leitura dos sensores analógicos e digitais. Conforme o Apêndice \ref{apendice:codigos}, Código \ref{fig:cod_arduino} demonstra a função de coleta dos parâmetros da água e o acionamento condicional dos atuadores.

Em seguida, o Raspberry Pi desempenha a função de gateway de comunicação. Um script desenvolvido em Python é responsável por estabelecer a conexão serial com o microcontrolador, realizar a leitura dos dados brutos e encaminhá-los via requisições HTTP POST para a API na nuvem. A lógica de implementação desse gateway, incluindo o tratamento de erros de conexão e serialização JSON, está detalhada no Apêndice \ref{apendice:codigos}, Código \ref{lst:python_gateway}.

Na camada de aplicação, o back-end, construído com o framework Spring Boot, expõe os endpoints necessários para a recepção e persistência dos dados. O Controller responsável por mapear as requisições, converter os objetos JSON e salvar as leituras no banco de dados PostgreSQL pode ser consultado no Apêndice \ref{apendice:codigos}, Código \ref{lst:backend_controller}. Essa arquitetura centraliza a regra de negócios, garantindo a integridade e a disponibilidade do histórico de monitoramento.

Por fim, a interface do usuário (front-end), desenvolvida em React, consome os dados processados pela API para exibir o estado atual da piscina em tempo real. A implementação da lógica de consumo da API (\textit{fetching}) e o controle de estado para acionamento manual dos atuadores estão apresentados no Apêndice \ref{apendice:codigos}, Código \ref{lst:frontend_fetch}, respectivamente.

% É necessário incluir trechos de código comentados (curtos), conforme exigência de TCC de software.
% Indicar onde: logo após o primeiro parágrafo da Construction.     (FEITO)

O Arduino, programado via IDE própria, capturou os parâmetros da água e acionou os atuadores quando necessário. O Raspberry Pi, utilizando scripts em Python, recebeu os dados do microcontrolador e os encaminhou ao servidor. O \textit{back-end}, desenvolvido em Spring Boot, estruturou a API responsável pela comunicação, enquanto o \textit{front-end} em React consolidou a interface de visualização.

% =======================
% SEÇÃO 3.6 — TRANSITION
% =======================

\section{Fase de Transição (\textit{Transition})}

Nesta fase, o sistema passou por testes práticos para validação do seu desempenho. Foram aplicados testes unitários nos sensores, incluindo calibração do NTC, verificação da estabilidade do sensor de nível e simulações de funcionamento crítico das bombas para evitar condições de \textit{dry-run}. Testes de integração verificaram o fluxo completo de dados, desde a leitura até a exibição na interface.

% Adicionar tabela com:
% - Teste realizado     (FEITO)

\begin{table}[H]
    \centering
    \caption{Resumo dos Testes Realizados e Resultados Obtidos}
    \label{tab:testes_realizados}
    
    % Definição das colunas:
    % p{4cm}: Largura fixa para o nome do teste
    % X: Ocupa o resto do espaço e quebra linha (para o objetivo longo)
    % p{3cm}: Espaço reservado para o resultado
    \begin{tabularx}{\textwidth}{ >{\raggedright\arraybackslash}p{4cm} >{\raggedright\arraybackslash}X >{\raggedright\arraybackslash}p{3cm} }
    \toprule
    \textbf{Teste Realizado} & \textbf{Objetivo do Teste} & \textbf{Resultado} \\ 
    \midrule
    
    Calibração do Sensor NTC & Verificar a precisão térmica comparada a um termômetro de referência & \\ 
    \addlinespace
    
    Calibração do Sensor de pH & Verificar a precisão do sensor comparando com testes de estojos para analisar parâmetros da água & \\ 
    \addlinespace
    
    Validação de Turbidez & Verificar linearidade e sensibilidade através do método de diluição seriada (água límpida vs. saturada) & \\ 
    \addlinespace
    
    Estabilidade do Sensor de Nível & Confirmar acionamento correto da boia sem oscilações espúrias (\textit{debounce}) & \\ 
    \addlinespace
    
    Proteção \textit{Dry-run} & Impedir o acionamento da bomba quando o reservatório está vazio & \\ 
    \addlinespace
    
    Integração de Dados & Validar o fluxo: Arduino $\to$ Raspberry $\to$ Banco de Dados & \\ 
    \addlinespace
    
    Latência da Interface & Medir tempo entre a leitura do sensor e atualização na tela & \\ 
    
    \bottomrule
    \end{tabularx}
    \caption*{Fonte: Autoria própria (2025).}
\end{table}
Também foram realizados testes de responsividade da interface e de comunicação entre o servidor e o dispositivo embarcado. Os ajustes finais envolveram correções de algoritmos de conversão, ajustes de temporização e adequações na exibição gráfica dos dados.

Apesar dos resultados satisfatórios, foram identificadas limitações inerentes à prototipagem em escala reduzida. A principal dificuldade técnica residiu na calibração fina dos sensores analógicos (turbidez e pH), que demonstraram sensibilidade a ruídos elétricos, exigindo tratamento de dados via software para garantir a confiabilidade descrita na \autoref{tab:testes_realizados}. Além disso, a dependência da infraestrutura de rede local evidenciou que falhas momentâneas de conectividade podem gerar lacunas no monitoramento em tempo real.

Como trabalhos futuros, sugere-se a validação do sistema em uma piscina de alvenaria em escala real, permitindo avaliar o comportamento dos sensores sob intempéries e uso contínuo. Recomenda-se também a incorporação de um sensor de Cloro Livre (ORP) para fechar o ciclo completo de automação química, bem como a implementação de algoritmos de inteligência artificial para manutenção preditiva, visando antecipar falhas nos equipamentos com base no histórico de operação.

% Inserir ao final desta seção uma reflexão crítica sobre limitações identificadas nos testes e possíveis trabalhos futuros derivados.      (FEITO)