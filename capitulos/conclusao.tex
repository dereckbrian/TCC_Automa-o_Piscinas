\chapter{Considerações Finais} \label{cap:conclusao}

Este trabalho teve como objetivo desenvolver um sistema automatizado para o monitoramento e manutenção de piscinas residenciais, integrando tecnologias de Internet das Coisas (IoT) a uma arquitetura de software distribuída. Para isso, foram realizadas pesquisas sobre parâmetros físico-químicos da água e protocolos de automação residencial, adaptando componentes de baixo custo para atuar no controle de bombas e na leitura de sensores. Também foi desenvolvida uma aplicação completa utilizando Spring Boot e React, permitindo o gerenciamento remoto do sistema. Com isso, foi possível validar a viabilidade técnica da solução proposta, demonstrando que o protótipo atingiu seus objetivos ao reduzir a necessidade de intervenção manual e promover maior segurança sanitária aos usuários.

A automação do tratamento de água, utilizando hardware acessível e software livre, representa um avanço significativo para a domótica residencial, permitindo maior precisão na dosagem de produtos e no acionamento de filtros. O estudo confirmou que a arquitetura de Gateway, onde o Raspberry Pi gerencia a comunicação e o Arduino executa o controle físico, oferece robustez e escalabilidade. Dessa forma, a hipótese de que a integração de sensores e atuadores otimiza o uso de recursos hídricos e energéticos foi validada, proporcionando uma gestão mais sustentável e econômica.

A contribuição deste trabalho é relevante para a área de tecnologia ao demonstrar a aplicação prática da metodologia RUP em projetos de IoT. A principal inovação prática reside na implementação de um ciclo de controle fechado, onde os parâmetros influenciam diretamente no funcionamento do sistema sem dependência humana. Metodologicamente, a pesquisa integra conceitos de eletrônica embarcada, redes de computadores e desenvolvimento web, oferecendo um modelo replicável para futuros projetos acadêmicos e comerciais.

Apesar dos avanços alcançados, algumas limitações ainda permanecem. A dependência da infraestrutura de rede Wi-Fi local mostrou-se um ponto crítico para o monitoramento remoto, embora as rotinas de segurança local do microcontrolador tenham mitigado os riscos de falhas operacionais.

Os resultados obtidos reforçam que a automação residencial é um campo acessível e em expansão. O trabalho demonstra que é possível desenvolver soluções tecnológicas complexas com recursos limitados, contribuindo não apenas para o conforto doméstico, mas também para a conscientização sobre o uso racional da água e a modernização dos processos de manutenção residencial.