\chapter{Considerações Finais} \label{cap:conclusao}

Este trabalho teve como objetivo desenvolver um sistema automatizado para o monitoramento e manutenção de piscinas residenciais, integrando tecnologias de Internet das Coisas (IoT) a uma arquitetura de software distribuída. Para isso, foram realizadas pesquisas sobre parâmetros físico-químicos da água e protocolos de automação residencial, adaptando componentes de baixo custo (Arduino e Raspberry Pi) para atuar no controle de bombas e na leitura de sensores. Também foi desenvolvida uma aplicação completa (Full Stack) utilizando Spring Boot e React, permitindo o gerenciamento remoto do sistema. Com isso, foi possível validar a viabilidade técnica da solução proposta, demonstrando que o protótipo atingiu seus objetivos ao reduzir a necessidade de intervenção manual e promover maior segurança sanitária aos usuários.

A automação do tratamento de água, utilizando hardware acessível e software livre, representa um avanço significativo para a domótica residencial, permitindo maior precisão na dosagem de produtos e no acionamento de filtros. O estudo confirmou que a arquitetura de Gateway, onde o Raspberry Pi gerencia a comunicação e o Arduino executa o controle físico, oferece robustez e escalabilidade. Dessa forma, a hipótese de que a integração de sensores e atuadores otimiza o uso de recursos hídricos e energéticos foi validada, proporcionando uma gestão mais sustentável e econômica.

A contribuição deste trabalho é relevante para a área de Ciência da Computação e Engenharia de Software ao demonstrar a aplicação prática da metodologia RUP em projetos de IoT. A principal inovação prática reside na implementação de um ciclo de controle fechado (closed-loop), onde a turbidez e o pH influenciam diretamente o acionamento dos equipamentos sem dependência humana. Metodologicamente, a pesquisa integra conceitos de eletrônica embarcada, redes de computadores e desenvolvimento web, oferecendo um modelo replicável para futuros projetos acadêmicos e comerciais.

Apesar dos avanços alcançados, algumas limitações ainda permanecem. Os sensores analógicos de baixo custo (pH e Turbidez) apresentaram sensibilidade a ruídos elétricos, exigindo tratamentos de sinal via software para garantir leituras estáveis. Além disso, a dependência da infraestrutura de rede Wi-Fi local mostrou-se um ponto crítico para o monitoramento remoto, embora as rotinas de segurança local do microcontrolador tenham mitigado os riscos de falhas operacionais.

Como perspectivas de trabalhos futuros, recomenda-se a incorporação de um sensor de ORP (Potencial de Oxirredução) para permitir a dosagem automatizada de cloro, fechando o ciclo químico completo. Sugere-se também a implementação de algoritmos de aprendizado de máquina (Machine Learning) para prever falhas na bomba baseando-se no histórico de consumo e vibração. Por fim, a validação do sistema em uma piscina de alvenaria em escala real seria fundamental para avaliar a durabilidade dos componentes submersos em uso contínuo.

Os resultados obtidos reforçam que a automação residencial é um campo acessível e em expansão. O trabalho demonstra que é possível desenvolver soluções tecnológicas complexas com recursos limitados, contribuindo não apenas para o conforto doméstico, mas também para a conscientização sobre o uso racional da água e a modernização dos processos de manutenção residencial.