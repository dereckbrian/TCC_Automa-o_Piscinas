\chapter{Considerações Finais}
\label{cap:conclusao}

Este trabalho teve como objetivo desenvolver um sistema automatizado para o monitoramento e a manutenção de piscinas residenciais, integrando tecnologias de Internet das Coisas (IoT) a uma arquitetura de software distribuída. Para alcançar esse propósito, foram investigados os principais parâmetros físico-químicos da água, bem como conceitos e protocolos associados à automação residencial, possibilitando a adaptação de componentes de baixo custo para a leitura de sensores e o controle de atuadores. Adicionalmente, foi desenvolvida uma aplicação completa, composta por \textit{back-end} em Spring Boot e \textit{front-end} em React, viabilizando o gerenciamento remoto e a visualização dos dados em tempo real.

Os resultados obtidos demonstraram a viabilidade técnica da solução proposta, evidenciando que o protótipo desenvolvido é capaz de reduzir significativamente a necessidade de intervenção manual no processo de manutenção da piscina, ao mesmo tempo em que promove maior segurança sanitária aos usuários. A automação do monitoramento e do acionamento dos dispositivos permitiu maior precisão no controle dos parâmetros avaliados, minimizando falhas associadas à operação manual e contribuindo para a estabilidade do tratamento da água.

A adoção de hardware acessível e software livre configura um avanço relevante no contexto da domótica residencial, ao demonstrar que soluções automatizadas podem ser implementadas de forma economicamente viável. O estudo confirmou que a arquitetura baseada em \textit{gateway}, na qual o Raspberry Pi atua como intermediário da comunicação e o Arduino executa o controle físico dos dispositivos, apresenta-se como uma alternativa robusta e escalável. Dessa forma, a hipótese central do trabalho (de que a integração entre sensores, atuadores e tecnologias IoT pode otimizar o uso de recursos hídricos e energéticos) foi validada, resultando em uma gestão mais eficiente, sustentável e econômica.

Do ponto de vista científico e tecnológico, a principal contribuição deste trabalho reside na aplicação prática da metodologia Rational Unified Process (RUP) em um projeto de IoT, evidenciando sua adequação para o desenvolvimento de sistemas que envolvem integração entre hardware e software. Destaca-se, ainda, a implementação de um ciclo de controle fechado, no qual os dados coletados pelos sensores influenciam diretamente o comportamento do sistema, sem dependência constante da intervenção humana. Metodologicamente, a pesquisa integra conceitos de eletrônica embarcada, redes de computadores e desenvolvimento web, oferecendo um modelo que pode ser replicado e adaptado em futuros projetos acadêmicos ou aplicações comerciais.

Apesar dos avanços alcançados, algumas limitações foram identificadas ao longo do desenvolvimento. A dependência da infraestrutura de rede Wi-Fi local mostrou-se um ponto crítico para o monitoramento remoto contínuo, podendo impactar a atualização da interface em cenários de instabilidade de conexão. Ainda assim, as rotinas de controle local implementadas no microcontrolador contribuíram para mitigar riscos operacionais, assegurando o funcionamento básico do sistema mesmo diante de falhas de comunicação.

De forma geral, os resultados obtidos reforçam que a automação residencial constitui um campo acessível e em constante expansão. Este trabalho demonstra que é possível desenvolver soluções tecnológicas relativamente complexas com recursos limitados, promovendo não apenas maior conforto e praticidade no ambiente doméstico, mas também a conscientização sobre o uso racional da água, a eficiência energética e a modernização dos processos de manutenção residencial.
