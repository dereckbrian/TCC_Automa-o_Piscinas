\chapter{Considerações Finais}
\label{cap:conclusao}

Este trabalho teve como objetivo desenvolver sistema automatizado para monitoramento e manutenção de piscinas residenciais, integrando tecnologias de Internet das Coisas (IoT) a uma arquitetura de \textit{software} distribuída. Para alcançar esse propósito, foram investigados os principais parâmetros físico-químicos da água, bem como conceitos e protocolos associados à automação residencial, possibilitando adaptação de componentes de baixo custo para leitura de sensores e controle de atuadores. Adicionalmente, foi desenvolvida aplicação completa, composta por \textit{back-end} em \textit{Spring Boot} e \textit{front-end} em \textit{React}, viabilizando o gerenciamento remoto e visualização dos dados em tempo quase real.

Os resultados obtidos demonstraram viabilidade técnica da solução proposta, evidenciando que o protótipo desenvolvido é capaz de reduzir a necessidade de intervenção manual no processo de manutenção da piscina, ao mesmo tempo em que promove segurança sanitária aos usuários. A automação do monitoramento e acionamento dos dispositivos permitiu aprimoramento da precisão no controle dos parâmetros avaliados, minimizando falhas associadas à operação manual e contribuindo para a estabilidade do tratamento da água.

A adoção de \textit{hardware} acessível e \textit{software} livre configura avanço relevante no contexto da domótica residencial, ao demonstrar que soluções automatizadas podem ser implementadas de forma economicamente viável. O estudo confirmou que a arquitetura baseada em \textit{gateway}, na qual o \textit{Raspberry Pi} atua como intermediário da comunicação e o \textit{Arduino} executa controle físico dos dispositivos, apresenta-se como alternativa robusta e escalável. Dessa forma, a hipótese do trabalho foi validada, resultando em gestão eficiente, sustentável e econômica.

Do ponto de vista científico e tecnológico, a principal contribuição deste trabalho reside na aplicação da metodologia Rational Unified Process (RUP) em projeto de IoT, evidenciando sua adequação para o desenvolvimento de sistemas que envolvem integração entre \textit{hardware} e \textit{software}. Destaca-se, a implementação de ciclo de controle fechado, no qual dados coletados pelos sensores influenciam o comportamento do sistema, sem dependência constante da intervenção humana. A pesquisa integra conceitos de eletrônica embarcada, redes de computadores e desenvolvimento \textit{web}, oferecendo modelo que pode ser replicado e adaptado em futuros projetos acadêmicos ou aplicações comerciais.

Apesar dos avanços alcançados, algumas limitações foram identificadas ao longo do desenvolvimento. A dependência da infraestrutura de rede \textit{Wi-Fi} local mostrou-se ponto crítico para o monitoramento remoto contínuo, podendo impactar a atualização da interface em cenários de instabilidade de conexão. Ainda assim, rotinas de controle local implementadas no microcontrolador contribuíram para mitigar riscos operacionais, assegurando funcionamento do sistema mesmo diante de falhas de comunicação.

Os resultados obtidos reforçam que a automação residencial constitui campo acessível e em constante expansão. Este trabalho demonstra que é possível desenvolver soluções tecnológicas com recursos limitados, promovendo não apenas conforto e praticidade no ambiente doméstico, mas também conscientização sobre o uso racional da água, eficiência energética e modernização dos processos de manutenção residencial.
