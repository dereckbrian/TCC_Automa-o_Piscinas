\chapter{Trabalhos Correlatos}

A revisão bibliográfica permitiu identificar diferentes abordagens tecnológicas aplicadas à automação de piscinas e ambientes residenciais correlatos, variando desde soluções baseadas em acionamentos temporizados até sistemas fundamentados em controladores microprocessados e controladores lógicos programáveis. Esses trabalhos apresentam contribuições relevantes, porém adotam estratégias distintas quanto ao nível de automação, aos parâmetros monitorados e aos mecanismos de interação com o usuário. A seguir, são analisadas as principais propostas identificadas, com ênfase em seus escopos, limitações e diferenças em relação à solução desenvolvida neste trabalho.

Uma das abordagens iniciais encontradas é a de \citeonline{campos2014automaccao}, que propôs um sistema amplo de automação residencial utilizando a plataforma Arduino. Nesse projeto, a piscina é tratada como um subsistema, cujo controle de filtragem é realizado exclusivamente por meio de agendamento temporal (\textit{timer}), sem a leitura de parâmetros físico-químicos da água. Embora o trabalho apresente uma integração eficiente entre diferentes ambientes residenciais, observa-se a ausência de sensores voltados à análise da qualidade da água, o que limita a capacidade do sistema de atuar de forma inteligente e adaptativa no tratamento químico da piscina.

Com foco específico no controle químico da água, \citeonline{brandao2018sistema} desenvolveram um sistema de malha fechada utilizando o Arduino Uno para o controle de pH e desinfecção. O diferencial dessa proposta reside na utilização de um sensor de potencial de oxirredução (ORP) para estimar a concentração de cloro livre, associado ao sensor de pH, com atuação realizada por bombas dosadoras peristálticas. Apesar da robustez do controle químico, a interação com o usuário restringe-se a um display LCD local, inexistindo conectividade com a Internet ou recursos de monitoramento remoto, o que reduz a flexibilidade e a escalabilidade da solução.

Na mesma linha de automação voltada ao equilíbrio químico, \citeonline{boeira2019autopool} apresentaram o protótipo denominado \textit{Autopool}, desenvolvido com base no Arduino Mega. O sistema integra sensores de pH, nível e chuva, além de bombas adaptadas para a dosagem de produtos químicos. Um aspecto relevante desse trabalho é a inclusão de um mecanismo de segurança por meio de sensor de movimento, com o objetivo de prevenir acidentes. Entretanto, o sistema opera de forma isolada (\textit{offline}), sendo configurado exclusivamente por meio de interface física local, e não contempla o monitoramento da turbidez da água para a automação do processo de filtragem física.

Em uma proposta mais recente, \citeonline{oliveira2023automaccao} desenvolveram um sistema de automação de área de lazer utilizando o microcontrolador ESP32, destacando-se pela conectividade Wi-Fi integrada e pelo dimensionamento da infraestrutura elétrica. O sistema possibilita o acionamento remoto de bombas e iluminação, caracterizando-se como uma solução de controle digital à distância. Todavia, a ausência de sensores para monitoramento da qualidade da água faz com que as decisões sobre o momento adequado de filtragem dependam diretamente da intervenção do usuário, não configurando, portanto, um ciclo de automação autônomo.

Em contraste com os trabalhos analisados, a solução proposta neste TCC diferencia-se por implementar um ciclo de automação fechado (\textit{closed-loop}) integrado aos princípios da Internet das Coisas (IoT). Enquanto propostas como as de \citeonline{campos2014automaccao} e \citeonline{oliveira2023automaccao} concentram-se em agendamento temporal ou acionamento remoto, e os trabalhos de \citeonline{brandao2018sistema} e \citeonline{boeira2019autopool} priorizam o controle químico local, o sistema aqui desenvolvido combina sensores de pH e turbidez para subsidiar decisões autônomas tanto sobre o tratamento químico quanto sobre a filtragem física da água. Ademais, a disponibilização de dados históricos e a gestão por meio de uma interface web ampliam a capacidade de monitoramento e análise do sistema ao longo do tempo.

A Tabela \ref{tab:trabalhos_correlatos} apresenta uma síntese comparativa das funcionalidades contempladas nos trabalhos correlatos e na proposta desenvolvida neste estudo, evidenciando os diferenciais da solução apresentada.

\begin{table}[H]
    \centering
    \caption{Comparativo de funcionalidades entre trabalhos correlatos e o sistema proposto}
    \label{tab:trabalhos_correlatos}
    \small
    \setlength{\tabcolsep}{-6.1pt}
    \begin{tabularx}{\textwidth}{ l *{6}{>{\centering\arraybackslash}X} }
    \toprule
    \textbf{Referência} & 
    \textbf{Automação \newline Química} & 
    \textbf{Leitura \newline de \newline Turbidez} & 
    \textbf{Interface \newline Web/IoT} & 
    \textbf{Controle \newline Autônomo} & 
    \textbf{Baixo \newline Custo} & 
    \textbf{Segurança \newline (Alarme)} \\ 
    \midrule
    \citeonline{campos2014automaccao}       &  &  & \checkmark &  & \checkmark & \checkmark \\ 
    \addlinespace
    \citeonline{brandao2018sistema}         & \checkmark &  &  & \checkmark & \checkmark &  \\ 
    \addlinespace
    \citeonline{boeira2019autopool}         & \checkmark &  &  & \checkmark & \checkmark & \checkmark \\ 
    \addlinespace
    \citeonline{sulimann2014automatizacao}  & \checkmark &  &  & \checkmark &  &  \\ 
    \addlinespace
    \citeonline{oliveira2023automaccao}     &  &  & \checkmark &  & \checkmark &  \\ 
    \midrule
    \textbf{Este Trabalho}                  & \checkmark & \checkmark & \checkmark & \checkmark & \checkmark &  \\ 
    \bottomrule
    \end{tabularx}
    \par \vspace{0.2cm}
    {\small \textbf{Fonte:} Autoria própria (2025).}
\end{table}

A análise comparativa evidencia a existência de uma lacuna no desenvolvimento de sistemas de baixo custo capazes de realizar, de forma simultânea, o monitoramento analítico de pH e turbidez, com atuação autônoma e conectividade via Internet. Essa lacuna, identificada de maneira recorrente nos trabalhos analisados, é diretamente abordada pela proposta apresentada neste estudo, a qual busca integrar monitoramento, tomada de decisão automática e acesso remoto em uma solução unificada.

Diante desse cenário, a automação aplicada à manutenção de piscinas apresenta-se como um campo em expansão, ainda carente de abordagens que conciliem baixo custo, autonomia operacional e integração com tecnologias de Internet das Coisas. Assim, o capítulo seguinte descreve o desenvolvimento da solução proposta, detalhando a metodologia adotada, a modelagem do sistema, as decisões arquiteturais e as etapas de implementação que materializam a proposta delineada a partir das lacunas identificadas nos trabalhos correlatos.
