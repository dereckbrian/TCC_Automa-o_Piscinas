\chapter{Trabalhos Correlatos}

A revisão bibliográfica permitiu identificar abordagens tecnológicas aplicadas à automação de piscinas e ambientes residenciais, variando de soluções baseadas em acionamentos temporizados a sistemas fundamentados em controladores microprocessados e lógico programáveis. Esses trabalhos apresentam contribuições relevantes e adotam estratégias distintas quanto ao nível de automação, parâmetros monitorados e mecanismos de interação com o usuário. A seguir, são analisadas as propostas identificadas, com ênfase em seus escopos, limitações e diferenças em relação à solução desenvolvida neste trabalho.

Uma das abordagens encontradas é a de \citeonline{campos2014automaccao}, que propôs sistema de automação residencial utilizando a plataforma \textit{Arduino}. Nesse projeto, a piscina é tratada como subsistema, cujo controle de filtragem é realizado exclusivamente de agendamento temporal (\textit{timer}), sem leitura de parâmetros físico-químicos da água. Embora o trabalho apresente integração eficiente entre diferentes ambientes residenciais, observa-se a ausência de sensores voltados à análise da qualidade da água, o que limita a capacidade do sistema de atuar de forma inteligente e adaptativa no tratamento químico da piscina.

Com foco no controle químico da água, \citeonline{brandao2018sistema} desenvolveram sistema de malha fechada utilizando \textit{Arduino Uno} para controle de pH e desinfecção. O diferencial dessa proposta reside na utilização de um sensor de potencial de oxirredução (ORP) para estimar a concentração de cloro livre, associado ao sensor de pH, com atuação realizada por bombas dosadoras peristálticas. Apesar da robustez do controle químico, a interação com o usuário restringe-se a um \textit{display} LCD, inexistindo conectividade com a Internet ou recursos de monitoramento remoto, o que reduz a flexibilidade e escalabilidade da solução.

Na mesma linha de automação voltada ao equilíbrio químico, \citeonline{boeira2019autopool} apresenta protótipo denominado \textit{Autopool}, desenvolvido com base no \textit{Arduino Mega}. O sistema integra sensores de pH, nível e chuva, bombas adaptadas para dosagem de produtos químicos. Um aspecto relevante desse trabalho é a inclusão de mecanismo de segurança por meio de sensor de movimento, com objetivo de prevenir acidentes. Entretanto, o sistema opera de forma isolada, sendo configurado exclusivamente por meio de interface física local, e não contempla monitoramento da turbidez da água.

Em proposta mais recente, \citeonline{oliveira2023automaccao} desenvolveram sistema de automação de área de lazer utilizando o microcontrolador ESP32, destacando-se pela conectividade \textit{Wi-Fi} integrada e dimensionamento da infraestrutura elétrica. O sistema possibilita acionamento remoto de bombas e iluminação, caracterizando-se como solução de controle digital à distância. Todavia, a ausência de sensores para monitoramento da qualidade da água faz com que decisões sobre o momento adequado de filtragem dependam diretamente da intervenção do usuário, não configurando, ciclo de automação autônomo.

Em contraste com os trabalhos analisados, a solução proposta neste trabalho diferencia-se por implementar ciclo de automação fechado (\textit{closed-loop}) integrado aos princípios da Internet das Coisas (IoT). Enquanto propostas como as de \citeonline{campos2014automaccao} e \citeonline{oliveira2023automaccao} concentram-se em agendamento temporal ou acionamento remoto, e os trabalhos de \citeonline{brandao2018sistema} e \citeonline{boeira2019autopool} priorizam controle químico local, o sistema desenvolvido combina sensores de pH e turbidez para subsidiar decisões autônomas sobre o tratamento químico da água. Ademais, a disponibilização de dados históricos e gestão por meio de interface \textit{web} ampliam a capacidade de monitoramento e análise do sistema.

A Tabela \ref{tab:trabalhos_correlatos} apresenta síntese comparativa das funcionalidades contempladas nos trabalhos correlatos e na proposta desenvolvida, evidenciando os diferenciais da solução apresentada.

\begin{table}[H]
    \centering
    \caption{Comparativo de funcionalidades entre trabalhos correlatos e o sistema proposto}
    \label{tab:trabalhos_correlatos}
    \small
    \setlength{\tabcolsep}{-6.1pt}
    \begin{tabularx}{\textwidth}{ l *{6}{>{\centering\arraybackslash}X} }
    \toprule
    \textbf{Referência} & 
    \textbf{Automação \newline Química} & 
    \textbf{Leitura \newline de \newline Turbidez} & 
    \textbf{Interface \newline Web/IoT} & 
    \textbf{Controle \newline Autônomo} & 
    \textbf{Baixo \newline Custo} & 
    \textbf{Segurança \newline (Alarme)} \\ 
    \midrule
    \citeonline{campos2014automaccao}       &  &  & \checkmark &  & \checkmark & \checkmark \\ 
    \addlinespace
    \citeonline{brandao2018sistema}         & \checkmark &  &  & \checkmark & \checkmark &  \\ 
    \addlinespace
    \citeonline{boeira2019autopool}         & \checkmark &  &  & \checkmark & \checkmark & \checkmark \\ 
    \addlinespace
    \citeonline{oliveira2023automaccao}     &  &  & \checkmark &  & \checkmark &  \\ 
    \midrule
    \textbf{Este Trabalho}                  & \checkmark & \checkmark & \checkmark & \checkmark & \checkmark &  \\ 
    \bottomrule
    \end{tabularx}
    \par \vspace{0.2cm}
    {\small \textbf{Fonte:} Autoria própria (2025).}
\end{table}

A análise comparativa evidencia lacuna no desenvolvimento de sistemas de baixo custo capazes de realizar, de forma simultânea, monitoramento analítico de pH e turbidez, com atuação autônoma e conectividade via Internet. Essa lacuna é diretamente abordada pela proposta apresentada, que busca integrar monitoramento, tomada de decisão automática e acesso remoto em solução unificada.

Diante desse cenário, a automação aplicada à manutenção de piscinas apresenta-se como campo em expansão, carente de abordagens que conciliem baixo custo, autonomia operacional e integração com tecnologias de Internet das Coisas. Assim, o capítulo seguinte descreve o desenvolvimento da solução proposta, detalhando a metodologia adotada, a modelagem do sistema, decisões arquiteturais e etapas de implementação que materializam a proposta delineada a partir das lacunas identificadas nos trabalhos correlatos.
