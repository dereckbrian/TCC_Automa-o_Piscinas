% ----------------------------------------------------------
% Introdução (exemplo de capítulo sem numeração, mas presente no Sumário)
% ----------------------------------------------------------
\chapter{Introdução}

O avanço tecnológico consolidou a automação como elemento essencial na otimização de processos produtivos. Desde a Primeira Revolução Industrial, no século XVIII, esse conceito evoluiu até o paradigma contemporâneo da Indústria 4.0, caracterizado pela integração entre sistemas ciberfísicos, sensoriamento inteligente e soluções digitais \cite{automacao}. A relevância desse movimento é evidenciada por dados da Confederação Nacional da Indústria, segundo os quais 72\% das empresas que implementaram tecnologias digitais registraram aumento de produtividade, enquanto 60\% reduziram custos operacionais \cite{CNI}. A expansão desse cenário repercutiu no ambiente doméstico, onde dispositivos conectados, assistentes virtuais e sistemas inteligentes vêm ampliando o conceito de automação residencial. O setor apresenta crescimento expressivo: no Brasil, a automação residencial avançou 21,8\% entre 2023 e 2024 \cite{cresceuBR}, e projeções globais estimam crescimento médio anual de 27,9\% até 2032 \cite{aumentoAosAnos}. Dentro dessa perspectiva de expansão tecnológica, torna-se pertinente examinar áreas específicas do ambiente doméstico que permanecem fortemente dependentes de processos manuais, como a limpeza e a manutenção de piscinas.

Embora a automação residencial se consolide em diversos domínios, a manutenção de piscinas residenciais ainda é majoritariamente manual, exigindo do usuário conhecimento sobre parâmetros físico-químicos da água, dosagens de produtos, cálculo de volume e identificação visual de anomalias \cite{poolmagazine2020automation}. Essas atividades, quando executadas sem precisão, podem gerar desperdício de água e energia, uso inadequado de substâncias químicas e custos elevados de manutenção. Além disso, falhas na dosagem ou análise dos parâmetros podem comprometer a saúde dos usuários e desempenho dos equipamentos \cite{lavor2019verificaccao}. Nesse contexto, formula-se o seguinte problema de pesquisa: Como projetar e implementar sistemas automatizados capazes de realizar a limpeza e a manutenção de piscinas residenciais, reduzindo intervenção manual e promovendo eficiência operacional, segurança sanitária e sustentabilidade no uso de recursos?

A partir desse questionamento, estabelece-se a hipótese: a integração entre sensores, atuadores e controladores microprocessados, associados a tecnologias de automação e Internet das Coisas (IoT), possibilita monitorar parâmetros essenciais da água, como pH, temperatura, turbidez, nível e acionar automaticamente equipamentos de tratamento, garantindo precisão na manutenção e minimizando erros humanos. Presume-se, que a automação possa otimizar recursos, aprimorar a qualidade da água e assegurar processos contínuos e confiáveis. Se confirmada, essa hipótese reforça a necessidade de investigar sua aplicabilidade técnica, econômica e operacional.

A justificativa para o desenvolvimento deste estudo fundamenta-se em três dimensões complementares. A dimensão social envolve a necessidade de preservar a saúde do usuário, considerando que procedimentos inadequados de tratamento da água podem gerar riscos sanitários. A dimensão ambiental relaciona-se ao uso racional de água e energia, uma vez que a automação tende a reduzir ciclos excessivos de filtragem e dosagens inadequadas de produtos químicos. Em relação à dimensão tecnológica, foi realizada análise com base em pesquisas bibliográficas, que revelou lacuna no desenvolvimento de sistemas que integrem de maneira completa monitoramento de parâmetros de qualidade da água (como pH e turbidez) com automação dos atuadores. Embora existam soluções para aquecimento ou controle químico, observa-se carência de estudos recentes que proponham arquitetura unificada de IoT, capaz de implementar ciclo de controle fechado e gestão remota de dados para piscinas residenciais.


\section{Objetivo Geral}

Desenvolver sistema automatizado para limpeza e manutenção de piscinas residenciais, com capacidade de reduzir a intervenção manual, otimizar uso de produtos químicos e promover economia de água e energia, integrando sensores, atuadores e dispositivos IoT.
  
\section{Objetivos Específicos}

\begin{itemize}
\item Minimizar erros operacionais durante o processo de limpeza por meio de automação e controle preciso de dosagem química.
\item Proporcionar praticidade e segurança nas etapas de limpeza e monitoramento da piscina.
\item Implementar monitoramento automatizado dos níveis de pH, turbidez, temperatura e nível da água, centralizando informações em interface web.
\item Reduzir a dependência de intervenções manuais por meio de alertas do sistema e acionamentos automáticos da bomba de água com base nos parâmetros de qualidade.
\end{itemize}

    \vspace{10em}

Observa-se que a automação aplicada à manutenção de piscinas representa campo em expansão, pouco explorado academicamente. Para sustentar a proposta apresentada, o próximo capítulo reúne fundamentos teóricos sobre piscinas e métodos tradicionais de tratamento, princípios da automação residencial, tecnologias empregadas em sistemas automatizados e os componentes necessários para integrar sensores e controladores inteligentes. Essa fundamentação constitui embasamento conceitual indispensável para compreender as etapas metodológicas e o desenvolvimento do sistema proposto.