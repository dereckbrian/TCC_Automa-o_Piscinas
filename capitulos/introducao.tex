% ----------------------------------------------------------
% Introdução (exemplo de capítulo sem numeração, mas presente no Sumário)
% ----------------------------------------------------------
\chapter{Introdução}

O avanço tecnológico consolidou a automação como elemento essencial na otimização de processos produtivos. Desde a Primeira Revolução Industrial, no século XVIII, esse conceito evoluiu até o paradigma contemporâneo da Indústria 4.0, caracterizado pela integração entre sistemas ciberfísicos, sensoriamento inteligente e soluções digitais \citeonline{automacao}. A relevância desse movimento é evidenciada por dados da Confederação Nacional da Indústria, segundo os quais 72\% das empresas que implementaram tecnologias digitais registraram aumento de produtividade, enquanto 60\% reduziram custos operacionais \citeonline{CNI}. A expansão desse cenário repercutiu no ambiente doméstico, onde dispositivos conectados, assistentes virtuais e sistemas inteligentes vêm ampliando o conceito de automação residencial. O setor apresenta crescimento expressivo: no Brasil, a automação residencial avançou 21,8\% entre 2023 e 2024 \citeonline{cresceuBR}, e projeções globais estimam crescimento médio anual de 27,9\% até 2032 \citeonline{aumentoAosAnos}. Dentro dessa perspectiva de expansão tecnológica, torna-se pertinente examinar áreas específicas do ambiente doméstico que permanecem fortemente dependentes de processos manuais, como a limpeza e a manutenção de piscinas.

Embora a automação residencial se consolide em diversos domínios, a manutenção de piscinas residenciais ainda é majoritariamente manual, exigindo do usuário conhecimento sobre parâmetros físico-químicos da água, dosagens de produtos, cálculo de volume e identificação visual de anomalias \citeonline{poolmagazine2020automation}. Essas atividades, quando executadas sem precisão, podem gerar desperdício de água e energia, uso inadequado de substâncias químicas e custos elevados de manutenção. Além disso, falhas na dosagem ou na análise dos parâmetros podem comprometer a saúde dos usuários e o desempenho dos equipamentos \citeonline{lavor2019verificaccao}. Nesse contexto, formula-se o seguinte problema de pesquisa: como desenvolver um sistema automatizado capaz de realizar a limpeza e a manutenção de piscinas residenciais, reduzindo a intervenção manual e promovendo eficiência, segurança e sustentabilidade no processo?

A partir desse questionamento, estabelece-se a seguinte hipótese central: a integração entre sensores, atuadores e controladores microprocessados, associados a tecnologias de automação e Internet das Coisas (IoT), possibilita monitorar parâmetros essenciais da água, como pH, temperatura, turbidez, nível e acionar automaticamente equipamentos de tratamento, garantindo maior precisão na manutenção e minimizando erros humanos. Presume-se, portanto, que um sistema automatizado possa otimizar recursos, aprimorar a qualidade da água e assegurar um processo contínuo e confiável. Se confirmada, essa hipótese reforça a necessidade de investigar sua aplicabilidade técnica, econômica e operacional.

A justificativa para o desenvolvimento deste estudo fundamenta-se em três dimensões complementares. A dimensão social envolve a necessidade de preservar a saúde do usuário, considerando que procedimentos inadequados de tratamento da água podem gerar riscos sanitários. A dimensão ambiental relaciona-se ao uso racional de água e energia, uma vez que a automação tende a reduzir ciclos excessivos de filtragem e dosagens inadequadas de produtos químicos. Em relação à dimensão tecnológica, foi realizada uma análise com base nas pesquisas bibliográficas existentes, que revelou uma lacuna no desenvolvimento de sistemas de baixo custo que integrem de maneira completa o monitoramento de parâmetros de qualidade da água (como pH e turbidez) com a automação dos atuadores. Embora existam soluções isoladas para aquecimento ou controle químico, observa-se uma carência de estudos recentes que proponham uma arquitetura unificada de IoT, capaz de implementar o ciclo de controle fechado e a gestão remota de dados para piscinas residenciais.


\section{Objetivo Geral}

Desenvolver um sistema automatizado para limpeza e manutenção de piscinas residenciais, com capacidade de reduzir a intervenção manual, otimizar o uso de produtos químicos e promover economia de água e energia, integrando sensores, atuadores e dispositivos IoT ao processo.
  
\section{Objetivos Específicos}

\begin{itemize}
\item Minimizar erros operacionais durante o processo de limpeza por meio de automação e controle preciso de dosagem química.
\item Proporcionar maior praticidade e segurança nas etapas de limpeza e monitoramento da piscina.
\item Implementar a automação dos processos de filtragem, aquecimento, medição de pH, verificação de temperatura e análise de turbidez da água.
\item Reduzir a dependência de serviços terceirizados de manutenção por meio da automação residencial inteligente.
\item Projetar e testar um módulo de monitoramento de pH com sensores IoT integrados ao sistema de controle.
\item Desenvolver e implementar um algoritmo de acionamento automático da bomba de filtragem com base nas variáveis de qualidade da água.
\end{itemize}

\section{TRABALHOS CORRELATOS}

Durante a pesquisa bibliográfica, foram identificados diferentes tipos de trabalhos, com diferentes propostas tecnológicas, variando desde acionamentos via temporizadores até sistemas com base em controladores lógicos industriais. A seguir, serão apresentadas essas diferentes abordagens para fins comparativos.

Uma das abordagens iniciais encontradas foi a de \citeonline{campos2014automaccao}, que desenvolveu um sistema de automação residencial amplo utilizando a plataforma Arduino. No projeto, a piscina é tratada como um subsistema, onde o controle da filtragem é realizado apenas por agendamento temporal (\textit{timer}), sem a leitura de parâmetros da qualidade da água. O sistema foca na integração de diversos ambientes, mas carece de uma atuação específica e inteligente sobre o tratamento químico da piscina com a utilização de sensores.

Com o foco voltado para o controle de parâmetros químicos, \citeonline{brandao2018sistema} propuseram o desenvolvimento de um sistema de malha fechada utilizando Arduino Uno para o controle de pH e desinfecção. O diferencial deste trabalho foi a utilização de um sensor de ORP (\textit{Oxidation Reduction Potential}) para estimar a concentração de cloro livre, além do sensor de pH. O sistema atua através de bombas dosadoras peristálticas. Entretanto, a interface com o usuário é restrita a um display LCD local, sem recursos de conectividade ou monitoramento remoto via Internet.

Na mesma linha de automação com o foco em química, \citeonline{boeira2019autopool} desenvolveu o protótipo "Autopool" utilizando um Arduino Mega. O projeto integra sensores de pH, nível e chuva, acionando bombas de para-brisa adaptadas para a dosagem de produtos (cloro, elevador e redutor de pH). Um ponto de destaque neste trabalho é a inclusão de um sistema de segurança com sensor de movimento para prevenir afogamentos. Contudo, assim como o trabalho anterior, o sistema opera de forma isolada (\textit{offline}), dependendo de uma interface física (teclado e display) para configuração, e não realiza o monitoramento da turbidez da água para automação da filtragem física.

Um projeto mais recentemente, desenvolvido pelo \citeonline{oliveira2023automaccao} que propõe a automação de uma área de lazer utilizando o microcontrolador ESP32\footnote{é um microcontrolador de baixo custo e baixo consumo, criado para projetos de Internet das Coisas (IoT) e automação, destacando-se por ter Wi-Fi e Bluetooth integrados,}. Este trabalho, tem como foco o acionamento remoto de bombas e iluminação via Wi-Fi e no dimensionamento robusto da infraestrutura elétrica. Contudo, o sistema opera majoritariamente como um controle remoto digital, dependendo do usário para decidir o momento de filtragem da água, uma vez que não possui sensores para monitorar a qualidade da água.

Em contraste com os trabalhos citados, a solução desenvolvida nesta pesquisa se diferencia por implementar um ciclo de automação fechado (\textit{closed-loop}) integrado à Internet das Coisas (IoT). Enquanto \citeonline{campos2014automaccao} e \citeonline{oliveira2023automaccao} focam no agendamento ou acionamento remoto, e \citeonline{brandao2018sistema} e \citeonline{boeira2019autopool} focam no equilíbrio químico local, o presente sistema utiliza sensores de pH e turbidez para tomar decisões autônomas tanto sobre o tratamento químico quanto físico (filtragem), disponibilizando dados históricos e gestão via interface Web.

A Tabela \ref{tab:trabalhos_correlatos} sintetiza a comparação entre os trabalhos analisados e a proposta deste TCC.

\begin{table}[H]
    \centering
    \caption{Comparativo de funcionalidades entre trabalhos correlatos e o sistema proposto}
    \label{tab:trabalhos_correlatos}
    
    % \small diminui um pouco a fonte para tudo caber confortavelmente
    \small 
    
    % --- CORREÇÃO DO "L" ENCIMA DO "AUTOMAÇÃO" ---
    % Reduzimos o espaço entre colunas de 6pt (padrão) para 2pt.
    % Isso dá espaço para o texto "respirar" dentro da célula sem invadir a vizinha.
    \setlength{\tabcolsep}{-6.1pt} 

    % Definição das colunas:
    % l = alinhado à esquerda (Referência)
    % X = colunas centrais com largura igual que quebram linha automaticamente
    \begin{tabularx}{\textwidth}{ l *{6}{>{\centering\arraybackslash}X} }
    \toprule
    
    \textbf{Referência} & 
    \textbf{Automação \newline Química} & 
    \textbf{Leitura \newline de \newline Turbidez} & 
    \textbf{Interface \newline Web/IoT} & 
    \textbf{Controle \newline Autônomo} & 
    \textbf{Baixo \newline Custo} & 
    \textbf{Segurança \newline (Alarme)} \\ 
    \midrule
    
    \citeonline{campos2014automaccao}       &  &  & \checkmark &  & \checkmark & \checkmark \\ 
    \addlinespace % Espacinho para facilitar a leitura
    
    \citeonline{brandao2018sistema}         & \checkmark &  &  & \checkmark & \checkmark &  \\ 
    \addlinespace
    
    \citeonline{boeira2019autopool}         & \checkmark &  &  & \checkmark & \checkmark & \checkmark \\ 
    \addlinespace
    
    \citeonline{sulimann2014automatizacao}  & \checkmark &  &  & \checkmark &  &  \\ 
    \addlinespace
    
    \citeonline{oliveira2023automaccao}     &  &  & \checkmark &  & \checkmark &  \\ 
    \midrule
    
    \textbf{Este Trabalho}                  & \checkmark & \checkmark & \checkmark & \checkmark & \checkmark &  \\ 
    \bottomrule
    \end{tabularx}
    
    \par \vspace{0.2cm}
    {\small \textbf{Fonte:} Autoria própria (2025).}
\end{table}
Dessa forma, observa-se que existe uma lacuna no que tange a sistemas de baixo custo capazes de realizar o monitoramento analítico da turbidez e pH simultaneamente, atuando de forma autônoma e conectada, lacuna esta que o presente projeto visa preencher.

    \vspace{2em}

Com base nessa contextualização, observa-se que a automação aplicada à manutenção de piscinas representa um campo em expansão, mas ainda pouco explorado academicamente. Para sustentar a proposta apresentada, o próximo capítulo reúne os fundamentos teóricos sobre piscinas e seus métodos tradicionais de tratamento, os princípios da automação residencial, as tecnologias empregadas em sistemas automatizados e os componentes necessários para integrar sensores e controladores inteligentes. Essa fundamentação constitui o embasamento conceitual indispensável para compreender as etapas metodológicas e o desenvolvimento do sistema proposto.