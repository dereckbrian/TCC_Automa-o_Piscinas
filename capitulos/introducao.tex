% ----------------------------------------------------------
% Introdução (exemplo de capítulo sem numeração, mas presente no Sumário)
% ----------------------------------------------------------
\chapter{Introdução}

\textcolor{black}{O avanço tecnológico consolidou a automação como elemento essencial na otimização de processos produtivos. Originada na Primeira Revolução Industrial, no século XVIII, essa prática evoluiu até o atual paradigma da Indústria 4.0, marcado pela integração entre sistemas ciberfísicos e soluções digitais \cite{automacao}. De acordo com a Confederação Nacional da Indústria \cite{CNI}, cerca de 72\% das empresas que implementaram tecnologias digitais registraram aumento de produtividade, enquanto 60\% relataram redução de custos operacionais, evidenciando o impacto positivo da automação na eficiência e economia de recursos.}

A automação extrapolou o ambiente industrial e passou a integrar o cotidiano doméstico, por meio de dispositivos inteligentes, assistentes virtuais, sistemas de iluminação e controle de temperatura, além de equipamentos autônomos para limpeza e monitoramento. Essa ampliação resultou no conceito de automação residencial, que visa promover conforto, segurança e eficiência energética. O setor apresenta crescimento acelerado: no Brasil, o mercado de automação residencial expandiu-se 21,8\% entre 2023 e 2024 \cite{cresceuBR}, e, em escala global, estima-se um aumento médio anual de 27,9\% até 2032 \cite{aumentoAosAnos}.

%Ver as fontes disso

\textcolor{red}{colocar pelo menos mais uns três parágrafos para ir transitando}

Apesar do avanço no campo da automação doméstica, a limpeza e manutenção de piscinas residenciais ainda dependem predominantemente de processos manuais. Essa limitação acarreta desperdício de água e energia, uso inadequado de produtos químicos e aumento dos custos de manutenção. Além disso, o manejo incorreto dessas substâncias pode gerar riscos à saúde humana e ao meio ambiente.

Nesse contexto, identifica-se o seguinte problema de pesquisa: como desenvolver um sistema automatizado que realize a limpeza e a manutenção de piscinas residenciais, reduzindo a necessidade de intervenção manual e promovendo eficiência, segurança e sustentabilidade no processo?

Assim, este trabalho propõe o desenvolvimento de um sistema automatizado de limpeza e manutenção de piscinas residenciais, fundamentado em princípios de automação e tecnologias da Internet das Coisas (IoT). A proposta busca demonstrar que a integração entre sensores, atuadores e controladores inteligentes pode tornar o processo de manutenção mais eficiente, sustentável e economicamente viável.



\section{Objetivos}

%Nesta seção são descritos os objetivos, geral e específicos, elencados para este trabalho.

\subsection{Objetivo Geral}

Desenvolver um sistema automatizado para limpeza e manutenção de piscinas residenciais, com capacidade de reduzir a intervenção manual, otimizar o uso de produtos químicos e promover economia de água e energia. 
  
\subsection{Objetivos Específicos}

\begin{itemize}
\item Minimizar erros operacionais durante o processo de limpeza por meio da automação e do controle preciso de dosagem química.
\item Proporcionar maior praticidade e segurança nas etapas de limpeza e monitoramento da piscina.
\item Desenvolver uma solução de baixo custo que viabilize o acesso à automação para diferentes perfis socioeconômicos.
\item Implementar a automação dos processos de filtragem, aquecimento, medição de pH, verificação de temperatura e análise de turbidez da água.
\item Reduzir a dependência de serviços terceirizados de manutenção por meio da automação residencial inteligente.
\item Projetar e testar um módulo de monitoramento de pH com sensores IoT integrados ao sistema de controle.
\item Desenvolver e implementar um algoritmo de acionamento automático da bomba de filtragem com base nas variáveis de qualidade da água.
\end{itemize}

Com base nessa contextualização, observa-se que a automação aplicada à manutenção de piscinas representa uma área em expansão, mas ainda pouco explorada no campo acadêmico. A seguir, apresenta-se a fundamentação teórica, que aborda os princípios técnicos relacionados à manutenção de piscinas, aos processos de automação residencial e às tecnologias envolvidas na integração de sensores e controladores inteligentes. Essa base conceitual é essencial para compreender as etapas metodológicas e o desenvolvimento do sistema proposto.
