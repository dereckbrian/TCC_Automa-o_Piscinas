% ----------------------------------------------------------
% Introdução (exemplo de capítulo sem numeração, mas presente no Sumário)
% ----------------------------------------------------------
\chapter{Introdução}

O avanço tecnológico consolidou a automação como elemento essencial na otimização de processos produtivos. Desde a Primeira Revolução Industrial, no século XVIII, esse conceito evoluiu até o paradigma contemporâneo da Indústria 4.0, caracterizado pela integração entre sistemas ciberfísicos, sensoriamento inteligente e soluções digitais \citeonline{automacao}. A relevância desse movimento é evidenciada por dados da Confederação Nacional da Indústria, segundo os quais 72\% das empresas que implementaram tecnologias digitais registraram aumento de produtividade, enquanto 60\% reduziram custos operacionais \citeonline{CNI}. A expansão desse cenário repercutiu no ambiente doméstico, onde dispositivos conectados, assistentes virtuais e sistemas inteligentes vêm ampliando o conceito de automação residencial. O setor apresenta crescimento expressivo: no Brasil, a automação residencial avançou 21,8\% entre 2023 e 2024 \citeonline{cresceuBR}, e projeções globais estimam crescimento médio anual de 27,9\% até 2032 \citeonline{aumentoAosAnos}. Dentro dessa perspectiva de expansão tecnológica, torna-se pertinente examinar áreas específicas do ambiente doméstico que permanecem fortemente dependentes de processos manuais, como a limpeza e a manutenção de piscinas.

Embora a automação residencial se consolide em diversos domínios, a manutenção de piscinas residenciais ainda é majoritariamente manual, exigindo do usuário conhecimento sobre parâmetros físico-químicos da água, dosagens de produtos, cálculo de volume e identificação visual de anomalias \citeonline{poolmagazine2020automation}. Essas atividades, quando executadas sem precisão, podem gerar desperdício de água e energia, uso inadequado de substâncias químicas e custos elevados de manutenção. Além disso, falhas na dosagem ou na análise dos parâmetros podem comprometer a saúde dos usuários e o desempenho dos equipamentos \citeonline{lavor2019verificaccao}. Nesse contexto, formula-se o seguinte problema de pesquisa: Como projetar e implementar um sistema automatizado capaz de realizar a limpeza e a manutenção de piscinas residenciais, reduzindo a intervenção manual e promovendo eficiência operacional, segurança sanitária e sustentabilidade no uso de recursos?

A partir desse questionamento, estabelece-se a seguinte hipótese central: a integração entre sensores, atuadores e controladores microprocessados, associados a tecnologias de automação e Internet das Coisas (IoT), possibilita monitorar parâmetros essenciais da água, como pH, temperatura, turbidez, nível e acionar automaticamente equipamentos de tratamento, garantindo maior precisão na manutenção e minimizando erros humanos. Presume-se, portanto, que um sistema automatizado possa otimizar recursos, aprimorar a qualidade da água e assegurar um processo contínuo e confiável. Se confirmada, essa hipótese reforça a necessidade de investigar sua aplicabilidade técnica, econômica e operacional.

A justificativa para o desenvolvimento deste estudo fundamenta-se em três dimensões complementares. A dimensão social envolve a necessidade de preservar a saúde do usuário, considerando que procedimentos inadequados de tratamento da água podem gerar riscos sanitários. A dimensão ambiental relaciona-se ao uso racional de água e energia, uma vez que a automação tende a reduzir ciclos excessivos de filtragem e dosagens inadequadas de produtos químicos. Em relação à dimensão tecnológica, foi realizada uma análise com base nas pesquisas bibliográficas existentes, que revelou uma lacuna no desenvolvimento de sistemas de baixo custo que integrem de maneira completa o monitoramento de parâmetros de qualidade da água (como pH e turbidez) com a automação dos atuadores. Embora existam soluções isoladas para aquecimento ou controle químico, observa-se uma carência de estudos recentes que proponham uma arquitetura unificada de IoT, capaz de implementar o ciclo de controle fechado e a gestão remota de dados para piscinas residenciais.


\section{Objetivo Geral}

Desenvolver um sistema automatizado para limpeza e manutenção de piscinas residenciais, com capacidade de reduzir a intervenção manual, otimizar o uso de produtos químicos e promover economia de água e energia, integrando sensores, atuadores e dispositivos IoT ao processo.
  
\section{Objetivos Específicos}

\begin{itemize}
\item Minimizar erros operacionais durante o processo de limpeza por meio de automação e controle preciso de dosagem química.
\item Proporcionar maior praticidade e segurança nas etapas de limpeza e monitoramento da piscina.
\item Implementar a automação dos processos de filtragem, aquecimento, medição de pH, verificação de temperatura e análise de turbidez da água.
\item Reduzir a dependência de serviços terceirizados de manutenção por meio da automação residencial inteligente.
\item Projetar e testar um módulo de monitoramento de pH com sensores IoT integrados ao sistema de controle.
\item Desenvolver e implementar um algoritmo de acionamento automático da bomba de filtragem com base nas variáveis de qualidade da água.
\end{itemize}

    \vspace{2em}

Com base nessa contextualização, observa-se que a automação aplicada à manutenção de piscinas representa um campo em expansão, mas ainda pouco explorado academicamente. Para sustentar a proposta apresentada, o próximo capítulo reúne os fundamentos teóricos sobre piscinas e seus métodos tradicionais de tratamento, os princípios da automação residencial, as tecnologias empregadas em sistemas automatizados e os componentes necessários para integrar sensores e controladores inteligentes. Essa fundamentação constitui o embasamento conceitual indispensável para compreender as etapas metodológicas e o desenvolvimento do sistema proposto.