% ----------------------------------------------------------
% Introdução (exemplo de capítulo sem numeração, mas presente no Sumário)
% ----------------------------------------------------------
\chapter{Introdução}

\textcolor{blue}{Com o avanço da tecnologia, a automação se tornou de extrema importância no aprimoramento de processos produtivos. Surgindo durante século XVIII, na 1ª Revolução Industrial e se seguindo até hoje na 4ª revolução, comumente conhecida como Indústria 4.0 \cite{automacao}. A automação integra atividades humanas desde os primórdios da civilização, evoluindo junto às necessidades sociais e tecnológicas. Dentro do setor industrial, esse avanço é bastante evidente. De acordo com a Confederação Nacional da Indústria (CNI), 72\% das empresas que adotaram a utilização de tecnologias digitais, relataram aumento da produtividade, enquanto 60\% indicaram redução dos custos operacionais como benefício direto da automação \cite{CNI}. Tais informações são de extrema importância para comprovar o impacto positivo da automação em eficiência e economia dos processos produtivos.}

Além de estar muito presente na indústria, a automação também é amplamente utilizada em tarefas do dia a dia, como em casas inteligentes com o uso de assistentes virtuais, sistemas de iluminação automatizada, controle de temperatura por termostatos, robôs aspiradores e dentre outras diversas situações que podem ser automatizadas. A automação residencial expandiu-se para além de um público especializado, consolidando-se como alternativa acessível a um número crescente de usuários, fazendo com que o mercado de automação residencial no Brasil crescesse de forma drástica, dados mostram que houveram um aumento de 21,8\% comparando o segundo trimestre de 2024 e o mesmo período em 2023 \cite{cresceuBR}. As projeções apontam para um crescimento contínuo e significativo do setor de automação residencial nos próximos anos, mostrando que o mercado global de automação residencial está crescendo cerca de 27,93\% de 2023 até 2032 \cite{aumentoAosAnos}.



\section{Objetivos}

%Nesta seção são descritos os objetivos, geral e específicos, elencados para este trabalho.

\subsection{Objetivo Geral}

Desenvolver um sistema automatizado de limpeza e manutenção de piscinas residenciais, que visa reduzir drasticamente a necessidade de intervenção manual em grande parte dos processos e gerar uma economia monetária na manutenção de piscinas residenciais. 
  
\subsection{Objetivos Específicos}

\begin{itemize}
\item Diminuir a quantidade de erros durante alguns processos de limpeza, gerando uma economia de produtos químicos e consequentemente de dinheiro.
\item Proporcionar maior praticidade nas etapas de limpeza e monitoramento da piscina.
\item Oferecer uma opção mais barata de automação para pessoas com renda mais baixa.
\item Automatizar os processos de filtragem, aquecimento, coleta de PH, verificação da temperatura e análise de turbidez.
\item Diminuição da necessidade de contratação de profissionais para realização da limpeza.
\item Desenvolver um modulo de monitoramento de pH utilizando sensores IoT.
\item Implementar algoritmo de acionamente de bomba de filtragem.
\end{itemize}
