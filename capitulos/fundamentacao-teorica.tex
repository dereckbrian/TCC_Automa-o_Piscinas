\chapter{Fundamentação teórica}
\label{cap:fundamentacao}
Este capítulo apresenta os fundamentos conceituais e técnicos que sustentam o desenvolvimento do sistema automatizado proposto. Inicialmente, descrevem-se os aspectos estruturais e os procedimentos de manutenção de piscinas, com ênfase nas práticas e parâmetros que orientam a qualidade da água. Em seguida, são expostos os princípios da automação residencial, os componentes e protocolos relevantes à integração de sensores e controladores, e, por fim, as ferramentas de hardware e software empregadas no projeto. A seleção e a organização dos tópicos procuram estabelecer a base teórica necessária para justificar escolhas de projeto e para interpretar os resultados apresentados nos capítulos subsequentes.

\section{PISCINAS E SUA MANUTENÇÃO}

A compreensão da automação aplicada à manutenção de piscinas exige o entendimento prévio de sua estrutura física, de seu funcionamento hidráulico e dos métodos tradicionais utilizados para preservar a qualidade da água. Esses elementos formam a base sobre a qual se fundamentam as etapas de monitoramento e controle automatizado, permitindo identificar quais parâmetros são passíveis de coleta e quais processos podem ser otimizados com o uso de sensores e atuadores. Para isso, esta seção apresenta um panorama histórico das piscinas, descreve seus componentes estruturais e discute os procedimentos manuais de limpeza (físicos e químicos) que orientam a manutenção convencional. A partir dessa exposição, evidenciam-se as limitações dos métodos tradicionais e a necessidade de soluções automatizadas. 

\subsection{HISTÓRICO E POPULARIZAÇÃO DAS PISCINAS}

O termo piscina deriva do latim piscis, utilizado para designar reservatórios destinados à recreação, natação, rituais ou uso coletivo \citeonline{piscinaOquee}. Existem registros de tanques de banho que remontam a aproximadamente 2600 a.C., como os “Grandes Banhos de Mohenjodaro”, cuja função possivelmente se relacionava a práticas cerimoniais. Conforme ilustrado na \autoref{fig:historia}, trata-se de uma das primeiras estruturas documentadas que apresentam características semelhantes às piscinas contemporâneas. Ao longo dos séculos, diferentes civilizações desenvolveram estruturas semelhantes, adaptadas a necessidades sociais, culturais e recreativas.

Com o avanço tecnológico do século XX, o uso de materiais como gunite (mistura de cimento, areia e água) e a incorporação de sistemas de filtração e cloração possibilitaram a manutenção contínua da água, sem necessidade de esvaziamento recorrente \citeonline{piscinaHistoria}. Tais inovações contribuíram para a popularização das piscinas residenciais e demandaram o desenvolvimento de práticas de manutenção física e química, que servem de referência para os sistemas automatizados contemporâneos.
       
\begin{figure}[H]
    \centering
    \caption{Primeiro registro histórico de piscina – “Grandes Banhos de Mohenjodaro”.}
    \label{fig:historia}
    \includegraphics[width=0.78\textwidth]{imagens/primeiraPiscina.png}
    \caption*{Fonte: \cite{piscinaHistoria}.}
\end{figure}


A seguir, apresentam-se os principais componentes estruturais que permitem o funcionamento de uma piscina moderna e que, futuramente, serão integrados à automação.


\subsection{COMPONENTES BÁSICOS DE UMA PISCINA}

Segundo \citeonline{refComponents}, as piscinas residenciais e públicas, apesar das diferenças de tipologia e porte, compartilham um conjunto de elementos funcionais que garantem a circulação, filtragem e tratamento da água. Esses componentes formam a estrutura hidráulica básica que permite a manutenção contínua da qualidade da água e o correto funcionamento de dispositivos auxiliares. A \autoref{fig:componentesPiscina} apresenta uma visão geral desses elementos estruturais, destacando a bomba, o filtro, os drenos, as tubulações e os pontos de retorno, que compõem o ciclo de movimentação e purificação do volume total da piscina.

\begin{figure}[H]
    \centering
    \caption{Componentes estruturais básicos de uma piscina.}
    \includegraphics[width=0.78\textwidth]{imagens/componentesPiscina.png}
    \label{fig:componentesPiscina}
    \caption*{Fonte: \cite{refComponents}.}
\end{figure}

A bomba motorizada cumpre a função central de impulsionar a água pelo sistema, enviando-a para as etapas de filtração e tratamento químico. Já o filtro, geralmente preenchido com areia ou elementos sintéticos, remove partículas sólidas, como poeira, folhas e microrganismos, garantindo maior transparência à água. Os drenos e skimmers são responsáveis pela coleta inicial da água, enquanto as tubulações de \textit{PVC} \footnote{Sigla para Poli(cloreto de vinila), um polímero termoplástico versátil, conhecido por sua durabilidade, resistência química e ampla utilização em tubos, conexões e revestimentos.} interligam todos os elementos hidráulicos, assegurando o fluxo contínuo. Em algumas configurações, incluem-se ainda aquecedores que regulam a temperatura da água, agregando conforto ao uso recreativo.

Além desses elementos fixos, a manutenção cotidiana envolve equipamentos manuais destinados à limpeza física da piscina. A \autoref{fig:acessoriosPiscina} reúne os principais acessórios utilizados nesse processo, como o aspirador de escova, a peneira, a escova de parede e o cabo telescópico, que permite alcançar regiões de difícil acesso. Segundo \citeonline{benedito2024projeto}, esses instrumentos são indispensáveis para remover resíduos decantados, partículas flutuantes e biofilmes aderidos às paredes da piscina, constituindo o primeiro nível de intervenção antes do tratamento químico.


\begin{figure}[H]
    \centering
    \caption{Principais acessórios utilizados na limpeza física de piscinas.}
    \includegraphics[width=0.92\textwidth]{imagens/acessoriosPiscina.jpg}
    \label{fig:acessoriosPiscina}
    \caption*{Fonte: Adaptado de \cite{benedito2024projeto}.}
\end{figure}


A compreensão desses componentes, tanto estruturais quanto acessórios, é fundamental para o desenvolvimento de sistemas automatizados, uma vez que muitos dos processos realizados manualmente, como circulação, remoção de impurezas e monitoramento da qualidade da água, podem ser otimizados por meio de sensores, atuadores e controladores eletrônicos, tema aprofundado nos tópicos subsequentes.


\subsection{NORMAS E PROCEDIMENTOS TÉCNICOS DE LIMPEZA MANUAL}
A manutenção adequada de uma piscina não depende apenas dos equipamentos estruturais apresentados na seção anterior, mas também da aplicação criteriosa de normas e procedimentos técnicos que asseguram a qualidade sanitária da água. Conforme destaca \citeonline{piscineiroProfissional}, a ausência de práticas corretas de limpeza pode resultar em diversos problemas de saúde, como dermatites, micoses e outras infecções, o que reforça a importância de um tratamento contínuo e devidamente monitorado. Para prevenir tais riscos, a Associação Brasileira de Normas Técnicas (ABNT) estabelece parâmetros essenciais que orientam o controle físico e químico da água. 

Diversos poluentes influenciam diretamente esses parâmetros, incluindo suor, urina, cabelos, óleos naturais da pele, insetos, folhas e formação de algas, conforme observa \citeonline{guiaTratamento}. Esses elementos alteram, sobretudo, o pH da água e a concentração de cloro, variáveis fundamentais para garantir segurança aos usuários. Segundo \citeonline{leite2020plataforma}, o pH é um indicador que expressa a acidez ou alcalinidade do fluido, variando entre valores menores que 7 (ácidos) e maiores que 7 (alcalinos). A \autoref{fig:phNivel} apresenta a faixa recomendada para piscinas, entre 7,2 e 7,8, parâmetro indispensável para prevenir irritações e preservar a integridade dos equipamentos.

        \begin{figure}[H]
                \centering
                \caption{Faixa de pH}
                \includegraphics[width=0.48\textwidth]{imagens/medidorPh.png}
                \label{fig:phNivel}
            	\caption*{Fonte: \cite{guiaTratamento}}
        \end{figure}

Embora a limpeza física, obtida por escovas, redes ou aspiradores, remova apenas impurezas visíveis, ela é insuficiente para eliminar substâncias dissolvidas, como suor, urina e óleos, que permanecem em suspensão ou solução e ultrapassam a capacidade de retenção dos filtros \citeonline{guiaTratamento}.Por esse motivo, a manutenção adequada deve integrar procedimentos químicos eficazes, de modo a combater microrganismos e restabelecer o equilíbrio da água.

Além disso, uma piscina considerada limpa precisa atender a critérios objetivos, como a ausência de bactérias do grupo coliforme ou \textit{Staphylococcus aureus}\footnote{Bactéria coco Gram-positiva, frequentemente encontrada na pele e nas fossas nasais humanas, responsável por infecções de gravidade variável.}
, boa visibilidade do fundo e superfície livre de sujeiras, conforme determinado por \citeonline{guiaTratamento} e pelas normas sanitárias vigentes. Esses requisitos reforçam que a manutenção envolve mais do que a aparência da água, exigindo uma análise sistemática dos parâmetros fisicoquímicos.

A determinação da quantidade de produtos necessários para o tratamento também depende do conhecimento da área e do volume da piscina, que variam de acordo com seu formato geométrico. Como apresentado nas equações desta seção, diferentes fórmulas são aplicadas em piscinas retangulares, circulares ou ovais, incluindo casos em que o fundo é inclinado, nos quais se considera a profundidade média. Esses cálculos garantem precisão na dosagem e evitam desperdícios, além de evitar desequilíbrios químicos que comprometeriam o processo de desinfecção. O dimensionamento da piscina é fundamental para determinar a quantidade adequada de produtos químicos e garantir uma higienização eficiente, sem excessos ou falhas de tratamento. O cálculo da área e do volume varia conforme o formato da piscina, sendo possível aplicar diferentes fórmulas geométricas para cada tipo de estrutura.

 \subsubsection*{\textcolor{black}{Piscina Retangular}}
        
            \[
            A = \text{comprimento} \times \text{largura}
            \]
            \[
            V = \text{comprimento} \times \text{largura} \times \text{profundidade}
            \]
            
        \subsubsection*{\textcolor{black}{Piscina Circular}}
            \[
            A = \pi r^2
            \]
            \[
            V = \pi r^2 h
            \]
            
        \subsubsection*{\textcolor{black}{Piscina Oval (Elíptica)}}
            \[
            A = \pi \cdot \frac{a}{2} \cdot \frac{b}{2}
            \]
            \[
            V = A \cdot h
            \]
            
            \textcolor{black}{Nos casos em que a piscina possui fundo inclinado, a profundidade considerada deve ser a média entre a parte mais rasa e a mais funda:}
            \[
            h_m = \frac{h_{\text{maior}} + h_{\text{menor}}}{2}
            \]

Com o volume da piscina devidamente determinado, é possível avançar para as etapas que compõem o tratamento químico da água. Segundo \citeonline{tccSilva} , o tratamento completo da água segue uma sequência de cinco etapas, oxidação, coagulação e floculação, decantação, filtração e correção do pH, semelhante aos processos utilizados em estações industriais de tratamento de água. Na oxidação, adiciona-se cloro para eliminar matéria orgânica e facilitar a remoção de metais como ferro e manganês. Em seguida, agentes coagulantes, como sulfato de alumínio ou cloreto férrico, desestabilizam partículas suspensas, que posteriormente se agrupam durante a fase de floculação. A decantação deposita essas partículas no fundo da piscina; após isso, o filtro retém as impurezas acumuladas. Por fim, realiza-se a correção do pH, garantindo estabilidade química e prevenindo tanto a corrosão das tubulações quanto danos aos usuários.

Entretanto, a execução manual dessas etapas está sujeita a falhas operacionais que comprometem a eficácia teórica do processo. Segundo \citeonline{genco2004}, a aplicação de produtos químicos sem a devida precisão frequentemente resulta em sobredosagem ou uso incorreto, gerando efeitos adversos. O excesso de algicidas, por exemplo, pode provocar espuma na superfície, enquanto o uso inadequado de decantadores à base de sulfato de alumínio pode deixar resíduos em suspensão, turvando a água. Um risco ainda mais crítico é a mistura acidental ou proposital de compostos incompatíveis, que pode provocar reações químicas extremas durante o manuseio.

Essas falhas no equilíbrio químico impactam diretamente o conforto dos usuários e a integridade da infraestrutura. A falta de controle rigoroso do pH gera desconforto imediato: valores abaixo de 7,2 causam irritação e aceleram a corrosão de equipamentos. Por outro lado, o pH acima de 7,8 reduz drasticamente a eficiência da desinfecção e favorece incrustações nas tubulações. Adicionalmente, a dosagem desbalanceada de oxidantes pode gerar subprodutos irritantes, responsáveis por odores fortes, frequentemente confundidos com excesso de produto.

No âmbito sanitário, a intermitência característica do tratamento manual cria janelas de contaminação. A ineficiência nos ciclos de filtragem e circulação, somada ao desequilíbrio químico, permite a proliferação de microrganismos e o aumento do risco de transmissão de doenças. Por fim, a imprecisão acarreta problemas estéticos, como a turbidez gerada pelo rápido desenvolvimento de algas ou alterações na cor da água devido à reação incorreta de produtos com metais presentes no tanque. 

Diante da complexidade e dos riscos apresentados pela manipulação incorreta, torna-se indispensável o conhecimento aprofundado dos insumos utilizados. A seção a seguir detalha os produtos químicos e acessórios empregados na limpeza, abordados na \autoref{subsec:produtos}, cujas características fundamentam as decisões de automação do sistema proposto.

\subsection{PRODUTOS QUÍMICOS E ACESSÓRIOS USADOS NA LIMPEZA DE PISCINAS}
\label{subsec:produtos}

O tratamento adequado e corretamente executado durante a limpeza de uma piscina, tanto físico quanto químico, é essencial para assegurar a qualidade da água e prevenir infecções ou doenças de origem hídrica. Dessa forma, é fundamental compreender quais produtos utilizar, como aplicá-los corretamente e qual o método mais eficiente para a realização do tratamento físico.

A manutenção química da água depende da correta escolha, dosagem e aplicação de produtos destinados ao controle dos parâmetros físico-químicos. Segundo \citeonline{guiaTratamento}, esses produtos atuam no ajuste do pH, na estabilização da alcalinidade, na desinfecção da água e na remoção de partículas e metais dissolvidos. A \autoref{tab:tabelaProdutos} apresenta uma tabela de dosagem que orienta a aplicação dos principais agentes químicos utilizados na rotina de manutenção.

\begin{table}[H]
    \centering
    \caption{Tabela de Dosagem de Produtos Químicos}
    \label{tab:tabelaProdutos}
    \includegraphics[width=\textwidth]{imagens/tabelaProdutos.png}
    \caption*{Fonte: \cite{guiaTratamento}.}
\end{table}

Os produtos empregados no tratamento químico incluem:

\begin{itemize}
    \item \textbf{Elevador de alcalinidade:} eleva a alcalinidade total, garantindo estabilidade ao pH.
    \item \textbf{Barrilha e pH+:} utilizados para elevar o pH quando este se encontra abaixo do ideal.
    \item \textbf{Redutor de pH (pH-):} empregados para diminuir o pH quando a água está excessivamente alcalina.
    \item \textbf{Hipoclorito de sódio, cloro, dicloro, multiação:} agentes sanitizantes responsáveis pela desinfecção.
    \item \textbf{Sulfato de alumínio e clarificantes:} agentes de coagulação e decantação.
    \item \textbf{Sulfato de cobre e algicida:} utilizados para combater a proliferação de algas.
    \item \textbf{Removedores de metais (Genquest, solução para água de poço):} eliminam manchamentos causados por íons metálicos.
\end{itemize}

A aplicação correta desses produtos depende da medição sistemática dos parâmetros da água. Para isso, utilizam-se estojos de análise específicos, capazes de identificar valores de pH, alcalinidade e teor de cloro. A Figura \ref{fig:estojoParametros} apresenta um exemplo de estojo de análise utilizado na avaliação da qualidade da água.

\begin{figure}[H]
    \centering
    \caption{Estojo para Análise de Parâmetros Químicos da Água}
    \label{fig:estojoParametros}
    \includegraphics[width=0.48\textwidth]{imagens/estojoMedidor.png}
    \caption*{Fonte: \cite{gencoEmpresa}.}
\end{figure}

Os valores obtidos nas análises orientam a escolha dos produtos a serem aplicados. Caso o pH esteja abaixo de 7,0, recomenda-se o uso de barrilha ou elevador de pH. Se a alcalinidade estiver reduzida, utiliza-se o elevador de alcalinidade. Quando o teor de cloro estiver baixo, adiciona-se cloro líquido ou granulado, conforme a dosagem apresentada anteriormente.

A turbidez é um dos principais indicadores da necessidade de tratamento químico. Em situações de água opaca, recomenda-se o uso de clarificantes. Quando a água apresenta partículas em suspensão, aplicam-se procedimentos de floculação ou decantação, geralmente realizados com sulfato de alumínio.

Após a aplicação dos produtos, é necessário aguardar entre 6 e 12 horas antes de iniciar a aspiração do fundo da piscina, garantindo assim a eficácia das reações químicas.

Com a compreensão desses produtos e procedimentos, evidencia-se a complexidade do tratamento manual, o que reforça a necessidade de soluções tecnológicas baseadas em automação residencial, assunto abordado na próxima seção.

\section{FUNDAMENTOS DA AUTOMAÇÃO RESIDENCIAL}

Após compreender os aspectos estruturais e os métodos tradicionais de manutenção de piscinas, torna-se necessário examinar os princípios da automação residencial, uma vez que o sistema proposto se insere nesse contexto tecnológico. Este tópico aborda a evolução histórica da automação aplicada ao ambiente doméstico, seus conceitos técnicos, componentes essenciais e protocolos de comunicação, que viabilizam o controle remoto e inteligente de diferentes dispositivos. Essa fundamentação permite compreender como as tecnologias emergentes podem ser aplicadas para aprimorar processos cotidianos, incluindo a manutenção automatizada de piscinas.

A automação residencial consiste na integração de sistemas tecnológicos destinados ao controle e à otimização de funções domésticas, como segurança, iluminação, climatização e comunicação. Essa integração, também conhecida como domótica\footnote{\textit{Domótica}: conjunto de tecnologias voltadas à automação e ao controle inteligente de ambientes residenciais.}, tem como propósito aprimorar o conforto, a segurança e a eficiência energética das residências \cite{automacaoResidencialCap1}.

O principal objetivo da automação residencial é proporcionar comodidade e segurança aos usuários, por meio do acionamento remoto e da integração de dispositivos inteligentes \cite{automacaoTecnologiaPraticidade}.

A automação residencial é composta por um conjunto de benefícios fundamentais que estruturam o conceito de casa inteligente. Entre seus principais pilares, destacam-se:

\begin{itemize}
    \item \textbf{Conforto:} tem como objetivo facilitar tarefas cotidianas, permitindo ao usuário controlar dispositivos como lâmpadas, ar-condicionados e sistemas de irrigação de forma remota \cite{automacaoTecnologiaPraticidade}.
    
    \item \textbf{Segurança:} a integração de câmeras, fechaduras eletrônicas e sensores de presença possibilita o monitoramento remoto da residência, reforçando a proteção e a praticidade \cite{automacaoTecnologiaPraticidade}.
    
    \item \textbf{Economia:} a automação contribui para o uso racional de energia, com sistemas capazes de desligar lâmpadas automaticamente e ajustar a climatização conforme a necessidade, evitando desperdícios e promovendo maior eficiência energética \cite{automacaoTecnologiaPraticidade}.
\end{itemize}

Para que a automação funcione de forma adequada, é necessário que os dispositivos possuam conectividade, acesso à internet e capacidade de comunicação com um sistema central de controle, responsável pela coleta e troca de informações entre os equipamentos.

Praticamente todos os aparelhos eletrônicos que possuem algum tipo de acionamento podem ser automatizados, como sistemas de iluminação, portões, climatização e segurança. Esses dispositivos são conectados a uma central de controle, que pode ser acessada por meio de um \textit{display touch}\footnote{\textit{Display touch}: superfície sensível ao toque que permite interação direta com o sistema.}, localizado na própria central, aplicativos para smartphones ou comandos de voz \cite{automacaoTecnologiaPraticidade}.


\subsection{HISTÓRICO E EVOLUÇÃO DA AUTOMAÇÃO RESIDENCIAL}

A automação residencial, embora recente quando comparada a outras áreas tecnológicas, apresenta avanços significativos ao longo das últimas décadas. Na década de 1970, surgiram nos Estados Unidos os primeiros módulos inteligentes baseados na transmissão de dados pela rede elétrica doméstica, utilizando a tecnologia PLC \textit{(Power Line Communication)} \citeonline{automacaoResidencialCap1}. Essa inovação marcou o início da integração entre dispositivos elétricos e sistemas de comunicação, permitindo que comandos simples fossem enviados por meio da fiação já existente.

Com o avanço da informática, o aprimoramento dos sistemas embarcados e a popularização da internet, a automação residencial passou a incorporar dispositivos capazes de monitorar e controlar equipamentos à distância, consolidando o conceito contemporâneo de residência conectada. Esse movimento possibilitou o desenvolvimento de soluções cada vez mais acessíveis e eficientes, ampliando o uso de sistemas inteligentes em diferentes ambientes domésticos.

A \autoref{tab:tecnologias-automacao} apresenta a evolução de algumas das principais tecnologias utilizadas na automação residencial ao longo dos anos, evidenciando o crescimento expressivo de funcionalidades como monitoramento de segurança, controle de iluminação, sistemas de áudio distribuído e gerenciamento energético.

\begin{table}[H]
    \centering
    \caption{Evolução das tecnologias de automação residencial ao longo dos anos.}
    \renewcommand{\arraystretch}{1.3}
    \setlength{\tabcolsep}{10pt}
    \begin{tabular}{lccccc}
    \hline
    \textbf{Tecnologia} & \textbf{2003} & \textbf{2004} & \textbf{2005} & \textbf{2006} & \textbf{2015(*)} \\
    \hline
    Cabeamento estruturado & 42\% & 61\% & 49\% & 53\% & 80\% \\
    Monitoramento de segurança & 18\% & 28\% & 29\% & 32\% & 81\% \\
    Multiroom audio & 9\% & 12\% & 15\% & 16\% & 86\% \\
    Home Theater & 9\% & 8\% & 11\% & 12\% & 86\% \\
    Controle de iluminação & 1\% & 2\% & 6\% & 8\% & 75\% \\
    Automação integrada & 0\% & 2\% & 6\% & 6\% & 70\% \\
    Gerenciamento de energia & 1\% & 5\% & 11\% & 11\% & 62\% \\
    \hline
    \end{tabular}
    \label{tab:tecnologias-automacao}

    \vspace{0.5em}
    \small
    \textbf{Fonte:} \cite{automacaoResidencialCap1}.\\
\end{table}

Ao compreender a evolução histórica e o crescimento das tecnologias aplicadas ao ambiente doméstico, torna-se possível identificar os fundamentos que sustentam os sistemas automatizados contemporâneos. O avanço dessas soluções evidencia uma trajetória marcada pela ampliação da conectividade, pela integração entre diferentes dispositivos e pela busca crescente por eficiência, conforto e segurança.

No próximo tópico, são apresentados os principais conceitos e componentes que constituem a base dos sistemas de automação, incluindo controladores, sensores, atuadores e protocolos de comunicação. Esses elementos são essenciais para compreender o funcionamento e a integração entre os dispositivos que possibilitam a automação em ambientes residenciais.


\subsection{CONCEITOS TÉCNICOS DE AUTOMAÇÃO RESIDENCIAL}

Tecnicamente denominada domótica\footnote{\textit{Domótica}: termo que designa a integração de tecnologias destinadas ao controle inteligente de ambientes residenciais.}, a automação residencial tem como principal objetivo acionar, monitorar, integrar e controlar diferentes variáveis de uma residência, como iluminação, climatização, áudio e vídeo, a fim de promover eficiência, comodidade e segurança ao usuário \citeonline{oliveira2019domotica}. No Brasil, o termo mais utilizado é automação residencial, uma tradução derivada da expressão americana \textit{home automation}. Apesar disso, essa tradução não abrange plenamente a concepção de domótica, que envolve não apenas o controle remoto, mas a integração inteligente de sistemas.

O uso de tecnologias residenciais tem crescido de maneira expressiva no país; entretanto, o setor da construção civil ainda não acompanha plenamente o ritmo de evolução observado em áreas como a indústria automotiva, que já emprega amplamente \textit{tecnologias embarcadas}\footnote{\textit{Tecnologia embarcada}: computador especializado, composto por hardware e software dedicados, integrados a um sistema maior para execução de funções específicas.} \citeonline{hipolito2018automaccao}. Essa diferença evidencia a necessidade de maior disseminação dos conceitos técnicos que sustentam a automação no ambiente doméstico.

De acordo com \citeonline{accardi2012automaccao}, a forma como os componentes de um sistema residencial se comunicam depende diretamente da arquitetura adotada, que pode ser centralizada ou descentralizada. Em uma arquitetura centralizada, todos os dispositivos se conectam a um único controlador principal, responsável por processar e executar todas as ações do sistema; esse arranjo é ilustrado na \autoref{fig:arquiteturaCentralizada}. Essa configuração exige que o controlador possua alta capacidade de processamento e confiabilidade operacional.

\begin{figure}[H]
    \centering
    \caption{Arquitetura centralizada de automação residencial}
    \label{fig:arquiteturaCentralizada}
    \includegraphics[width=0.53\textwidth]{imagens/arquiteturaCentralizada.png}
    \caption*{Fonte: \cite{hipolito2018automaccao}.}
\end{figure}

Na arquitetura descentralizada, diferentes controladores coexistem e se comunicam entre si por meio de um barramento de dados\footnote{\textit{Barramento de dados}: sistema de comunicação que permite a troca de informações entre dispositivos sem depender de um ponto central de controle.}, compartilhando o gerenciamento dos dispositivos interconectados; esse modelo encontra-se esquematizado na \autoref{fig:arquiteturaDescentralizada}. O modelo descentralizado distribui a responsabilidade entre múltiplos módulos, reduzindo a dependência de um único ponto de falha e aumentando a flexibilidade da instalação.

\begin{figure}[H]
    \centering
    \caption{Arquitetura descentralizada de automação residencial}
    \label{fig:arquiteturaDescentralizada}
    \includegraphics[width=0.53\textwidth]{imagens/arquiteturaDescentralizada.png}
    \caption*{Fonte: \cite{hipolito2018automaccao}.}
\end{figure}

A compreensão dessas arquiteturas é essencial para o desenvolvimento e a implementação de sistemas automatizados, pois define a maneira como os dispositivos se organizam, trocam informações e executam comandos. Nas seções seguintes, serão apresentados os principais conceitos e componentes técnicos que constituem a base dos sistemas de automação residencial, incluindo controladores, sensores, atuadores e protocolos de comunicação.



\subsubsection{COMPONENTES BÁSICOS}

A automação residencial é composta por diversos elementos que, em conjunto, permitem o controle eficiente dos dispositivos instalados no ambiente doméstico. Esses componentes vão desde sensores simples até centrais complexas de automação. A seguir, são apresentados os principais elementos que estruturam essa tecnologia.

\begin{itemize}
    \item \textbf{Camadas de dispositivos:}
    
    \begin{enumerate}
        \item \textbf{Sensores:} 
        Segundo \citeonline{leite2020plataforma}, o sensor é um dispositivo sensível ao ambiente no qual está inserido, capaz de detectar alterações em variáveis como temperatura, luminosidade ou movimento. Sua função consiste em captar essas mudanças e convertê-las em sinais elétricos que possam ser interpretados por um controlador e posteriormente utilizados pela rede de automação.
        
        \item \textbf{Atuadores:}
        São dispositivos eletromecânicos acionados pelo sistema para executar funções específicas, como ativar lâmpadas, fechaduras magnéticas, motores, válvulas ou sirenes \cite{hipolito2018automaccao}. O atuador materializa a ação física determinada pela lógica programada.
        
        \item \textbf{Controladores:}
        Responsáveis por monitorar os dados coletados pelos sensores e acionar os respectivos atuadores conforme a lógica definida. O controlador pode operar como módulo independente ou integrar-se a centrais mais complexas \cite{hipolito2018automaccao}.
        
        \item \textbf{Interfaces:}
        Dispositivos que permitem ao usuário interagir com o sistema automatizado, como painéis digitais, páginas web, aplicativos móveis ou assistentes de voz \cite{accardi2012automaccao}. Essas interfaces constituem a camada de acesso humano ao sistema.
    \end{enumerate}
    
    \item \textbf{Camada de comunicação/rede:}
    Segundo \citeonline{accardi2012automaccao}, a comunicação entre os dispositivos ocorre por meio de protocolos, que funcionam como acordos que definem regras e padrões para a troca de informações. Assim, o protocolo estabelece como os equipamentos interagem entre si dentro do sistema.
    
    Entre os protocolos mais utilizados em automação residencial estão Ethernet, X-10, HomePNA e Wi-Fi. Alguns foram desenvolvidos especificamente para ambientes residenciais, enquanto outros derivam de aplicações industriais ou comerciais.
    
    \item \textbf{Camada de controle/automação lógica:}
    Também denominada central de automação, essa camada representa o núcleo lógico do sistema, responsável por gerenciar os dispositivos conectados. Ela processa os dados recebidos e executa ações conforme as instruções programadas.
    
    A configuração da central é realizada por meio de software dedicado, acessado a partir de computadores ou dispositivos móveis. Essa estrutura é escalável\footnote{\textit{Escalável}: característica de sistemas capazes de aumentar sua capacidade ou complexidade sem perda de desempenho.}, permitindo a adição contínua de novos dispositivos conforme a necessidade de expansão.
\end{itemize}

Nesse contexto, o próximo capítulo apresenta o desenvolvimento de um sistema automatizado para limpeza de piscinas, aplicando os conceitos estudados e demonstrando como a integração entre hardware e software pode oferecer uma solução segura, eficiente e inovadora para a manutenção residencial.

\section{FERRAMENTAS UTILIZADAS PARA AUTOMAÇÃO}

Esta seção apresenta as principais ferramentas, dispositivos e tecnologias empregadas no desenvolvimento do sistema automatizado proposto, abrangendo tanto os componentes físicos de hardware quanto as plataformas de controle utilizadas na integração entre sensores, atuadores e algoritmos de decisão.

\subsection{COMPONENTES FÍSICOS E DE CONTROLE (HARDWARE)}

O sistema de automação requer componentes capazes de processar informações, interagir com o ambiente físico, executar ações mecânicas e realizar medições em tempo real. Entre esses elementos estão plataformas de prototipagem, microcomputadores, sensores especializados e atuadores eletromecânicos.

\subsubsection*{Arduino}

Segundo \citeonline{silva2025automacao}, o Arduino é uma plataforma de prototipagem eletrônica de código aberto que permite o desenvolvimento de projetos simples ou complexos de forma acessível. Seu caráter open-source possibilitou a formação de uma comunidade global de desenvolvedores, tornando-o um dos elementos centrais da cultura maker. Essa cultura, por sua vez, deriva do movimento \textit{Do It Yourself} (DIY), que incentiva a criação de projetos próprios com o uso de ferramentas acessíveis, como impressoras 3D e microcontroladores \cite{brockveld2017cultura}.

A placa Arduino é composta por diversos componentes, sendo o microcontrolador o núcleo responsável pelo processamento das instruções. Conforme destaca Massimo Banzi, um dos criadores da plataforma, o Arduino é consideravelmente menos potente que um computador convencional, porém extremamente útil para construção de dispositivos interativos dada sua simplicidade, baixo custo e modularidade \cite{silva2025automacao}. 

\begin{figure}[H]
    \centering
    \caption{Placa Arduino Uno com pinos digitais e analógicos}
    \label{fig:arduino}
    \includegraphics[width=0.65\textwidth]{imagens/arduinoDesenho.png}
    \caption*{Fonte: \cite{silva2025automacao}.}
\end{figure}

A \autoref{fig:arduino} apresenta o Arduino Uno, que possui 14 pinos digitais e 6 pinos analógicos, permitindo a conexão com uma variedade de sensores e atuadores, possibilitando a interação direta entre o sistema e o ambiente físico.

\subsubsection*{Raspberry Pi}

De acordo com \citeonline{juca2018aplicaccoes}, o Raspberry Pi é considerado um dos menores computadores do mundo, apresentando dimensões semelhantes às de um cartão de crédito. O dispositivo inclui porta HDMI\footnote{\textit{High-Definition Multimedia Interface}: interface digital para transmissão de áudio e vídeo em alta definição.}, portas USB, pinos de conexão GPIO, entrada Ethernet, conector P2 para áudio, módulo Wi-Fi e Bluetooth integrados.

\begin{figure}[H]
    \centering
    \caption{Raspberry Pi Modelo B+}
    \label{fig:raspberry}
    \includegraphics[width=0.55\textwidth]{imagens/raspberry.png}
    \caption*{Fonte: \cite{juca2018aplicaccoes}.}
\end{figure}

O hardware do Raspberry Pi funciona de maneira integrada em uma única placa, o que reduz custos e possibilita que o dispositivo execute diversas funções de um computador convencional, como acesso à internet, edição de textos e reprodução de vídeos. Além disso, seus pinos GPIO permitem interação com sensores, possibilitando aplicações em automação e sistemas embarcados.

\subsubsection*{Sensores Específicos}

Os sensores desempenham papel fundamental na automação, pois realizam a leitura de variáveis físicas ou químicas e as convertem em sinais elétricos interpretáveis pelo sistema. Entre os sensores utilizados no projeto, destacam-se:

\begin{itemize}

    \item \textbf{Sensor de Temperatura:}
    segundo \citeonline{leite2020plataforma}, identifica variações térmicas em equipamentos ou processos, permitindo ajustes automáticos conforme a condição medida.

    \item \textbf{Sensor de pH:}
    mede o nível de acidez ou alcalinidade da água, informação essencial para o controle químico em piscinas, lagos ou reservatórios \cite{leite2020plataforma}.

    \item \textbf{Sensor de Turbidez:}
    avalia o grau de turbidez de um líquido pela comparação entre um feixe de luz incidente e outro transmitido através de uma amostra \cite{cardoso2011sensor}.

    \item \textbf{Sensor de Nível:}
    de acordo com \citeonline{souza2018sensor}, detecta a altura ou volume de líquidos ou materiais granulares em um reservatório, emitindo sinais conforme o nível varia.

\end{itemize}

\subsubsection*{Atuadores (Motores)}

Os atuadores são responsáveis por executar ações físicas decorrentes das decisões tomadas pelo sistema. Em um ambiente automatizado, eles podem acionar bombas, válvulas, motores, travas ou iluminação, de acordo com os dados enviados pelos sensores ou com comandos diretos do usuário \cite{florencio2015central}.

No contexto da automação de piscinas, um exemplo de atuador é a bomba de água. Quando o sensor de nível identifica redução significativa no volume do reservatório, o controlador aciona a bomba para restabelecer o nível adequado. Da mesma forma, o usuário pode ativar manualmente o sistema por meio de uma interface digital.

As bombas de água utilizadas em sistemas residenciais funcionam, em sua maioria, de forma submersa, instaladas dentro do reservatório do filtro. Entre suas vantagens estão a facilidade de instalação e a operação silenciosa, uma vez que o corpo da bomba permanece submerso \cite{lucifabio2023aquario}.

\begin{figure}[H]
    \centering
    \caption{Bomba de água utilizada como atuador no sistema}
    \label{fig:bombaagua}
    \includegraphics[width=0.55\textwidth]{imagens/bombaAgua.png}
    \caption*{Fonte: \cite{lucifabio2023aquario}.}
\end{figure}

A compreensão dos componentes físicos utilizados no sistema permite visualizar como sensores, controladores e atuadores interagem para formar uma estrutura automatizada funcional. No capítulo seguinte, apresenta-se o desenvolvimento do sistema proposto, evidenciando a integração entre hardware e software necessária para aprimorar o processo de limpeza de piscinas residenciais.


\subsection{FERRAMENTAS DE SOFTWARE E METODOLOGIAS DE DESENVOLVIMENTO}

Linguagens de programação, segundo \citeonline{linguagemProgramacao}, constituem um conjunto de regras semânticas e sintáticas que permitem a comunicação de instruções a um computador. Por meio delas, o desenvolvedor define os dados utilizados, a forma de armazenamento e as ações que o sistema deve executar mediante condições específicas. Essas linguagens possibilitam a construção de programas, sites, aplicações móveis e diversos outros tipos de software.

No desenvolvimento de sistemas modernos, observa-se a divisão entre front-end e back-end. O front-end corresponde à camada visual e interativa com a qual o usuário mantém contato direto. Essa camada pode ser construída com HTML, CSS e JavaScript, bem como com frameworks como React, que ampliam a produtividade e a modularidade do processo de desenvolvimento \citeonline{dadesenvolvimento}. Já o back-end, conforme \citeonline{calcca2022analise}, é responsável por toda a infraestrutura lógica que sustenta as ações realizadas no front-end, incluindo autenticação, comunicação com o banco de dados e processamento de operações internas.

\subsubsection*{Java}

A linguagem Java é amplamente utilizada no desenvolvimento de aplicações web e corporativas. Classificada como orientada a objetos e multiplataforma, apresenta desempenho, segurança e confiabilidade adequados para aplicações robustas, incluindo sistemas empresariais, tecnologias de servidor, aplicações móveis e soluções voltadas para big data \cite{calcca2022analise}.

\subsubsection*{Framework e Spring Boot}

Segundo \citeonline{calcca2022analise}, um framework consiste em um conjunto estruturado de componentes reutilizáveis que oferece ao desenvolvedor uma base pré-configurada, reduzindo a necessidade de criação de código do zero. Entre os frameworks Java, destaca-se o Spring Boot, projetado para agilizar o desenvolvimento de aplicações back-end ao simplificar configurações iniciais e disponibilizar módulos integrados. O Spring Boot também possui compatibilidade com o Spring Security, o que facilita a implementação de mecanismos de autenticação e controle de acesso.

\subsubsection*{React}

De acordo com \citeonline{sousa2025desenvolvimento}, React é uma biblioteca JavaScript criada pelo Facebook, voltada para o desenvolvimento de interfaces web interativas. Sua arquitetura baseada em componentes reutilizáveis permite maior organização e reaproveitamento de código, reduzindo o tempo de desenvolvimento e promovendo maior consistência visual e funcional no front-end.

\subsubsection*{Ambiente de Desenvolvimento Integrado do Arduino}

O Ambiente de Desenvolvimento Integrado (IDE) do Arduino é responsável por compilar e enviar instruções para a placa microcontrolada. A IDE traduz o código escrito em linguagem de alto nível para instruções compreendidas pelo microcontrolador, permitindo desde operações simples, como acionar um LED, até o controle de bombas e motores a partir de dados coletados por sensores \cite{silva2025automacao}.

A interface da IDE é intuitiva, possibilitando que usuários com diferentes níveis de experiência desenvolvam seus projetos. Além disso, oferece ferramentas de compilação, verificação e comunicação com a placa, tornando-se essencial para a integração entre o ambiente digital e o físico.

\subsubsection*{Banco de Dados}

Segundo \citeonline{date2004introduccao}, um banco de dados é um sistema destinado ao armazenamento, organização e recuperação estruturada de informações. Ele permite inserir, consultar, atualizar e excluir registros conforme a necessidade da aplicação. Em um sistema de automação, o banco de dados pode armazenar parâmetros coletados por sensores, registros de operações, dados históricos e configurações do sistema, possibilitando um acompanhamento preciso do funcionamento do ambiente automatizado.

\subsubsection*{Engenharia de Software}

A engenharia de software busca desenvolver sistemas de alta qualidade de maneira eficiente e econômica, aplicando princípios de engenharia ao processo de criação, manutenção e documentação do software \citeonline{de2006introduccao}. Esse processo envolve diferentes etapas do ciclo de vida do software, desde a concepção até a implantação, que incluem análise, projeto, implementação, testes e manutenção.

Os engenheiros de software devem empregar abordagens sistemáticas e metodologias adequadas ao problema, às restrições e aos recursos disponíveis, de modo a aumentar a probabilidade de sucesso do projeto e garantir que o software atenda às necessidades dos usuários dentro do prazo e orçamento estabelecidos.

\subsubsection*{RUP}

O Rational Unified Process (RUP), segundo \citeonline{piske2003rup}, é uma metodologia estruturada de desenvolvimento de software baseada em práticas recomendadas (*best practices*) e voltada à redução de riscos durante o projeto. O RUP caracteriza-se por ser iterativo e incremental: o sistema é desenvolvido em ciclos sucessivos, nos quais novas funcionalidades são incorporadas ao produto \citeonline{moreiraprocessos}.

De acordo com \citeonline{vasconcelosconcepccao}, o RUP é composto por quatro fases principais:
\begin{itemize}
    \item \textbf{Concepção}: definição do problema, escopo e objetivos fundamentais;
    \item \textbf{Elaboração}: modelagem dos casos de uso e arquitetura inicial;
    \item \textbf{Construção}: desenvolvimento e integração do software;
    \item \textbf{Transição}: testes práticos, avaliação de desempenho e ajustes finais.
\end{itemize}

\subsubsection*{Linguagem UML}

Segundo \citeonline{castro2013paradigma}, a Linguagem de Modelagem Unificada (UML) é um conjunto de notações destinado à representação visual de sistemas de software. Sua função é auxiliar na especificação, documentação e compreensão da estrutura e do comportamento do sistema. A UML atua como meio de comunicação entre analistas e usuários, aumentando a clareza na definição dos requisitos.

Entre os diversos tipos de diagramas existentes, o Diagrama de Casos de Uso é amplamente utilizado nas etapas iniciais, por permitir a representação das interações entre os usuários e o sistema de maneira clara e objetiva.

    \vspace{2em}

 Diante das ferramentas e metodologias apresentadas, observa-se que o desenvolvimento de sistemas de automação exige a integração entre tecnologias de software, técnicas de engenharia e boas práticas metodológicas. Essa combinação permite construir soluções robustas, seguras e alinhadas às necessidades do usuário.

            