\chapter{Fundamentação teórica}

\section{PISCINAS e sua manutenção}

    \subsection{HISTÓRICO E POPULARIZAÇÃO DAS PISCINAS}

        \textcolor{red}{Foi pedido a redução desse tópico, concentrando-se na difusão residencial a partir do século XX.}
    
        Piscina que vem do latim piscis “peixe”, pode ser definida como um tanque cheio de água com inúmeros fins, sejam eles: Natação, mergulhos, saltos ornamentais ou simplesmente para fins recreativos\cite{piscinaOquee}. Com registros desde 2600 A.C "Os Grandes Banhos de Mohenjodaro" considerado um dos primeiros tanques de água pública, feito de tijolos e coberto por gesso. Contudo, acredita-se que esse tanque foi feito apenas para fins religiosos.
         \begin{figure}[H]
         	\centering
         	\caption{ }  
        	\centering
         	\label{fig:cont}
        	\includegraphics[width=0.78\textwidth]{imagens/primeiraPiscina.png}
        	\caption*{Fonte: Fibratec}
         \end{figure}

         \textcolor{green}{Removi o paragrafo que fala sobre a Grécia antiga deixando a parte sobre o século XX e deixei o primeiro.}
        
        Com o avanço da tecnologia no século 20, as piscinas foram recebendo novos sistemas, como a cloração e filtração que disponibilizavam água limpa para a piscina que anteriormente era necessário ter a troca completa da água para ser limpa. No ocidente as piscinas começaram a se popularizar com a invenção do gunite (mistura de cimento, areia e água), um material que facilitava a instalação, possibilitava projetos mais flexíveis e um custo bem mais baixo, além de ser muito utilizado também em túneis, canais, barragens, estabilização de encostas e diversas outras obras de engenharia civil\cite{piscinaHistoria}.

        \textcolor{red}{Colocar mais sobre o presente como ela se popularizou de fato}

    \subsection{COMPONENTES BÁSICOS DE UMA PISCINA}
        \textcolor{red}{Colocar os componentes básicos de uma piscina e explica-los de forma rápida}
    
        \textcolor{blue}{Segundo \cite{refComponents} as piscinas de forma abstrata, são extremamente simples, sendo apenas grandes reservatórios de água para o uso recreativo. Podendo existir diferentes tipos como: piscina de ondas do parque aquático, particular, pública e entre outras. Contudo, em sua maioria todas as piscinas possuem alguns componentes básicos que são necessários para a filtração e tratamento químico.}

        \textcolor{blue}{Diante disso, se deve ter uma mínima noção dos componentes que a mesma possui, sendo eles: bomba motorizada, filtro de água, alimentador químico, drenos, devoluções e encanamentos de PVC e em alguns casos pode haver um aquecedor a fim de manter a temperatura da piscina mais elevada.}

        \begin{itemize}
        
            \item \textcolor{blue}{Bomba Motorizada:}
                \textcolor{blue}{Responsável por circular toda a água, puxando da bacia e levando até outros processos.}

            \item \textcolor{blue}{Filtro de Água:}
                \textcolor{blue}{Remover parte das impurezas da água como: folhas, poeiras e micro-organismos.}

            \item \textcolor{blue}{Alimentador Químico:}
                \textcolor{blue}{Distribuir os produtos químicos responsáveis pela limpeza da piscina.}

            \item \textcolor{blue}{Drenos:}
                \textcolor{blue}{Removem a água utilizada para limpeza, escoamento ou manutenção.}

            \item \textcolor{blue}{Devoluções:}
                \textcolor{blue}{Pontos de reabastecimento da água da piscina.}

            \item \textcolor{blue}{Encanamentos de PVC:}
                \textcolor{blue}{Liga todos os componentes da piscina, permitindo todo o transporte da água.}
                %Colocar uma imagem de cada item respectivo ou uma no final que tenha todos
                
        \end{itemize}

        \begin{figure}[H]
         	\centering
         	\caption{ }  
        	\centering
         	\label{fig:cont}
        	\includegraphics[width=0.48\textwidth]{imagens/componentesPiscina.png}
            \caption*{Componentes básicos de uma piscina}
        	\caption*{Fonte: \cite{refComponents}}
         \end{figure}

        \textcolor{blue}{Todos esses componentes tem como objetivo bombear a água em um ciclo contínuo, passando por todos os sistemas, como filtragem, e tratamento químico, porém, ainda existem os componeres que são responsáveis por auxiliar na limpeza física.}

         \textcolor{red}{colocar os componentes para a limpeza física}
        

    \subsection{NORMAS E PROCEDIMENTOS TÉCNICOS DE LIMPEZA MANUAL}

       \textcolor{red}{Desenvolver com base nas normas da ABNT (como NBR 10339 - Instalação de piscinas) e manuais técnicos. Explicar a frequência de limpeza, produtos obrigatórios, riscos do uso incorreto, etc.}

       \textcolor{blue}{Segundo a \cite{piscineiroProfissional}, a falta de um procedimento correto de limpeza de uma piscina, pode vir a acarretar sérios problemas de saúde para aquelas que a utilizam, como: Dermatite, micose e outros. Seu tratamento deve ser constante e feito de forma eficiente, tal qual o resultado seja sempre o determinado segundo normas regidas pela ABNT. De acordo com \cite{guiaTratamento} são diversos os fatores poluentes de uma piscina, porém, é de extrema importância citar alguns deles como: Suor e urina; pelos e cabelos; óleos de pele; insetos; folhas; formação de algas; e diversos outros.}

       \textcolor{blue}{Tais poluentes podem afetar diretamente os parâmetros químicos de uma piscina, como o pH e o cloro. Segundo \cite{guiaTratamento}, efetuando uma limpeza física utilizando escovas ou redes, só é possível remover a parte visível dos poluentes, como: Folhas, insetos e lodo. Por isso surge também a necessidade de efetuar um tratamento químico efetivo, pois outros tipos de poluentes como o suor ou urina, se misturam com a água. Tal acontecimento se deve pelos agentes filtrantes serem incapazes de filtrar certos poluentes.}

       %Tentar inserir alguma imagem no final de algum desses três paragrafos pois esta tudo muito grande

       \textcolor{blue}{É de suma importância que uma piscina limpa atenda a alguns pré-requisitos, que são eles: Ausência de bactérias do grupo coliforme ou Staphylococcus aureus (Bactéria que fica alojada no nariz), uma boa visibilidade do fundo da piscina, superfície livre de sujeiras e o pH na faixa ideal entre 7,2 e 7,8. Diante disso, para se obter esses pré-requisitos exigidos, é importante que a água possua esses três princípios básicos:}
    
       \begin{itemize}
            \item \textbf{\textcolor{blue}{Água Limpa:}} \textcolor{blue}{Água transparente e sem a presença de sedimentos.}
            
            \item \textbf{\textcolor{blue}{Água Balanceada:}} \textcolor{blue}{Segundo todos os parâmetros prescritos, sem risco de prejudicar o banhista.}
             
            \item \textbf{\textcolor{blue}{Água Saudável:}} \textcolor{blue}{Livre de micro-organismos que podem vir a prejudicar o banhista.}
            
        \end{itemize}

        \begin{figure}[H]
                \centering
                \caption{ }  
            	\centering
                \label{fig:cont}
            	\includegraphics[width=0.48\textwidth]{imagens/medidorPh.png}
                \caption*{Faixa de pH}
            	\caption*{Fonte: \cite{guiaTratamento}}
        \end{figure}


        \textcolor{blue}{Antes da limpeza é importante ter o conhecimento da área e o volume da piscina, para que seja utilizado a quantidade correta de produtos para obter uma limpeza precisa e livre de desperdícios. Para calcular a área e o volume pode ser utilizado diferentes formulas dependendo do formato da piscinas, segue as formulas:}

        \subsubsection*{\textcolor{blue}{Piscina Retangular}}
        
            \[
            A = \text{comprimento} \times \text{largura}
            \]
            \[
            V = \text{comprimento} \times \text{largura} \times \text{profundidade}
            \]
            
        \subsubsection*{\textcolor{blue}{Piscina Circular}}
            \[
            A = \pi r^2
            \]
            \[
            V = \pi r^2 h
            \]
            
        \subsubsection*{\textcolor{blue}{Piscina Oval (Elíptica)}}
            \[
            A = \pi \cdot \frac{a}{2} \cdot \frac{b}{2}
            \]
            \[
            V = A \cdot h
            \]
            
            \textcolor{blue}{Nos casos em que a piscina possui fundo inclinado, a profundidade considerada deve ser a média entre a parte mais rasa e a mais funda:}
            \[
            h_m = \frac{h_{\text{maior}} + h_{\text{menor}}}{2}
            \]
            

        \textcolor{blue}{Segundo \cite{tccSilva} o processo de limpeza é similar ao utilizado em industrias como em uma Estação de tratamento de água, seguindo uma sequência de cinco etapas.}

        \begin{itemize}
            \item \textbf{\textcolor{blue}{Oxidação:}} \textcolor{blue}{Ocorre a mistura do cloro a fim de oxidar metais como ferro e manganês para facilitar a retirada de matéria orgânica}
            
            \item \textbf{\textcolor{blue}{Coagulação e Floculação:}} \textcolor{blue}{Consiste na junção de sulfato de alumínio e, esporadicamente, cloreto férrico para desequilibrar as partículas, seguindo pela circulação da água para formar flocos}
             
            \item \textbf{\textcolor{blue}{Decantação:}} \textcolor{blue}{Acontece quando as particulas coaguladas e floculadas se alojam no fundo da piscina, em razão da circulação lenta do fluido. Dependendo do produto utilizada, essa etapa pode durar cerca de 6 horas.}

            \item \textbf{\textcolor{blue}{Filtração:}} \textcolor{blue}{Contenção do acumulo de sujeira das etapas anteriores, que geralmente é feito por um filtro de areia.}

            \item \textbf{\textcolor{blue}{Correção de pH:}} \textcolor{blue}{Análise e ajuste do pH da água, geralmente utilizando um medidor para identificar o valor. O procedimento evita a deterioração dos canos.}
            
        \end{itemize}

        \textcolor{blue}{Em virtude do exposto, torna-se evidente que aplicar de forma correta os procedimentos técnicos estabelecidos é indispensável para uma melhor manutenção da qualidade da água. Neste sentido, a precisão e a segurança demandadas pelo processo dependem diretamente da escolha e do manuseio adequado dos produtos químicos, tema que será abordado na próxima seção.}


    \subsection{PRODUTOS QUÍMICOS E ACESSÓRIOS USADOS NA LIMPEZA DE PISCINAS}
        \textcolor{red}{Manter. Expandir explicando pH ideal, uso de cloro, algicidas, floculantes, entre outros}

        \textcolor{blue}{O tratamento preciso e bem executamo na hora da limpeza de uma piscina, tanto o físico como químico, garantem uma boa qualidade da água, a fim de evitar possíveis infecções ou doenças transmitidas pela água. Por isso, é importante entender quais produtos utilizar e como utilizar, bem como entender a melhor forma de efetuar o tratamento físico. O principal objetivo dessa seção é mostrar quais produtos e ferramentas se deve utilizar para garantir uma boa qualidade da água e evitar possíveis doenças para o usuário.}

        \begin{itemize}
        
            \item \textbf{\textcolor{blue}{Precedimento Químico:}} \textcolor{blue}{Segundo \cite{guiaTratamento}, é todo a parte que envolve a adição de produtos químicos na água para garantir sua qualidade e evitar riscos a saúde dos usuários, ajustando a alcalinidade o pH e também garantindo a desinfecção da água pelos micro-organismos e bactérias utilizando cloro, bem como outros produtos com o intuído de tratar outros parâmetros. Abaixo está uma tabela com os produtos e a quantidade adequada que deve ser colocada a cada mil litros.}

                \begin{figure}[H]
                    \centering
                    \caption{ }  
                	\centering
                    \label{fig:cont}
                	\includegraphics[width=1.00\textwidth]{imagens/tabelaProdutos.png}
                    \caption*{Tabela de Dosagem de Produtos}
                	\caption*{Fonte: \cite{guiaTratamento}}
                \end{figure}
    
                \begin{itemize}
    
                    \item \textbf{\textcolor{blue}{Elevador de Alcalinidade:}} \textcolor{blue}{A alcalinidade da água está relacionado com a sua capacidade de neutralizar ácidos, se comportando como uma especie de contenção para manter o pH estável. O elevador de alcalinidade tem como objetivo elevar a alcalinidade para o nível ideal que é entre 80 a 120 ppm.}
    
                    \item \textbf{\textcolor{blue}{Barrilha, Elevador de pH, pH+:}} \textcolor{blue}{Produtos com sua composição alcalina que tem como principal função elevar o pH quando baixo.}
    
                    \item \textbf{\textcolor{blue}{Redutor de pH, pH-:}} \textcolor{blue}{Composição ácida que tem a função de diminuir o pH}
    
                    \item \textbf{\textcolor{blue}{Hipoclorito de Sódio, Cloro, Dicloro, Multiação 3x1:}} \textcolor{blue}{Produtos Sanetizantes que basicamente tem como objetivo eliminar os micro-organismos na água.}
    
                    \item \textbf{\textcolor{blue}{Sulfato de Alumínio, Clarificantes:}} \textcolor{blue}{Faz as partículas de sujeira presente na piscina passem por um processo chamado decantação que em resumo, aglomera as partículas e as leva para o fundo da piscina, facilitando o processo de aspiração e filtração.}
    
                    \item \textbf{\textcolor{blue}{Sulfato de Cobre, Algicida:}} \textcolor{blue}{É utilizando quando a piscina chega no processo de esverdeada, eliminando algas e o lodo.}
    
                    \item \textbf{\textcolor{blue}{Genquest, Solução Água de poço:}} \textcolor{blue}{Remove manchas e cores de metais dissolvidos na água da piscina.}
    
                \end{itemize}
                
            \item \textbf{\textcolor{blue}{Medição de Parâmetros e Ajuste do pH:}} \textcolor{blue}{Para medir os parâmetros deve-se utilizar estojos de analise do respectivo parâmetro que vai ser analisado. Ajustar os parâmetros são o ponto chave para um tratamento eficiente da água. Segue um exemplo de estojo para a analise de diferentes parâmetros como o pH, cloro e Alcalinidade.}

            \begin{figure}[H]
                    \centering
                    \caption{ }  
                	\centering
                    \label{fig:cont}
                	\includegraphics[width=0.48\textwidth]{imagens/estojoMedidor.png}
                    \caption*{Estojo para Análise de Parâmetros}
                	\caption*{Fonte: \cite{gencoEmpresa}}
            \end{figure}

            %Colocar imagens de medidor
            \textcolor{blue}{Os produtos a serem utilizados dependem da alteração e do parâmetro alterado, caso o pH apareça abaixo de 7, se deve usar o elevador de pH ou o barrilha, se a Alcalinidade constar abaixo do ideal, é necessário utilizar o elevador de Alcalinidade, e por fim, se o cloro estiver baixo também, será necessário aplicar cloro liquido ou granulado, como consta na tabela mostrada anteriormente.}


            \item \textbf{\textcolor{blue}{Limpeza Fisica:}} \textcolor{blue}{Dado as informações fornecidadas até o presente momento, deve-se ter um entendimento sobre a turbidez da água, que é a presença de particulas em suspensão na água. Para fazer o devido tratamento, é recomendado fazer o uso fazer o uso de um decantador que aglomera as particulas de sujeira levando-as para o fundo da piscina para que possa ser feita a devida aspiração. Dito isso, existes algumas maneiras de resolver esse problema, são elas:}
                \begin{itemize}
                    \item \textbf{\textcolor{blue}{Clarificação:}} \textcolor{blue}{teste}
                    \item \textbf{\textcolor{blue}{Decantação ou Floculação:}} \textcolor{blue}{teste}
                \end{itemize}

            
                        
        \end{itemize}
         
   
\section{FUNDAMENTOS DA AUTOMAÇÃO}
    \textcolor{blue}{Uma abordagem sucinta para se ter uma melhor contextualização e entendimento sobre a automação, é que a automação se consistem em basicamente todo sistema que por meio da entrada de dados, o próprio execute uma ação de forma autônima inerente ao dado que foi fornecido, desde que sejam ações isentas de qualquer tipo de interferência humana.}

    \textcolor{blue}{De acordo com \cite{guiaAutomacao}, a automação pode seguir diversos ramos do mercado, existindo assim diversas variações da tal, sendo algumas delas: }

    \begin{itemize}
            \item \textbf{\textcolor{blue}{Automação industrial:}}
               \textcolor{blue}{ A automação industrial se consiste na plena escolha de uma tecnologia que se melhor ajusta ao ciclo de produção de uma industria, visando sempre assegurar o custo-beneficio. Dentro do contexto da automação industrial, o conceito é basicamente tornar automático um processo caracterizado como industrial, fazendo o constante uso de dispositivos como: Atuadores, sensores e controladores. Exemplos de processos caraterizados como industriais são fábricas de automóveis e produtoras de biodiesel\cite{livroAutomacao}. Dentro de uma industria automatizada existem alguns níveis, que são eles:}
                
                    \begin{itemize}
                        \item \textbf{\textcolor{blue}{Nível de campo:}} \textcolor{blue}{É onde se encontram os dispositivos de campo, tanto de entrada como de saída. Sendo alguns deles os sensores, atuadores, válvulas, etc.}
                        
                        \item \textbf{\textcolor{blue}{Nível de controla:}} \textcolor{blue}{Formado por Controladores Lógicos Programáveis (CLPs), e onde toda a lógica do que cada dispositivo de saída deve fazer de acordo com o que foi recebido pelo de entrada\cite{clps}.}
                        
                        \item \textbf{\textcolor{blue}{Nível de supervisão>:}} \textcolor{blue}{Constituído geralmente por monitores junto a softwares que garantem a interface homem-máquina para que tudo possa ser monitorado.}
                    \end{itemize}
              
            \item \textbf{\textcolor{blue}{Automação comercial:}}
                \textcolor{blue}{Consiste em melhorar os processos internos de uma empresa, diminuindo de maneira considerável os erros que possam a ser cometidos pelos funcionários, bem como: erros em cálculos, digitação, planilhas, e entre outros. Dessa forma a empresa consegue passar mais credibilidade com relação aos seus dados\cite{automacaoComercial}.}

            \item \textbf{\textcolor{blue}{Automação residencial:}}
               \textcolor{blue}{ Envolve o uso da tecnologia em prol do melhoramento da segurança, otimização de atividades e até um maior conforto dentro de uma residência. Exemplos:}

                \begin{itemize}
                        \item \textbf{\textcolor{blue}{Controle de acesso por biometria;}} 
                        \item \textbf{\textcolor{blue}{Controle de luminosidade dos vários ambientes;}} 
                        \item \textbf{\textcolor{blue}{Portaria automática;}}
                        \item \textbf{\textcolor{blue}{Controle de temperatura;}}
                    \end{itemize}
                
                
        \end{itemize}

    \subsection{HISTÓRICO E EVOLUÇÃO DA AUTOMAÇÃO}

        \textcolor{red}{Manter com cortes de trechos repetitivos. Deixar conciso}
    
        O ser humano desde os primórdios da civilização, vem procurando meios de como automatizar suas tarefas, reduzindo bastante seus esforços físicos com a utilização de ferramentas desde a pré-história. Exemplos antigos como a roda e os moinhos movidos a força natural utilizando animais ou até mesmo o vento, revelam um instinto ancestral por automatizar processos e demonstra um enorme impulso por eficiência. Avanços mais significativos começam a aparecer no século XVIII durante a primeira Revolução Industrial na Inglaterra.  Esse período é marcado pela maneira como as mercadorias são desenvolvidas, devido ao advento das máquinas a vapor capazes de operas com maior precisão do que um ser humano operaria manualmente. Diante disso, a utilização de máquinas a vapor gerou um enorme impacto no mercado mundial, começando um processo de substituição de força manual e hidráulica que até então era predominante.

        Durante o século XIX, como o advento da energia elétrica, novas tecnologias surgiram no setor de comunicação como o telegrafo e o telefone, permitindo com que a tecnologia e a automação se aproximasse cada vez mais do usuário convencional e não apenas centralizada nas grandes industrias\cite{adventoMidias}. Com o forte aumento na industrialização também vieram as grandes melhorias nos sistemas de transporte, incluindo locomotivas a vapor, expansão de rodovias e embarcações mais modernas.

        Já no século XX com um enorme salto na tecnologia junto a enorme utilização da energia elétrica nas indústrias para alimentar as máquinas, deixando todo o processo mais rápido e flexível. Por consequência a todos esses avanços da automação nas indústrias, foi permitido a criação dos circuitos integrados, computadores, servomecanismos e controladores programáveis que começaram a fazer parte da automação inaugurando uma nova era no que se diz respeito a processos autônomos \cite{maisSobreXX}. Todo o progresso tecnológico alcançado nesse percurso, demonstra que o desejo de automatizar tarefas não é um fenômeno recente, mas sim uma característica enraizada na trajetória da humanidade\cite{historiaAutomacao}. 
        
        Dessa forma, observa-se que a tecnologia e a automação não são exclusividades do século XXI, tendo em vista que sua presença e evolução vêm sendo registradas desde tempos remotos. Seguem algumas delas:
        
        \begin{itemize}
            \item \textbf{Relógio de água (4.000 AC):}
                Também conhecido como clepsidra, o relógio de água é considerado um dos primeiros métodos de automação e medição de tempo da história, com sua origem no Oriente Médio e na Ásia. Seu funcionamento consistia em um recipiente superior com um pequeno orifício em sua parte inferior e um recipiente inferior que tinha marcações internas já calibradas. Bem parecido com uma ampulheta o recipiente superior era preenchido com água, e a medida que o tempo passava a água escoava ate o inferior, e se utilizava das marcações para saber a passagem do tempo\cite{automacaoAntiga}.
              
            \item \textbf{Mecanismo de Antikythera (Por volta de 150 a.C.):}
                Descoberto em 1901 nos destroços de um naufrágio perto de uma ilha grega, o Mecanismo de Antikythera foi um dos artefatos tecnológicos mais impressionantes da antiguidade já descobertos. Sendo basicamente um computador analógico, seu objetivo era prever movimentos de corpos celestes, composto por pelo menos 30 engrenagens de bronze. Suas principais funções eram prever a movimentação do sol, lua e possivelmente de alguns planetas conhecidos, além de também mostrar os ciclos lunares e prever eclipses.\cite{automacaoAntiga}.

            \item \textbf{Prensa de Gutenberg (1454):}
                Anteriormente ao século XV, fazer a cópia de um livro ou panfleto demandava um trabalho enorme, visto que todos os escritos precisavam ser feitos a mão. Em 1454, isso muda com a criação da Prensa de Gutenberg, uma invenção que revolucionou a forma de viver das pessoas na época. Seu funcionamento baseava-se na organização de letras e pontuações em pequenos blocos de metal, chamados tipos móveis. Esses tipos eram organizadas de acordo com a palavra ou texto que seria escrito, formando uma página. Depois que colocado a tinta sobre a matriz de tipos, ela era prensada de forma uniforme por uma máquina sobre o papel, copiado completamente o texto de forma muito mais eficiente do que era feito anteriormente.\cite{automacaoAntiga}
                
        \end{itemize}

    \subsection{CONCEITOS TÉCNICOS DE AUTOMAÇÃO}

        \textcolor{red}{Reformular trechos. Sugerir mova estrutura: Sensores, atuadores, controladores (CLPS, ESP32, arduíno, etc.), softwares e interfaces (ex: Integração com apps/web, Iot)}
    
        De acordo com \cite{livroAutomacao}, a automação pode ser definida como o processo em no qual um controlador realiza uma tarefa mediante a um comando previamente estabelecida por um software que o controla. Dentro do contexto de automação industrial, o conceito é basicamente a utilização de dispositivos de entrada e saída de dados (sensores e atuadores), com suas variáveis de processos constantemente sob observação, a fim de garantir que o resultado desejado seja obtido, ou seja, todo o processo visa a garantia de que com a variável coletada pelo sensor seja interpretada da forma correta pelo atuador dentro dos limites preestabelecidos. 
        
        Um cenário bem simples a ser compreendido, é um sensor de temperatura conectado a um ar-condicionado, com a utilização de um software é estabelecido que a temperatura fique entre 22 e 27 graus. Nesse a variável de processo é a temperatura que não pode ultrapassar e nem reduzir a temperatura previamente estabelecida, caso ocorra o sistema responsável irá decidir o que será feito com o ar para que a temperatura volte a ficar dentro dos limites que foram estabelecidos.

        \subsubsection{SENSORES}
            Sendo completamente indispensável, um sensor é o dispositivo que faz o intermédio entre a informação coletada de alguma variável física do processo (temperatura, pressão, etc), e o seu objetivo final. Seja esse objetivo apenas o monitoramento de alguma dessas variáveis ou uma mudança completamente autônoma do processo com base na informação fornecida pelo sensor.

            Dado o conceito do que é um sensor, deve-se entender que existem diversos tipos de saídas diferente e variações de sensores. Suas saídas podem ser analógicas e digitais, sendo a analógica uma saída continua e proporcional a grandeza medida, já a digital é uma saída discreta que representa apenas dois estados chamados de booleanos (0 ou 1). Exemplo de sensor analógico e digital é o de pressão e de presença\cite{sensor}.
            
        \subsubsection{ATUADORES}
            Atuador tem por definição algo que atua sobre outra coisa. Segundo o \cite{atuadoresOutros} pode também ser definido utilizado o ser humano como analogia, sendo o atuador, os bracos ou músculos no geral que atuam sobre determinado elemento, no contexto de automação, atuando sobre variáveis manipuladas do sistema que deve ser controlado. 
            
            De forma mais técnica pode ser definido como um servomecanismo, que com a utilização de energias de líquidos, gases e entre outros, fornecem e transmitem força mecânica (movimento) para o funcionamento de outro dispositivo. Podendo ser divididos quase sempre em três tipos: Hidráulicos que utilizam da energia gerado por líquidos, em sua maioria água ou óleo. Os Pneumáticos, utilizando gases que quase sempre são o ar-comprimido e por fim o elétrico que utiliza a energia elétrica \cite{atuadoresAutomacao}.

        \subsubsection{CONTROLADORES}
            Aqui eu explico o que são os controladores

        \subsubsection{SOFTWARES E INTERFACE}
            Aqui eu explico os softwares.

    \subsection{ARQUITETURA DE AUTOMAÇÃO RESIDENCIAL}

        \textcolor{red}{Propor inclusão: Esquemas arquiteturais comuns, como barramento, sistemas embarcados, gubs de automação}
    
        Dentre todos os diversos tipos de automação presentes no mercado, como a automação no transporte, saúde, marketing e comercial, a automação residencial vem se destacando muito ao longo dos anos obtendo cada vez mais investimentos e adesão pelas pessoas. Com o desenvolvimento de ferramentas como aspiradores automáticos, assistentes virtuais (como a Alexa), termômetros inteligentes e entre outros, a automação residencial tem se mostrado uma solução inovadora. Esses dispositivos têm como principal objetivo melhorar a qualidade de vida das pessoas, proporcionando conforto, praticidade e economia de tempo. Além disso, contribuem significativamente para aumento da segurança com o uso de câmeras mais inteligentes e outros dispositivos, eficiência energética e uma série de outros benefícios\cite{automacaoResidencial}.

        %Crescente (melhorar os dados depois)
        Acredita-se que um dos principais motivos do crescimento da automação residencial, além da constante e óbvia evolução do mercado de tecnologia, foi o Covid-19. O impacto da pandemia no mundo foi muito grande, e na automação residencial não foi diferente, testemunhando um crescimento de cerca 11,9\% em 2020, algo fora da curva se comparar com crescimento médio ano a ano entre 2017 e 2019\cite{automacaoCovid}(verificar se a fonte ainda pode ser aberta). Por consequência três anos depois em 2023 o tamanho do mercado global de automação residencial foi de US\$ 90,75 bilhões, em 2024 foi de US\$ 101,79 bilhões e existem projeções que até 2032 será de US\$ 237,07 bilhões \cite{automacaoDepoisCovid}.
        
    
    \subsection{APLICAÇÃO DA AUTOMAÇÃO RESIDENCIAL}
        \textcolor{red}{Expandir com ênfase em: Eficiência energética, conforto e segurança, dispositivos acessíveis (relevante ao projeto do aluno)}
    
        
\section{AUTOMAÇÃO DE PISCINAS}

    Contexto mais geral sobre a automação de piscinas, como ela funciona e como surgiu as ideias e as primeiras e como se encontra o mercado hoje

    \subsection{CONCEITO E FUNCIONAMENTO GERAL}
        Aqui eu colo o processo de limpeza de uma piscina automatizada, listando vários tipo de automação e incluindo o do meu projeto(talvez).

    \subsection{DIFERENÇAS ENTRE PROCESSOS MANUAIS E AUTOMATIZADOS}

        \textcolor{red}{Desenvolver com base em estudos técnicos (aprodutos existentes como sodramar, nautilus, etc.) Citar exemplos de bombas inteligentes, ozonizadores, sensores de pH automatizados.}
    
        Aqui como mais acima eu já expliquei sobre a automação manual e automatizada, eu rapidamente relembro e depois discorro pontos positivos e negativos de ambas com o fim de comparação, sempre tentando enaltecer a automatizar, afinal é de fato melhor e é isso que eu pretendo provar.
    
    \subsection{TECNOLOGIAS ESPECÍFICAS USADAS NA AUTOMAÇÃO DE PISCINAS}
        \textcolor{red}{Novo subtópico importante, aqui o aluno pode falar sobre: 
        Sensores de ph e orp
        medidores de turbidez
        bombas peristalticas para dosagem automática
        controladores (espn32, raspberry Pi, etc...)}

    \subsection{ASPECTOS SANITÁRIOS E SAÚDE PÚBLICA}
        \textcolor{red}{Ampliar com base na vigilância sanitária e riscos da má manutenção de piscinas. Sugestão: normas ANVISA ou artigos sobre dermatites, otites, doenças bacterianas e parasitárias}

    \subsection{ACESSIBILIDADE E DEMOCRATIZAÇÃO DA AUTOMAÇÃO}
       %\textcolor{red}{Reescrever com liguagme técnica e respeitosa. Sugerir expressões como: "Viabilidade técnica e econômica para população com menor poder aquisitivo", "proposta de custo acessível para residências de médio padrão"}


\section{INTERNET DAS COISAS (IoT) APLICADA A AUTOMAÇÃO RESIDENCIAL}
    \textcolor{red}{Justificativa: se o projeto utiliza sensores e dados monitoráveis remotamente, é essencial explicar conceitos como:
        comunicação wifi ou bluetooth
        sensores inteligentes
        integração com aplicativos
        segurança de dados}

\section{SUSTENTABILIDADE E EFICIÊNCIA NA AUTOMAÇÃO DE PISCINAS}
    \textcolor{red}{justificativa: reforça o benefícios a automação, incluindo:
        economia de água e energia
        dosagem precisa de químicos
        redução de impacto ambiental}

