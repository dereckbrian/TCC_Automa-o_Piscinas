% resumo em inglês
\begin{resumo}[Abstract]

\begin{otherlanguage*}{english}

Considering the repetitive work involved in the development of the Teacher Work Plan (TWP) in the Federal Institute of Mato Grosso do Sul (IFMS), an automated method is proposed for building an optimized TWP by using constraint programming.

The method is implemented with the Google OR-Tools library and is modeled after on built in mandatory constraints and teacher preferences that are described as the program input.

A graphical interface was developed using JavaScript, JQuery, HTML and CSS to make the system more user-friendly.

The method was evaluated using a real teacher situations, where an idea of the efficiency and performance of the method could be evaluated. Results showed that the method was able to build the timetable according to the rules and preferences of the teacher, minimizing the exchange of activities and free time between them.

		\vspace{\onelineskip}
		
		\noindent 
		\textbf{Keywords}: constraint programming, scheduling, optimization.
	\end{otherlanguage*}
\end{resumo}