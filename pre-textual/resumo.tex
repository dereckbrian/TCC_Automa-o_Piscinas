\setlength{\absparsep}{18pt} % ajusta o espaçamento dos parágrafos do resumo
\begin{resumo}
    A crescente aplicação de tecnologias de automação em ambientes residenciais tem impulsionado o desenvolvimento de sistemas voltados à otimização de tarefas cotidianas. Neste contexto, o presente trabalho propõe o desenvolvimento de um sistema automatizado para limpeza e manutenção de piscinas residenciais, fundamentado em princípios de automação residencial e na Internet das Coisas (IoT).

    sistema integra sensores, atuadores e controladores microprocessados com o objetivo de realizar o monitoramento e o tratamento automatizado da água, reduzindo a necessidade de intervenção manual.
    
    A proposta busca oferecer maior eficiência no uso de recursos, segurança na manipulação de produtos químicos e sustentabilidade no consumo de água e energia. A pesquisa abrangeu a revisão de normas técnicas, o estudo dos componentes mecânicos e eletrônicos empregados, bem como o desenvolvimento de um protótipo funcional.
    
    Os resultados obtidos demonstram que a automação do processo de limpeza e tratamento de piscinas é viável e pode minimizar falhas humanas, otimizar o tempo de manutenção e garantir padrões adequados de qualidade da água. O sistema desenvolvido apresenta-se, portanto, como uma solução prática e acessível, alinhada às tendências tecnológicas de domótica e automação inteligente.

	\textbf{Palavras-chave}:  Automação residencial. IoT. Piscinas residenciais. Manutenção automatizada.
\end{resumo}