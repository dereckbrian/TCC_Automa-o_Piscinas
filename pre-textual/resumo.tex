\setlength{\absparsep}{18pt} 
\begin{resumo}
    A crescente incorporação de tecnologias de automação em ambientes residenciais tem impulsionado o desenvolvimento de sistemas voltados à otimização de tarefas cotidianas. Nesse contexto, este trabalho apresenta o desenvolvimento de um sistema automatizado para limpeza e manutenção de piscinas residenciais, fundamentado nos princípios da automação residencial e da Internet das Coisas (IoT).O sistema integra sensores, atuadores e controladores microprocessados com a finalidade de monitorar e tratar a água de forma automatizada, reduzindo significativamente a necessidade de intervenção manual. A proposta busca ampliar a eficiência no uso de recursos, aumentar a segurança na manipulação de produtos químicos e promover práticas sustentáveis no consumo de água e energia.A metodologia adotada envolveu a revisão de normas técnicas, o estudo dos componentes mecânicos e eletrônicos empregados, bem como o desenvolvimento e validação de um protótipo funcional. Os resultados indicam que a automação dos processos de limpeza e tratamento de piscinas é tecnicamente viável, contribui para a redução de falhas humanas, otimiza o tempo de manutenção e assegura padrões adequados de qualidade da água.Conclui-se que o sistema desenvolvido configura-se como uma solução prática, acessível e alinhada às tendências contemporâneas de domótica e automação inteligente.

    \vspace{\onelineskip}
		
		\noindent 
		\textbf{Palavras-Chave}: Automação Residencial. IoT. Piscinas Residenciais. Manutenção Automatizada.
	
\end{resumo}