\setlength{\absparsep}{18pt} % ajusta o espaçamento dos parágrafos do resumo
\begin{resumo}
    Observando o trabalho repetitivo envolvido na construção do Plano Individual de trabalho Docente (PIT) no Instituto Federal de Mato Grosso do Sul (IFMS), foi proposto um método para construir e otimizar o plano de trabalho semanal dos professores usando o paradigma de programação por restrições.
    
    O método, implementado utiliza a biblioteca Google OR-Tools - que é baseado em restrições obrigatórias e preferências do professor que são descritas em formulários na interfáce.
    
    A interface gráfica utiliza JavaScript, JQuery, HTML e CSS para tornar o sistema mais amigável ao usuário.
    
    O método foi avaliado utilizando a situação de professores, onde a eficiência e velocidade do método pode ser avaliada. Resultados mostraram que o método conseguiu construir o horário de acordo com as regras e preferências do docente, minimizando as trocas de atividades e o tempo livre entre estas.

	\textbf{Palavras-chave}:  Programação por restrições, escalonamento, otimização.
\end{resumo}